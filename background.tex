\section{Background and Motivation} \label{sec:background}

What do people who talk about `comprehensible systems', `interpretable learning', `opaque systems', and `explainable AI' really care about? This paper tries to provide a more formal look at what components of a user's trust \edit{assurances for intelligent systems} would want to influence \edit{and how this can be practically accomplished by intelligent systems}. \nisarcomm{feels like more belongs here, specifically: an overview/summary of the main arguments and key takeaways from each of the following subsections in this section: a paragraph at this part should basically outline the skeleton of the key questions and ideas that the reader should be looking out for. Again, this can be written after the rest of the section itself is more solidified...}

\subsection{Motivation}
    Generally, humans have always wanted to trust the tools and systems that they create.  To this end many metrics and methods have been created to help assure the designers and users that the tools and systems are in fact capable of being applied in certain ways, and/or behave as expected. Of course, at this point `trust' is quite an imprecise term, and needs to be more formally defined; this will be done in section \ref{sec:trust}

    The situation has grown more complicated in recent years because the advanced capabilities of the systems being created can at times be difficult for even those who designed them to comprehend and predict. There are, for instance, systems that have been designed to learn from extremely large amounts of data and expected to regularly perform on never before seen data. In some cases, such systems have been designed to perform tasks that might take humans entire lifetimes to complete. 

    Below is a small sample of some application areas that exist and a possible reason why they -- perhaps unknowingly -- have an interest in creating trustworthy systems.
    
	% \begin{sidewaysfigure}[htbp]
        % \includegraphics[width=7.5in]{Figures/WhoCares_cleaned}%
        % \caption{A diagram showing some of the academic disciplines that want to trust AIs more fully}
        % \label{fig:WhoCares}
    % \end{sidewaysfigure}
%
    \begin{description}
        \item [Interpretable Science:] Scientists need to be able to trust that the models created using data analysis, and be able to draw insights from them. Scientific discoveries cannot depend on methods that are not understood.
        \item [Reliable Machine Learning:] It is critical to have safety guarantees for AIAs that have been deployed in real-world environments. Failing to do so can result in serious accidents that could cause loss of life, or significant damage.
        \item [Artificial Intelligence/Machine Learning:] There is a need to interpret how and why theoretical AIA models function. This is due to the need to know they are being applied correctly, but also to be able to design new methods to overcome weaknesses in the existing methods.
        \item [Government:] Governments are beginning to enforce regulations on the interpretability of certain algorithms in order to ensure that citizens can understand why many services make the decisions and predictions that they do. A specific example are the algorithms deployed by credit agencies to approve/reject loans.
        \item [Medicine:] Medical professionals need to understand why data-driven models give predictions so that they can choose to follow the recommendations. While AIAs can be a very useful tool, ultimately doctors are liable for the decisions they make and treatments they administer.
        \item [Cognitive Assistance:] Systems are being designed as aids for humans to make complex decisions, such as searching and assimilating information from databases of legal proceedings. When an AIA presents perspectives and conclusions as summaries of this data, it must be able to also present evidence and logic to justify them.
    \end{description}

    The interests of the author lie specifically in the design of unmanned vehicles that operate in concert with human operators in uncertain environments. In this setting, it is desirable for the unmanned vehicle to be able to communicate with a human in some way in order to help them properly use the vehicle. The hope is that, in doing so, the performance of the team can be improved by appropriately utilizing the strengths of both the human and unmanned vehicle.

    Revisiting a key point from the introduction, humans want to design assurances to help them appropriately trust AIAs. There are a few research fields that have formally and explicitly considered trust between humans and technology. Some examples are: e-commerce, automation, and human-robot interaction. However, due to their main goals they have mainly focused on implicit properties of the systems that affect trust.

    Conversely, there are other research fields that have informally considered how to affect the trust of designers and users via explicitly designed assurances. However, due to their informal treatment of trust, it is unknown and unclear how effective these designed assurances might be, or what principles ought to be considered when designing assurances for general AIAs.

    \section{Methodology} \label{sec:methodology}
    In this survey, I attempt to look at research of those who are formally and informally addressing the idea of human-AIA trust. In particular, I focus on a ideas that might be applicable to the trust relationship between a single human user (User), and a single autonomous vehicle. While theoretically a two-way trust model could be considered, I will only be considering a one-way trust relationship, that is that the autonomy has perfect trust towards a user.

    It should be noted that it is almost impossible to perform a comprehensive survey of all assurances due to the broad nature of assurances in general. One could rightly argue that metrics like gain and phase margins are assurances for control engineers, as are training and test accuracy for machine learning practioncioners. However, it is my opinion that the somewhat narrow view of the surveyed literature does not significantly hinder the definition or classification of assurances.

    In order to find applicable research I first looked at papers that formally addressed trust and tried to create models of it; this with the aim of trying to understand how it might be influenced. Secondly, I looked at some historical research regarding trust between humans and some form of non-human entity. This mainly lead to e-commerce literature, automation literature, and human-robot interactions. Third, I investigated work regarding `interpretable', `comprehensible', `transparent', `explainable', \ldots and other types of learning and modeling methods. Finally, I searched for research disciplines that are investigating methods that would be useful as assurances, but of which trust is not the main focus.

    With this information, I try to make an informed definition and classification of assurances based off of empirical information of methods that are currently in use or being investigated. In doing so I was able to identify several areas that are open for further research. 


\subsection{Related Work}
    \citet{Lillard2016-yg} addressed the role of assurances in the relationship between humans and AIAs, and provides much of the foundation for describing the relationships between assurances and trust in human-AIA interaction. Here, the assurance framework is presented in a way that is both more general and more detailed, albeit with the same end goal of being applied in a very similar end application. 
    
    For instance, instead of assuming that the user-AIA trust loop is very strict and well structured, and that user trust does not take into account the institutional component proposed by \citet{McKnight2001-fa}, we account for the notion of institutional trust in  human-AIA relationships.

    Regarding the relationship with the work of \citet{McKnight2001-fa}, who constructed a typology of interpersonal trust,  we adopt the position that besides being applied to the e-commerce industry their trust model also applies to relationships between humans and AIAs (as in \citet{Lillard2016-yg}). However, here it is argued that assurances cannot directly have an affect on the user trust-related behaviors (TRBs) The premise is that no autonomy (or vendor) should be able to control the TRBs of a human.

    The contributions of this work are to illuminate the connection between general AIA capabilities and general user trust. To classify assurances. To suggest that TRBs need to be calibrated \emph{not} trust. \textbf{add more here} \nisarcomm{not sure what last sentence means??}

\subsection{Motivating Application and Basic Definitions}
    Before continuing to the detailed definitions it is useful to have a concrete example on which we can refer for grounding examples. As previously mentioned the specific interests of the author lie in the design of unmanned vehicles that can work in cooperation with human operators.

    Specifically, consider an unmanned ground vehicle (UGV) in a road network with unmanned ground sensors (UGSs). The road network also contains a pursuer that the UGV is trying to evade while exiting the road network. A human operator monitors and interfaces with the UGV during operation. The operator does not have a detailed knowledge of how the UGV functions or makes decisions, but can interrogate the system, modify the decision making stance (such as `aggressive' or `conservative'), and provide information and advice to the UGV. In this situation the operator could benefit from the UGVs ability to express confidence in its ability to escape given the current sensor information, and work with the operator to modify behavior if necessary.

	\begin{figure}[htbp]
    	\centering
     	\includegraphics[width=0.4\textwidth]{Figures/RoadNet}
    	\caption{Application example of unmanned ground vehicle (UGV) in a road network, trying to evade a pursuer. The UGV has access to unmanned ground sensors (UGSs) (also an a unmanned aerial system (UAS) that can be used as a mobile sensor), as well as information and decision making advice from a non-expert human operator. The operator's actions towards the UGV and trust-based.}
        \label{fig:RoadNet}
    \end{figure}

    In this scenario trust-cycle terms can be defined as follows:

    \begin{description}
        \item [Artificially Intelligent Agent:] The UGV
        \item [Trust:] The operator's willingness to rely on the UGV when the outcome of the mission is at stake
        \item [Trust-Related Behaviors:] The operator's behaviors towards the UGV, including the information provided, and the commands given
        \item [Assurances:] Implicit and explicit communication from the UGV that has an effect on the operator's trust
    \end{description}

    The following sections will discuss each of the above terms in more detail, and provide more formal and general definitions.

%aias.tex
%%%\nisarcomm{merge, trim...move before example?}

    %Intelligent technology spans a wide spectrum of capabilities. With regards to autonomous systems, these 
    %Autonomous systems can describe anything from 
    %%a thermostat
    %a simple assembly line robot to the fabled HAL 9000. 
    While the main interest of the authors is geared towards human trust in `advanced' technology for dynamic robotic decision making under uncertainty, this survey we will take a more holistic view and use the term Artificially Intelligent Agent (AIA) to encompass a broad range of technologies that can be considered `autonomous'. 
    %%, to gather generally applicable insight. 
    An AIA is defined here as an agent that acts on an internally/externally generated goal, and possesses, to some extent, at least one of the capabilities shown in Fig.~\ref{fig:AIcapabilities} ~\cite{Russell2010-wv,Nilsson2009-rp,Luger2008-vf}. 
    While an AIA can describe anything from a simple assembly line robot to the fabled HAL 9000, this definition underscores the idea that many assurances that exist for one set of AIAs can be adapted and generalized for use in other AIAs. 
    %%For instance, chi-square consistency tests for Kalman filter state estimation algorithms for autonomous vehicle perception, target tracking and control systems \cite{Bar-Shalom2001-tg}. 
    In other words, this definition sets a scope for the bodies of research that are likely to have investigated assurances and assurance principles, which can be extended to any intelligent computing system. 
    The range of AIA capabilities also helps establish what kinds of assurances might be needed in future systems. 
    For example, assurances for an AIA that only carry out planning tasks will probably differ in design or implementation from assurances for an AIA that only carry out perception tasks. 
    
    %\nisarcomm{for me todo: one more important idea that hasn't been articulated here, but that we rely on from very first sentence of paper, is that AIA is seen as subordinate delegate to human -- this is key idea since it bridges definition of AIA to need for understanding user trust. can cite Chris Miller's work?}
    
    It should be noted that an AIA is assumed to operate with a degree of autonomy that is \emph{delegated} by a user. That is, an AIA is self-directed and self-sufficient in its task to the extent that the user's `intent frame' (desired goals, plans, constraints, stipulations and/or value statements) can be met by the AIA, regardless of how it actually accomplishes these. %or what intermediate decisions it needs to make to achieve these ends consistently. 
    Following \citet{Miller2014-av}, this view of autonomy as a delegation relationship refines need for `transparent AIAs' by avoiding a contradiction of purpose that stems from an otherwise naive interpretation. From a naive standpoint, one could argue that if AIAs are developed primarily in the first place to alleviate the burden of complex reasoning and other undesirable workloads by removing users from the task at hand entirely, then this purpose is undercut by exposure and explanation of sophisticated AIA inner workings to the user. 
    However, if AIAs are subordinates that are delegated tasks by users (who must still act as supervisors), the meaning of `transparency' shifts away from concern over how exactly an AIA accomplishes a task, towards concern over whether or not an AIA can execute the task as per the user's intent frame. 
    This delegation-based view naturally sets up the question of user trust in AIAs. 
    %
    %To this end, an artificially intelligent system needs to possess at least some of the capabilities shown in Figure~\ref{fig:AIcapabilities}~\cite{Russell2010-wv,Nilsson2009-rp,Luger2008-vf}. 
    %Some might argue that it is also necessary to add other categories like creativity and social intelligence~\cite{Tao2005-kh}. 
    %\brettcomm{SEEMS TO DETRACT---}Some of these categories are also not clearly separable; for instance, where does the capability to `plan' end, and `reasoning' begin? Nevertheless, these capabilities are conceptually useful in defining an AIA:     
%    \begin{description}
%        \item[Artificially Intelligent Agent (AIA):] an agent that acts on an internally/externally generated goal, and possesses, to some extent, at least one of the capabilities shown in Fig.~\ref{fig:AIcapabilities} ~\cite{Russell2010-wv,Nilsson2009-rp,Luger2008-vf}. .
%    \end{description}

	\begin{figure}[t!]%[thbp]
    	\centering
     	\includegraphics[width=0.55\textwidth]{Figures/AI_capabilities}
    	\caption{Set of possible AIA capabilities.}
        \label{fig:AIcapabilities}
    \end{figure}

%    The broad range of AIAs implied by this definition is most usefully viewed in terms of scope and adaptability. Scope refers to the range of possible applications for an AIA: does it have a small number of specialized application, or can it be used in many different applications? Adaptability refers to the ability of the AIA to become better at executing its goal over time. Low adaptability has often been associated with `weak AI' whereas high adaptability is often associated with `strong AI'.  Figure~\ref{fig:StrongWeak} depicts these axes for some (real and fictitious) AIAs.
%
%	\begin{figure}[htbp]
%    	\centering
%     	\includegraphics[width=0.7\textwidth]{Figures/strong_weak_narrow_broad.pdf}
%    	\caption{Illustration of the range of systems encompassed by the AIA definition. Horizontal axis reflects the scope of the AIA, the vertical axis reflects the adaptability of the AIA. \nisarcomm{todo: REMOVE}}
%        \label{fig:StrongWeak}
%    \end{figure}

%    Arguably, we might instead have used the term `artificial intelligence' (AI) instead of AIA. However, `AI' carries too much ambiguity (in its fullest meaning, it would possess all capabilities from Figure~\ref{fig:AIcapabilities}, and more). AIA allows the broad inclusion of \emph{any} system in the adaptability/scope plane. The research discipline of machine learning (ML) is a subset of the AI research landscape. Individual ML algorithms might be thought of as being a narrowly scoped AI that is contained within only one of the AIA capabilities. 
%
    %One might also question the need to define AIAs in the first place. 
    %This is to aid in the search for and understanding of assurances. As will be shown later, different methods of assurance can be found over the entire range of AIAs, so that an automation system such as a factory robot might be able to use similar assurances -- or more generally, similar principles of assurance -- as might a self-driving car, and vice-versa. The capabilities of AIAs (Fig.~\ref{fig:AIcapabilities}) are the sources of assurances; in other words, assurances cannot exist without some grounding set of AIA capabilities. 
%
 %   This definition %, while broad, is still useful because it 
 %   encompasses many systems that are typically described as `artificially intelligent'. 
\subsection{User Trust} \label{sec:trust}
    In designing assurances that affect trust-based user behaviors, it is critical to know what drives those behaviors. Because of this, some time must be spent to understand what trust is. 

    Trust is critical in interpersonal relationships, and it affects the dynamics of intelligent multi-agent systems as simple as one-on-one personal interactions  \cite{Lewicki2006-hj}, to more complicated ones such as financial markets and governments \cite{Fukuyama1995-un}. Consequently, researchers in psychology, sociology, and economics have historically sought to understand the fundamental principles of trust, each with the aim of understanding their field better \cite{Gambetta1988-pi}. Moral philosophers have also thought intently about the topic \cite{Baier1986-im}.

    Due to wide interest spanning many disciplines it is difficult, if not impossible, to write a succinct definition of trust that would appease all interested parties. Besides that, trust is actually a very broad concept that evades precise definitions at a high level. However the following definition, adapted from \cite{McKnight2004-vv}, is broad enough to avoid too much contention:

    \begin{description}
        \item [Trust:] a psychological state in which an agent willingly and securely becomes vulnerable, or depends on, a trustee (e.g., another person, institution, or an AIA), having taken into consideration the characteristics (e.g., benevolence, integrity, competence) of the trustee.
    \end{description}

    \subsubsection{Trust between AIAs and humans?}
        Trust is generally understood to exist between people. Is it possible for a human to enter into a trusting relationship with an AIA?
        % In their paper regarding important human factors that should be considered when designing autonomous machines \cite{Sheridan1984-kx} are seemingly the first to discuss the idea that trust relationships between humans and autonomous systems are important, and to suggest that humans need some assurance that the ``commands will be carried out properly''. They also mention the idea that ``there needs to be an accurate perception of [the autonomous system's] trustworthiness''. Finally they suggest that ``appropriate criteria for trust need to be studied to develop a theory of trust in supervisory control''.
        % Perhaps motivated by \citeauthor{Sheridan1984-kx}\cite{Sheridan1984-kx}, a few years later \cite{Muir1987-mk}, and later in more detail \cite{Muir1994-ow}, create a psychologically based model of trust that considered the ``component expectations of trust'' of \cite{Barber1983-yc} and the dynamic evolution of trust from \cite{Rempel1985-sg}, to make a framework for studying trust in human-machine relationships.
        % \citet{Muir1996-gt} reported the results of two experimental studies to investigate the validity of her proposed model. She claims that these were the first experiments to explicitly ask "operators to rate their trust in automated equipment", and to see if they could do so under normal operating conditions. She found that operators were able to rate their trust in the automation, and that the level of trust changed based on different performance characteristics of the automated system. In her own words: "These results suggest that operators' subjective ratings of trust and the properties of the automation which determine their trust, can be used to predict and optimize the dynamic allocation of functions in automated systems".
        That humans actually do feel trust towards machines has been experimentally confirmed several times in research using common subjective psychological questionnaires. Some examples include: \citet{Muir1996-gt,Reeves1997-ad,Groom2007-bz,Mcknight2011-gv,Riley1996-qm,Bainbridge2011-pl,Kaniarasu2012-mo,Salem2015-md,Desai2012-rc, Freedy2007-sg, Wang2016-id, Inagaki1998-cl, Kaniarasu2013-ho}. 

        Several academic experiments have investigated the possibility of trust existing between humans and (according to the terminology of this survey) AIAs. All found that some level of trust can be formed in such relationships. For instance, \citet{Lacher2014-yc} points out that people trust an AIA at different levels. For example, an operator would have different perspectives on trust based on their level of interaction with the AIA. The designer of an AIA would also trust the AIA differently than an end user, due to the differing nature of the trust relationship from one to the other. 
        % \citet{Lankton2008-ct} claims, and finds some support for the idea, that trust in technology is fundamentally different from interpersonal trust between humans. They demonstrate the validity of the hypothesis by using a survey of 427 college students regarding Facebook. However, the authors point out that this study was based on a single set of survey data about facebook, and may not be unbiased or apply to other technologies. Beyond this, it is the author's opinion that the `fundamental differences' they point out are not that divergent from the human-human trust model.

        \citet{Tripp2011-cq} investigate the variation of trust between humans and different levels of technology. They run experiments with three different levels of technology: Microsoft Access, a recommender system, and Faceobook. They found that `human-like' trust applied more to Facebook, while `system-like' trust applied more to MS Access. They conclude that if the system is `human enough', then a human trust model is appropriate.

        Given this research, we will take the position of presenting a human-human trust model and use it as a basis for human-AIA trust -- with the understanding that the strength of the model varies with the complexity of the AIA. In other words, some features of the model will have varying level of significance over the range of adaptability and capability of AIAs.

\subsubsection{A Model of Human-AIA Trust}
        We now present a model of human-AIA trust, which will cast insights on assurances that will be discussed later. It should be noted that this model is being presented as \emph{one possible model} that can be helpful in understanding assurances -- it is neither the only model, or a perfect model. As research advances, such models will likely continue to evolve, and the ideas of assurances will naturally evolve as well.
            % This has been recently attempted in the context of human-AIA relationships \cite{Lahijanian2016-nd}, but with an overly simplistic reduction of trust. The reality is that trust is extremely complex, and so dealing with it in the setting of human-AIA relationships is going to be complex.

        In work relating to business management, \citet{McKnight1998-ty}, and later \citet{McKnight2001-fa}, performed what is, arguably, the first multi-disciplinary survey and unification of trust literature, which also condensed it into a single typology. The resulting model is shown, with some minor adaptations, in Figure \ref{fig:UserTrust}. The figure illustrates the three categories that make up a human's trust. There are causal arrows that connect the different components. The `Dispositional Trust' block is generally considered by psychologists, and deals with long-term psychological traits that develop in a person from childhood. The `Instututional Trust' block is generally studied by sociologists, and represents the level to which a person trusts social/commercial structures. Finally, the `Interpersonal Trust' block is deals directly with one-on-one relationships and can generally fluctuate more quickly than the other two.

        \begin{figure}[htbp]
            \centering
            \includegraphics[width=0.9\textwidth]{Figures/UserTrust}
            \caption{Interdisciplinary trust model proposed by \citet{McKnight2001-fa}. The three main categories are delineated, and corresponding disciplines that are interested are listed within parentheses. Connections indicate a causal relationship. The suggestion regarding time scales of the development of trust is the author's addition, trust development is discussed more in \cite{Lewicki2006-gp}, and \cite{Lewicki2006-hj}}
            \label{fig:UserTrust}
        \end{figure}

        In the context of AIAs the components of the three categories from figure \ref{fig:UserTrust} are defined as follows:

        \begin{description}
            \item [Disposition to Trust:] The extent to which one displays a consistent tendency to be willing to depend on AIAs in general across a broad spectrum of situations and persons
            \item [Institution-Based Trust:] One believes that regulations are in place that are conducive to situational success in an endeavor
            \item [Trusting Beliefs:] One believes that the AIA has one or more characteristics beneficial to oneself
            \item [Trusting Intentions:] One is willing to depend on, or intends to depend on, the AIA even though one cannot control its every action
        \end{description}

        Each of these main categories of trust has components defined in Figure \ref{fig:Assurance_classes}. These components were defined through the compilation of many research studies across research disciplines, and because of this represent the most accurate notion of the components of trust available. It is asserted here that these are the principal drivers of TRBs, the targets at which assurances must be directed.

        \begin{sidewaysfigure}[htbp]
            \includegraphics[width=8in]{Figures/Assurances.pdf}%
            \caption{Assurance targets based on the component definitions of the main categories of trust: `Disposition to Trust', `Institution-Based Trust', `Trusting Beliefs', and `Trusting Intentions'}
            \label{fig:Assurance_classes}
        \end{sidewaysfigure}

\subsection{Trust-Related Behaviors} \label{sec:trbs}
Something that is well accepted among researchers of all disciplines is that trust ultimately leads to some kind of behavior or action; this idea was highlighted by \citet{Lewis1985-pr}.  \citet{McKnight2001-fa} call these `trust-related behaviors` (TRBs), which is the term that will be used in this survey. In the case of a human-AIA relationship, the author is concerned with TRBs could include the kinds of tasks the human user assigns to the UGV such as accepting and following through on its plan, or directing that a new plan be made.

\subsubsection{Calibration of Trust-Related Behaviors}
    Trust is not a quantity that can be objectively measured. Rather, its relative magnitude must be observed through changes in TRBs, or qualitative surveys \cite{Muir1996-gt}. Of these two approaches, TRBs are the more objective measure due to the fact that people are not always consistent in their ratings, and may sincerely feel different levels of trust while performing similar TRBs. \citet{Parasuraman1997-co} were interested in understanding the use of automation by humans, and defined terms related to  along these lines. Here it is proposed that, by extension, those terms also apply to the relationship between humans and AIAs. Within this scope the definitions are as follows:%\footnote{a third term `Abuse' was left out because it concerns the decision by humans to install automation in an environment that was inappropriate. It is my belief that this can be wrapped under the umbrella of `Misuse'},
    
    \begin{description}
        \item [Misuse:] The over-reliance on an AIA -- which could manifest itself in expecting too much accuracy from and AIA, or believing that it can be applied in an application that it was not designed to function in
        \item [Disuse:] The under-utilization of and AIA -- which could be manifest in a user turning off the AIA, or failing to use all of its capabilities 
        % \item [\textbf{Abuse}:] I don't agree that abuse should be included -- we should talk about it
        \item [Abuse:] Inappropriate application of automation (where \emph{application} in this case means the choice to deploy an AIA in a certain context, such as the choice to use a quad-copter underwater).
        % \nisarcomm{I don't think you should leave it out -- there is an important and subtle difference between over-reliance and misapplication of a capability/technology -- assurances to prevent the former are thus not going to be the same as assurances that try to avoid the latter. Also, the installation of automation in an environment that is inappropriate doesn't necessarily translate the same way to autonomous systems -- in particular, since autonomous systems often have to be self-directed and make decisions under uncertainty on their own. i.e. using the Google car to tow another vehicle behind it would not necessarily count as over-reliance/misuse, but could possibly be construed as abuse -- it is in the right environment, and it is still carrying out a function that any rational user might possibly conceive as a capability it ought to possess, since the car is still essentially carrying out the same function as before and doing what any car might be expected to do -- however, a user may not be aware of the fact that placing another vehicle in tow behind the car might introduce perception errors and planning errors, since the vehicle may not be able to fully account for the altered dynamics or changed sensor field, etc. -- it was not really designed for towing tasks, but someone might get lucky and get away with pulling it off a few times. In contrast, using the Tesla autopilot on the highway (even though it results in a crash while driver is watching a movie) is not abuse but clearly a case of misuse/overreliance -- the system encountered a known failure condition in a scenario in which it was expected to be deployed (highway driving), but the operator/user was not prepared to either anticipate or deal with the situation -- here the driver was habituated into overtrusting the system, since nothing bad happened before, i.e. no failures were encountered for the system in its intended operating condition and environment. }
    \end{description}

    Again referring to the diagram in Figure \ref{fig:SimpleTrust_one_way} the AIA must influence the user's TRBs by way of assurances. We propose that the goal of an AIA should be that the user should not misuse, disuse, or abuse it. In other words, the space of all trust-related behaviors can be made up of misuse, disuse, abuse, and appropriate behaviors (all behaviors that aren't misuse, disuse, or abuse).

    % \nisarcomm{in light of previous comments above, it is worth spelling out in just a few sentences what difference in translation of misuse, abuse, and misuse is when going  from automation to AIAs -- i.e. to what extent to the concepts carry over directly, and which parts or considerations might have to be modified to accommodate the more autonomous nature of AIAs? This really is where some of the insights on the relationship between trust and assurances start to emerge.}
    
    % To be more formal, let the total set of TRBs as $\mathcal{T}$. Then as subsets of $\mathcal{T}$ define the set of misuse actions as $\mathcal{M}$, the set of disuse actions as $\mathcal{D}$, and the set of abuse actions as $\mathcal{A}$. Next, define the total set of inappropriate TRBs $\mathcal{I}$ as the union of $\mathcal{I} = \mathcal{M}\cup \mathcal{D}\cup\mathcal{A}$. Having defined the set of inappropriate actions, the set of appropriate TRBs can be defined as $\mathcal{U}$, the compliment of the set of inappropriate TRBs $\mathcal{U} = \mathcal{I}^\prime$. This is illustrated in Figure \ref{fig:appropriate_use}, where the set of appropriate actions $\mathcal{U}$ is the gray colored area (i.e. all TRBs \emph{not} in either of the three sets of inappropriate TRBs). \textbf{This is probably too complicated, I could probably just say that $\mathcal{U}$ is the set of all actions not included in either of M,A, or D}.
    %
	% \begin{figure}[htbp]
        % \centering
         % \includegraphics[width=0.4\textwidth]{Figures/misuse_disuse_abuse}
        % \caption{Graphic representing the total space of user actions, in which the inappropriate uses $\mathcal{M}$, $\mathcal{D}$, and $\mathcal{A}$ lie. The set of inappropriate uses $\mathcal{I}$ is the union of $\mathcal{M}$, $\mathcal{D}$, and $\mathcal{A}$. The appropriate set of actions $\mathcal{U}$ is the compliment of $\mathcal{I}$, or the part of $\mathcal{T}$ that does not include $\mathcal{I}$.}
        % \label{fig:appropriate_use}
    % \end{figure}
    
    Given this definition, in order to ensure that humans use AIAs appropriately, it is critical that the user TRBs be calibrated to elicit behaviors that are within the set of appropriate behaviors, which can only be done by influencing the user trust. This is a point that, to some extent, has been informally mentioned in \citet{Muir1994-ow,Muir1987-mk,Lillard2016-yg,Lee2004-pv,Hutchins2015-if}.

    A critical oversight of other researchers who mention `calibration' is that they suggest calibrating \emph{trust} as opposed to TRBs. \citet{Dzindolet2003-ts} studied the effect of performance feedback on user's self-reported trust, and found that it increased; however the appropriate TRBs toward the system did not reflect the level of self-reported trust. This shows the danger of calibrating ``trust'', as opposed to calibrating the TRBs.

    Calibrating TRBs focuses on concrete and measurable behaviors that are universally applicable. In contrast, calibrating trust involves influencing a quantify that is directly immeasurable, and that, when measured indirectly, is subject to the biases and uncertainties of humans, along with inherent differences between different users. Viewing the task from this point of view, the findings of \citeauthor{Dzindolet2003-ts} are not surprising.

    % \textbf{I'd like to go off on a little rant about this, but I'm not sure if it is appropriate. There is a TON of literature that talks about calibrating trust. Calibrating trust is asking for trouble, when we actually care about TRBs. VERY LITTLE RESEARCH  (none?) HAS BEEN DONE CONSIDERING ONLY TRBS, IT IS MOSTLY JUST SELF-REPORTED TRUST, WHICH DZINDOLET HAS SHOWN TO BE SHAKY GROUND} \nisarcomm{This is a very interesting point -- definitely worth mentioning as a conclusion/takeaway of this survey and worth restating/discussing in a bit more detail at the end for open opportunities and future work, etc.}

    It is desirable for AIAs to be designed in order to encourage appropriate TRBs, as opposed to the alternative of purposefully misleading users misuse or abuse. There is a valid argument that many of today's AIAs that ignore (or whose designers ignored) TRBs and assurances can be `unwittingly malicious' in that they do not actively attempt to guide user's TRBs to lay within the space of appropriate TRBs.
    % A benevolent AIA the user's TRBs should be appropriate \nisarcomm{I am not sure what you mean here -- when I read this sentence out loud, it doesn't really come together: what do you mean by the `perspective of a benevolent AIA'?? Also, `benevolent AIA' seems to ascribe some sort of intentionality to an AIA, which was not really discussed or mentioned at all before in your AIA definition}.
    % This in contrast to a malicious AIA that tries to manipulate the TRBs of a human to overlap with misuse, \edit{abuse} or disuse to some extent \nisarcomm{again, unclear what precisely you mean here when reading this out loud -- are you trying to make the point that simply getting a user to trust a system more is not meaningful or beneficial on its own, since it's not hard to dupe people into trusting something that's actually bad for them? (note this doesn't necessarily square with the idea you seem to be proposing of `tricking' people into abusing, misusing, or disusing AIAs -- the connotation of abuse, misuse and disuse is more that these are tendencies that people tend to converge to on their own; misuse is arguably the only one that people might actually get actively tricked into in terms of having their trust levels being actively manipulated by an AIA; interestingly, disuse can also arise in `neutral/benevolent' systems, e.g. the Mars Rovers: disuse of the advanced features onboard the rovers arises due to cultural influences at NASA, i.e. don't let rovers make decisions that can't be backtracked or accounted for fully, since the cost of failure is too high and thus the engineers are highly risk averse -- the rover autonomy itself arguably plays no role in shaping this disuse}. There is a valid argument that many of today's systems that ignore TRBs and assurances are unwittingly malicious in that they do not actively attempt to guide user's TRBs to lay within the space of appropriate TRBs .

    % Generally trust between a human and AI could be depicted as in Figure \ref{fig:SimpleTrust_two_way}, where each has TRBs that must be calibrated, and each provides certain feedback, which will be called assurances, in order to do so. In a more simple scenario, where the AI implicitly trusts the human user the trust relationship can be depicted as shown in Figure \ref{fig:SimpleTrust_one_way}, where only the user has TRBs that are being calibrated.

	% \begin{figure*}[htbp]
        % \centering
        % \begin{subfigure}[t]{0.48\textwidth}
            % \centering
            % \includegraphics[width=0.95\textwidth]{Figures/SimpleTrust_two_way}
            % \caption{Diagram showing a general case of a two-way trust relationship between an AI and a human. Arrows that are not connected to boxes represent some action outside of the trust loop.}
            % \label{fig:SimpleTrust_two_way}%
        % \end{subfigure}
        % \hfill
        % \begin{subfigure}[t]{0.48\textwidth}
            % \centering
            % \includegraphics[width=0.95\textwidth]{Figures/SimpleTrust_one_way}
            % \caption{Diagram illustrating a general one-way trust relationship between a human and an AI. In this case the AI has, what could be considered perfect trust in the user.}
            % \label{fig:SimpleTrust_one_way}%
        % \end{subfigure}
        % \caption{Feedback Loops For One and Two-way human-AI Trust Relationships}
        % \label{fig:SimpleTrust}
    % \end{figure}
    
    % Figure \ref{fig:SimpleTrust_dist} serves as a simple example illustrating the possible disparity between the user TRB distribution and the appropriate TRB distribution. In this case assurances would be used to minimize the difference between the two distributions.

	% \begin{figure}[htbp]
        % \centering
         % \includegraphics[width=0.4\textwidth]{Figures/SimpleTrust_dist.png}
        % \caption{Diagram illustrating the point that a hypothetical user TRB distribution might not match the appropriate TRB distribution. In this case the AI should provide assurances in order to minimize the difference between the two.}
        % \label{fig:SimpleTrust_dist}
    % \end{figure}
%

An assurance is an AIA property or behavior that can either increase or decrease user trust. The term `assurances' is perhaps earliest used in the context of human-AIA relationships by \citet{Sheridan1984-kx}. \citet{McKnight2001-fa} allude to this kind of feedback in e-commerce relationships as `Web Vendor Interventions'. \citet{Corritore2003-gx} refer to assurances as `trust cues' that can influence how online users trust e-commerce vendors. \citet{Lee2004-pv} discuss `display characteristics', which are methods by which an autonomous systems can communicate information to an operator. More recently, \citet{Lillard2016-yg} provided a formal definition of assurances for autonomous systems that is similar to the one used here. 

\begin{figure}[t]%[htpb]
    \centering
    \includegraphics[width=0.6\linewidth]{Figures/RefinedTrust_one_way.pdf}
    \caption{Figure depicting the details of the human-AIA trust cycle.}
    \label{fig:refined_trust}
\end{figure}

Assurances can be classified in several different ways. One way to classify an assurance is by its \emph{Information Source:}. Assurances must be informed by some kind of information, whether that means real-time observation of TRBs in order to have feedback, or well accepted concepts of cognitive science as guiding principles of design.
Another approach is to identify the \emph{Source/Target} pair: In a human-AIA trust relationship, assurances link the AIA to the user. The user has multi-dimensional trust in the AIA (see Fig.~\ref{fig:Assurance_classes}), and each AIA capability has multiple dimensions of `trustworthiness'. In designing an assurance it is useful to explicitly identify the source capability, and the target trust dimension (i.e. a certain assurance may have been designed as a `planning-competence').
An assurances can be considered \emph{Component} or \emph{Composite:} A component assurance stems from one AIA capability to one trust dimension. A composite assurance originates from multiple AIA capabilities to one trust dimension.
Another consideration is whether the assurance is \emph{Tutoring} or \emph{Telling:} An assurance that is dynamic to the different characteristics, and experience of users is a `tutoring' assurance. It is designed to help a user learn, over time, to trust appropriately. Conversely, all other assurances are `telling' in that they are static in regards to separate users.
\emph{Mode of Expression:} Assurances can also be classified by their mode of expression. This includes the method and medium by which the assurance is expressed.
There are many open questions regarding each of these categories; they are discussed further in Sec.~\ref{sec:future_work}, regarding future work.

\emph{Level of Integration:} Herein the `level of integration' of assurances are surveyed. This is useful because it addresses a natural consideration in the design process of AIAs; it also encapsulates well the key approaches that are in use. In this context `integration' refers to the level of effect the assurance has on the core functions of the AIA. As an example: an assurance that, if missing, greatly effects the AIA functionality is considered integral to the AIA. Conversely, a missing assurance that has no effect on the AIA functionality is not integral; we also call this `supplemental'. Between these two extremes there is a natural continuum of integration on which we can classify the different algorithmic approaches to designing assurances; we do so in Sec.~\ref{sec:synthesis}.

