\section{Future Work} \label{sec:future_work}
\brettcomm{Add more stuff about avenues for future work}

\begin{itemize}
    \item Draw from other disciplines that have `useful' tools. (stuff from old Q4)
    \item How to know if assurances are effective. Are explicit assurances being perceived as desired? Are implicit assurances overpowering?
    \item component and composite assurances. What happens when assurances are combined vs. when they exist in isolation? How should assurances be combined?
    \item methods, and modes of expression. What human limitations must be considered
    \item from the perspective of AIAs are there assurances that exist for each of the capabilities that can communicate to the different trust targets?
    \item currently assurances are typically `displayed' as static information to a user, can a user be taught over time by planning assurances?
    \item what cognitive effects inhib assurances
    \item how to select the expression of an assurance? Easy to calculate hard to communicate\ldots
    \item measuring effects of assurances
    \item distrust
    \item two-way trust
\end{itemize}

\subsubsection{Component and Composite Assurances}
Assurances can be either component or composite. This was seen a little through the survey. The definitions are as follows:

\begin{description}
    \item [Component Assurance:] An assurance that originates from a single AIA capability source, and targets a single trust dimension target.
    \item [Composite Assurance:] The combination of more than one component assurance into a single assurance. 
\end{description}

\begin{figure}[!htbp]
    \centering
    \includegraphics[width=0.9\textwidth]{Figures/Assurance_component_composite.pdf}
    \caption{Figure illustrating the difference between component and composite assurances. The existence of multiple assurances does not imply a composite assurances, rather the combination of multiple component assurances into a single assurance constitutes a composite assurance.}
    \label{fig:assurance_mapping}
\end{figure}

Figure \ref{fig:assurance_mapping} illustrates the concepts of component and composite assurances.

\paragraph{Component Assurances:} Component assurances are perhaps the most well researched in the existing literature. This is likely because several verified component assurances are the predecessors to composite ones. A component assurance might include displaying the confidence of a classification prediction, or visualizing a model as discussed in section \ref{sec:q2}.

\paragraph{Composite Assurances:} Composite assurances are assurances that are built of several components. A notable example is the work by \citet{Aitken2016-cv} who propose a measurement called `self-confidence', applicable to Partially Observable Markov Decision Processes (POMDPs). This metric combines five component assurances into a single composite assurance that is meant to distill the information into a value that a novice operator could understand easily. This paper was discussed in more detail in \ref{sec:q2}. 

\subsubsection{Explicit and Implicit Assurances}
\citet{Sheridan1984-kx} briefly alluded to the existence of explicit and implicit assurances when they discussed the nature of how humans behave when working with automated systems. They suggested that the operator's perception of the automated system can be effected by `performance' and its `reports on its own performance'. The terms are more formally defined as,

\begin{description}
    \item [Explicit:] Assurances that are purposefully given to affect the trust of a user.
    \begin{itemize}
        \item Legible motion \cite{Dragan2013-wd}, which is motion calculated with the intent of being more understandable by a human
        \item $R^2$ value, gives some indication of how well the regression accounts for the variance of the data
    \end{itemize}
    \item [Implicit:] All other assurances that aren't explicit.
    \begin{itemize}
        \item Reliability in completing a task. Generally, the object of success is not to affect the user's trust (although this is a nice side-effect).
        \item The way an autonomous vehicle appears. For example something that looks neat will have a different effect on trust, than an AIA with wires dragging on the ground. 
    \end{itemize}
\end{description}

It is important for designers of AIAs to be aware of both, but anything that is consciously designed with the goal of affecting trust is automatically an explicit assurance, from the perspective of the AIA. Another way of stating the ideas is that trust relationships between humans and AIAs will form, but all assurances will be implicit if designers do not consciously consider the trust relationship. 

\subsubsection{Tutoring vs Telling} \label{sec:teach_tell}
    This is also a point rarely seen in the literature surveyed, but critical in the context of different users and long-term human-AIA interaction. We suggest that assurances can also be classified by whether they consider tutoring (or leading) the user to help them understand, or whether they just produce a static and unchanging value regardless of the user, their experience, or their expertise. This is a point mentioned (in terms of explainability) by \citet{Lacave2002-cu}, and \citet{Lacher2014-yc}.

    A tutoring assurance would be a planned, dynamic, sequence of assurances that would change in time to adapt to the user's needs (as discussed in section \ref{sec:consider_human}. This might include modification of assurances to help a user avoid boredom, or to use the system differently in varying circumstances. For example the first time an autonomous vehicle encounters snow with a certain user, it might take time to give special assurances. Or a user that is so used to an AIA that its TRBs begin to drift to disuse, and the AIA gives a special assurance to correct that.

    It isn't surprising that, to our knowledge, no research has been done with respect to tutoring a user in a trust relationship. This is a complex problem to address that would involve understanding how different users learn, and what an appropriate strategy would be to teach them to have appropriate TRBs. However, a rich resource (not investigated in this paper) would be the work on tutoring systems. There is definitely an open area for research that investigates the advantages of tutoring assurances versus those of telling assurances, and how easy they are to implement in contrast to the added time and effort needed to design tutoring assurances.

\subsubsection{Source-Target Classification}
    It is convenient to refer to assurances by way of their source and target. More specifically their source AIA behavior (see Figure \ref{fig:AIcapabilities}) and their user trust target component (see figure \ref{fig:Asurane_classes}. Intuitively, there may be a set of different algorithms that are useful for making assurances that convey information about planning to the competence dimension of the user's trust. It is easier to refer to these assurances in terms of their source and target. So, for this example that class of algorithms would be the `planning-competence' class.

    The source AIA capability for an assurance might be most easily thought of by the algorithm it has to operate on. If the assurance operates on, or comes from, a planning algorithm then it would be considered a `planning' source. If the algorithm operated on data for learning patterns, the it would be considered a `learning' source.

    The trust dimensions shown in Figure \ref{fig:Assurance_classes} are the possible targets for assurances. The categories mirror those of the trust model proposed by \citet{McKnight2001-fa}, but with the emphasis on what an AIA has the ability to most readily influence (and consequently where most research is found). The boxes with the beveled corner identify and define the different classes of assurances. All classes are included here for completeness and generality. Although, while it is hypothetically possible for an AIA to influence a persons general `Trusting stance' given enough time\footnote{One might imagine an AIA that specifically speaks to the human about the benefits or drawbacks about trusting even though there might not be evidence to do so, similar to the role a counselor might play}, the gray boxes are not considered further in this survey, as practically no direct research exists in the realm of human-AIA relationships.
    
    Not only is the source-target notation useful shorthand for communicating about the purpose of the assurance, but it is useful in classifying the range of assurance algorithms that exist. There may also be a class of algorithms that span multiple source-target capabilities. For example there may be a kind of algorithm that can give a `learning-competence' assurance, as well as a `planning-competence' assurance. This is especially true since many of the AIA capabilities can overlap. Also, the effects of assurances cannot be guaranteed to affect only one trust dimension (see section \ref{sec:imprecise}).


\subsection{Distrust}
The treatment of assurances in this survey is based, in part, on a model of interpersonal trust. For completeness it is important to mention the concept of \textit{distrust}, as reviewed and discussed by \citet{Lewicki1998-ox}, and formalized in \citet{McKnight2001-gz}. Low trust is not the same as distrust, and low distrust is not the same as trust. \citet{McKnight2001-gz} suggest that ``the emotional intensity of distrust distinguishes it from trust'', and they explain that distrust comes from emotions like: wariness, caution, and fear. Whereas, trust stems from emotions like: hope, safety, and confidence. Trust and distrust are orthogonal elements that define a person's TRB towards a trustee. In this survey, distrust was not considered. However it must be made clear that any \emph{complete} treatment of trust relationships, and for our purposes, designed assurances, must consider the dimensions of distrust as well as those of trust. For now, this is left as an avenue for future research (one yet to be picked up by mainstream AIA researchers).

\subsection{Expression and Perception of Assurances} \label{sec:express_assurances}
Expression and perception of assurances have been combined in this section because they share several critical aspects. The key points to be considered here in design of assurances are:
(1) Mediums; (2) Methods; and (3) Efficacy.     
%     \begin{itemize}
%         \item Mediums
%         \item Methods
%         \item Efficacy
%     \end{itemize}
    
For explicit assurance design, the medium and method of expression must account for the AIA's limitations. 
Here medium denotes the means by which an assurances is expressed; this could be through any of the senses by which humans perceive, such as sight, sound, touch, smell, and taste. The method of assurance is the way by which the assurance is expressed. For example: a time series plot may be conveyed visually in the typical way, or via a spoken or textual description; in this case the plot or text description is the method, and sight or sound are the different mediums through which it can be communicated. An AIA might be limited in methods of expression, e.g. because it does not have a display or a speaker. %%In such cases, how is the user supposed to receive assurances?

A designer must also consider whether a human can perceive the assurances being given. If so, to what extent is the information from the assurance transfered, or how efficacious is the assurance? A few examples include: an AIA giving an auditory assurance in a noisy room and the user not hearing it (such as an alert bell in a factory where the workers use ear-plugs), or an AIA attempting to display an assurance to a user that has obstructed vision. 
An AIA may also have the ability to store data about its performance, and compute a statistic regarding its reliability -- but if it cannot successfully express (or communicate) that information in some way, the information is useless. 
If an assurance is not expressed, or not perceived by the user, it is useless and has no effect. 

%%...need to connect this paragraph to the previous one, otherwise it's a bit of a non-sequitur
It is also important to bear in mind that users will always produce some kind of TRB when interacting with an AIA (even if this only means choosing to ignore the AIA and not use it), and in the absence of explicit assurances users will instead use implicit assurances to inform their TRBs. 
However, users generally will not know which assurances are implicit or explicit -- e.g. humans participating in research from Quadrants I and II were generally not aware which aspects of the AIAs they interacted with were/were not deliberately designed by the researchers as assurances. 
Hence, there is always a danger that users can latch onto the `wrong' assurances, i.e. AIA features that are not meant to be interepreted as assurances but are nevertheless easily perceived (possibly moreso than intended explicit assurances). There is also the danger of overwhelming the user with too many easily perceivable explicit assurances, e.g. sounding and displaying several alarm indicators at once in an aircraft cockpit. 
%%Recall from Section~\ref{sec:assurances} that, to a user, all assurances are the same, i.e. any property or behavior of an AIA that affects trust is an assurance to a user, and it doesn't matter whether the assurance was designed for that purpose (explicit) or not (implicit). 

\subsubsection{Mediums}
In general, assurances are most often expressed visually. For example an AIA might give visual performance feedback to display different performance characteristics \cite{Chadalavada2015-wx,Muir1996-gt}. Written or spoken natural language can also be used \cite{Wang2016-id} -- given the impressive strides made by NLP researchers and practitioners lately, it is nowadays a simple matter to convert between written natural language and spoken natural language. 
These can be used to augment or replace more conventional audio-visual assurance indicators traditionally used and studied for human-machine/human-automation interaction, e.g. blinking lights, colored boxes in graphical displays, ringing bells/buzzers, recorded voice alerts, etc.
    
Other senses (touch, smell, and taste) are not well explored in literature related to human-AIA trust. Generally, any human sense could be used as a medium. Besides sight and sound, tactile feedback has been used extensively in robotics for `haptic feedback' (where the user receives mechanical feedback through the robot controls). This medium is used to create a more immersive user interface in robotics, to help users feel more connected to the robot (especially important for telerobotics applications). 
While one can imagine smell and taste having obvious applications in designing assurances for a cooking robot, other applications very likely exist and are open to further research.

\subsubsection{Methods}
Assurances associated with displaying AIA performance variables sound banal (e.g. flow rate for an automated pump \cite{Muir1996-gt}), but actually involves a nuanced point: the displayed performance value actually serves to inform the user's own mental model of the trustworthiness of an AIA capability. That is, the user's trust in the AIA's capability does not change only in response to the instantaneous `goodness/badness' of the AIA's performance, but accounts for the past history of the AIA's performance as well as any observed discrepancies between the AIA's expected behavior and its actual behavior.  
The user's trust dimensions (`competence', 'predictability', etc) are then affected by their perception of trustworthiness according to the combined model and data delivered by the display. 
This approach (also noted and discussed by \cite{Wickens1999-la,Sheridan1984-kx,Hutchins2015-if}) is effective, but relies heavily on the implicit assumption that the user will create a `good enough statistical model' of the AIA's behavior from data presented by the AIA. With this in mind, one might train a user to recognize signs of failure/success in different interactions with an AIA as assurances \cite{Freedy2007-sg,Desai2012-rc,Salem2015-md}. 
The main drawback of this idea is that it still relies on users' ability to construct `good enough' mental models of AIA behavior and characteristics from noisy observations to avoid misinterpreting AIA behaviors. 
However, this training can require intensive and costly special effort for non-expert or non-specialist users. 
A more ideal approach in such cases would be to design explicit assurances that help users construct correct/consistent mental process models of AIA behavior and thus reduce the risk of misinterpretation.

More direct methods of expressing assurances include displaying the intended actions, e.g. to indicate movements via visual projection of a planned mobile robot path \cite{Chadalavada2015-wx}. This is subtly but significantly different from making the user infer the intended action. Analogously, natural language expressions (written or otherwise) attempt a more active method of assurance expression. One might also display plans and logic in different formats, e.g. tables, trees, radar charts  \cite{Van_Belle2013-ph, Huysmans2011-th, Hutchins2015-if}, to remove some uncertainty regarding the user's ability to create an adequate mental model. %%%As humans are fond of saying ``You can't assume that I can read your mind!'', in essence more passive expressions from AIAs are relying on humans to read AIA's `minds' (we can't even do that with other humans).

It is often assumed that making an AIA more `human-like' will make it more trustworthy. 
An algorithm may be human-like when it represents knowledge in a human-understandable way, or executes logic in a way that a human can follow. 
A robot that is humanoid becomes more human-like in appearance \cite{Bainbridge2011-pl}, and thus implicitly projects that it has certain physical (and possibly mental) capabilities in common with humans as well. 
A system that uses natural language becomes more human-like in communication \cite{Lacave2002-cu}, and again projects that it has certain capabilities to understand or possibly hold a conversation at some level with a human user. 
The human-AIA trust relationship depends on assurances that, in essence, are conversions from AIA capabilities to human-perceptible/human-understandable behaviors and properties.  
Since assurances are the means of communication by which humans develop trust in AIAs, it is expected that all assurances have to at least be made human-understandable in some way (otherwise assurances will be totally ineffective). Therefore, it can be argued that assurances must make AIAs `human-like' in some regard.  

Interestingly, however, the converse is not true: making an AIA more `human-like' in any arbitrary way does not imply that it automatically provides assurances that make it more trustworthy. 
In \cite{Dragan2013-wd} the AIA is made more trustworthy by making the robot motions more human-like, whereas in \cite{Wu2016-ei} making the AIA more human-like resulted in a decrease of trustworthiness. In this case the difference came from the type of task: in the first case, the AIA (a robot) was physically working in proximity to a human, while in the other case the user was playing a competitive game against the AIA (a computer program). 
It has been observed that humans trust more `human-like' AIAs in more human-like ways \citet{Tripp2011-rx}. 
It is thus plausible to suppose that the term `human-like' can be more formally defined in terms of the difficulty that a typical user would have in relating to the AIA. 
Following on this idea, the benefits/drawbacks of human-like characteristics would be influenced by a user's general impressions and feelings of how trustworthy humans are in similar situations. 
This would also involve aspects of psychology and sociology, and would be very difficult to control and account for. 
Nevertheless, the problem of coping with such factors is an open and important research question that will impact the design of assurances for AIAs. 

It is also worth considering, in more detail, what implications the existence of implicit and explicit assurances means practically for AIA system designers when it comes to considering and implementing assurances. 
Since it is unrealistic for designers to take all possible kinds assurances into account, they will need to focus their efforts on how to identify and focus on only the most important ones. 
The foremost consideration is that an analysis of the interaction between the human and user needs to be made in order to identify the critical assurances for a given scenario. 
For example, in the road network problem, an analysis might find that the most critical assurances are about the competence of the UGV's planner. 
In this case the designer must take time to design an explicit assurance that is directed at the user's perception of the AIA's competence -- let's call this a `planning-competence' assurance. 
One difficulty arises from this approach is that there doesn't seem to be a way to determine what other implicit assurances might drown out explicit assurances. 
Continuing the example, the system designer may come up with a well thought out planning-competence assurance, but failed to consider the effect of how the UGV appears -- it may be old, have loose panels, and rust holes. Generally, designers overlook implicit assurances (i.e. do not consider them explicitly in design) because they assume that they will have no effect (i.e. why does it matter if there are rust-holes if the UGV works?). This can stem from ignorance of human-AIA trust dynamics, or failure to identify which assurances are most important to users.
%
% \edit{...move to end...trim also -- not sure it's saying much...and feels out of place given next paragraphs}
% Any of these methods can be more or less effective based on the task and context in which they are used. 
% How should uncertainly be displayed (i.e. as a distribution, summary statistics, fractions or decimals)?  Unsurprisingly we find that the answer is `it depends' \cite{Chen2014-dk,Wallace2001-fm,Kuhn1997-qc,Lacave2002-cu}. 

    While it might be desirable, it is generally unreasonable and practically inefficient to attempt a study of \emph{every possible} assurance from an AIA to a user and then select the most important. Perhaps one way a designer might try to identify which assurances are important is to perform user studies, to obtain feedback about which characteristics of the AIA most affected user trust. An approach like this would help determine if explicit assurances are being picked up, and if there are implicit assurances that are overly influential or that overwhelm explicitly designed assurances. With such feedback, designers would have a realistic idea about whether their explicitly designed assurances are having the desired effect on user TRBs. We use the UGV road-network problem to illustrate: after designing an explicit assurance, the supervisor-UGV team could work together in a training mission. Afterwards, the supervisor could rank the different behaviors/properties of the AIA affected their trust in it. In this way, the critical implicit and explicit assurances will be identified. If the explicit assurance is near the top of the list of influencing assurances, then it is working; if not a re-design may need to occur. 
Of course, even this approach has its own caveats, as factors such as the experience of the user, or the nature of the information being displayed, must be taken into account, as these will affect the user's ability to interpret explicit assurances or extract implicit assurances on their own. Absent data for analyzing such considerations, the best that can be done is to select explicit assurances that will work for the largest group of typical users of the AIA. A sufficiently advanced AIA might also learn how best to communicate to individual users. %%, although such adaptability can also make it difficult to formally establish the efficacy of assurances via user studies. 

One final point is that several potential sources for explicit assurances lack well-established human-understandable expressions, and thus are not yet widely utilized as effective assurances. For example, it is unclear how an AIA can best express that it has been formally validated and verified for similar operational settings. 
Similarly, it is unclear how information related to random variables can be best communicated to users besides showing them histogram or probability distribution plots (only useful for 1 or 2 dimensional distributions), or displaying statistics such as means and variances. 
Investigating and understanding how such useful, but otherwise difficult to understand, types of information can be expressed as explicit assurances will be critical to enhancing human-AIA trust relationships. 

\subsubsection{Efficacy}
Some kinds of expression are very `one-dimensional' in that they only rely on one medium or method. This, again, has been seen in practice by the use of plotting a certain AIA performance variable value over time. Because of this, much of the research to date involves assurances that are not robust to loss in transfer, i.e. the approaches rely heavily on a specific medium and method to work, otherwise the whole assurance is rendered useless. 
Hence, the problem of robustly communicating assurances remains an open research question. 
An analogy can be made here to a person speaking to another person with their voice, while also making facial expressions and gestures with their hands, thus simultaneously utilizing several mediums/methods helps to ensure the effectiveness of an assurance. 
If the person instead tries to simply repeat the same message over and over many times to the other person in a monotone voice without changing facial expressions or making any other gestures, then this would be considered inefficient, especially if the message is not received or considered to be effective after the first attempt. %This raises the idea of efficiency in expression. 
    
    %...again this paragraph feels out of place and disjointed...going to try to rearrange the ideas here, since they don't make sense in current form, not clear what the point is...
    %%
%    Perhaps less obvious is a situation in which the user has to supplement an incomplete assurance. A user can create a mental model of the trustworthiness of an AIA capability based on repeated observations over time. Creating this mental model takes time/effort, and the model is prone to cognitive biases. 
 %   In this case the assurance is communicated slowly and indirectly. Generally, a highly effective assurance would have precise information communicated in a way that is easy for the user to perceive, with little loss. Whereas, an inefficient and ineffective assurance may be more vague and wasteful (i.e. repeating the same thing many differet times), and susceptible to loss in communication. The solutions to efficacy lay in selecting appropriate methods, and mediums for expression of the assurance, and by designing for appropriate levels of redundancy to ensure that the assurance is received.
We might also consider situations where a user has to supplement an incomplete assurance from an AIA on their own. 
For instance, in the UGV road network scenario, suppose the hypothetical planning-competence assurance discussed earlier is implemented in such a way that it is communicated slowly and indirectly back to the supervisor. 
The supervisor can create and use a mental model of the AIA's capabilities (based on repeated observations over time or previous interactions) to `fill in' what an AIA might mean when it tries to express this assurance (e.g. to anticipate what it might say or display next).  
Creating and leveraging this mental model takes time/effort on the part of the user, and the process of interpreting the assurance by guessing at its content or meaning opens becomes prone to cognitive biases. %(e.g. if the user becomes impatient and starts to incorrectly categorize/infer the meaning of assurances printed on a display, and then acting on the incorrect interpretations before they have even finished printing). 
Ideally, a highly effective assurance communicate in a precise and direct way that is easy for the user to recognize and understand, with little loss in content or ability to act correctly on the assurance in a given situation. 
The keys to efficacy lie in the selection of appropriate methods and mediums for assurance expression, and in the design of appropriate levels of redundancy to ensure that assurances are correctly received and interpreted in a timely manner.

\subsection{Observing Effects of Assurances} \label{sec:measuring_effects}
    Since assurances are meant to influence TRBs, it is important to quantify those effects so that:  1) the AIA system designer can understand how effective the assurances are; and 2) the AIA can observe and respond to/adjust the efficacy of its assurances. To our knowledge, there has not been any work that enables an AIA to observe user responses to assurances and then adapt behaviors appropriately (at least not in the trust cycle setting). 
    Yet, this capability is crucial for enabling AIAs to meet different user's needs. 
Theoretically, any method that is made for the designer to measure the effects of assurances could also be deployed by the AIA itself to assess the effects of assurances on user TRBs. 
The surveyed literature gives some insights into how that has been done to date; namely, there are two main approaches: (1)  Gather self-reported changes in trust from human users; and (2) Measure changes in user's TRBs. 
    
\subsubsection{Self-Reported Changes in Trust} Assessing self-reported changes in trust involves asking users to answer questions, such as `how trustworthy do you feel the system is?'; or `to what extent do you find the system to be interpretable?', either while using a system or afterwards \cite{Mcknight2011-gv,Muir1996-gt,Wickens1999-la,Salem2015-md,Kaniarasu2013-ho}. These kinds of questions are useful in verifying whether the assurances are having the expected effects. It is not unreasonable to imagine that an AIA might be equipped to ask users questions about their trust, process those responses, and modify assurances appropriately.

Self-reports are the most useful when trying to understand the true effects of an assurance. Does a certain assurance, assumed to affect `situational normality', actually do that? 
Does displaying a specific plot actually convey information about `predictability'? 
There is much room for research in this area, which can be used to inform the selection of the methods of assurance. 
However, changes in self-reported trust do not always result in changes in TRBs \cite{Dzindolet2003-ts}. From the AIAs perspective this means that --- unless the object of the assurances is to make the person's level of self-reported trust change --- the assurances may not be providing any tangible benefit. 
As previously discussed, a more concrete objective for designing assurances in human-AIA interaction is to elicit appropriate TRBs from the human user. 
From this perspective, measuring changes in TRBs is the more direct and objective approach to assessing effectiveness.% of assurances.

\subsubsection{Measuring Changes in TRBs} Researchers often measure how long AIAs are able to run under full autonomy, before the autonomy is turned off by users \cite{Freedy2007-sg,Desai2012-rc}. 
Other researchers assess user's willingness to cooperate with AIAs \cite{Salem2015-md,Wu2016-ei,Bainbridge2011-pl}. 
A more ideal metric is the likelihood that users will use certain AIA capabilities `appropriately'. 
However, this is more difficult to formally define/calculate in different situations. 
As a concrete example for the UGV road network problem, %%%there is not an option to `turn off' the UGV's autonomy --but the user could switch off the planning feature...
the remote supervisor can make decisions such as accepting a plan or policy formulated by the UGV, or switching off the autonomous planner to provide their own plan to be implemented by the UGV. 
In this situation, the effect of assurances might be measured by how likely the operator is to accept a generated plan, instead of overriding it (recall that the goal may not be to have the generated plan accepted 100\% of the time, but rather that it be accepted with respect to how appropriate it is in a given context).

In practical application, assurances designed to lead the user to believe that the AIA is more competent, predictable or reliable than the user initially believed do not achieve their objectives if the user doesn't treat the AIA any differently than before/without the assurances. 
This assumes that it is possible for appropriate TRBs to be defined and observable in the first place. 
If, for example, an appropriate TRB hypothetically involves user verification of a sensor reading, can the AIA perceive whether or not such behavior takes place? 
%\edit{...good following: need to revise/polish...}
If the user is queried about this, can the user always be trusted to provide an honest/correct response or behave appropriately? 
%Is there a way to verify the user behavior is actually appropriate? 
This issue has gained notoriety with the current generation of autonomous cars, where users still need to attentively sit in the driver's seat in case the vehicle cannot perform correctly. This underscores the importance of designing methods for perceiving (in)appropriate TRBs. 
%
% \subsection{The Imprecise Nature of Assurances} \label{sec:imprecise_nature}
    % Due to the nature of trust (and humans in general), a single assurance might be targeted at influencing the competence dimension of trust, but it may also have effects on other dimensions. As an example an assurance that targets predictability may also have an affect on the probability of depending.
%
    % Besides being difficult to separate effects on a single user, individual users are different as well. Thus no assurance will have an identical effect when given to two separate users. This makes it difficult to have precise effects on user trust behaviors.
%
    % One might attempt to mitigate this uncertainty by using expressions that are more precise than others, such as displaying a probability distribution rather than on a maximum likelihood. This gets into some considerations about how the presentation of information affects the ability of a human to understand.

