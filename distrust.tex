\subsection{A Note Regarding Distrust}
    The treatment of assurances in this survey are based, in part, on a model of trust. For completeness it is important to mention distrust. As reviewed and discussed by \citet{Lewicki1998-ox}, and formalized in \cite{McKnight2001-hm,McKnight2001-gz} low trust is not the same as distrust, neither is low distrust the same as trust. \citet{McKnight2001-gz} suggest that ``the emotional intensity of distrust distinguishes it from trust''. They explain that distrust comes from emotions like: wariness, caution, and fear. Whereas, trust stems from emotions like: hope, safety, and confidence. Trust and distrust are orthogonal elements that define a person's TRB towards a trustee. \citet{Lewicki1998-ox} list several behaviors that stem from different combinations of trust and distrust; these are shown in Figure \ref{fig:distrust_table}.

    \begin{figure}[!htbp]
        \centering
        \includegraphics[width=0.6\textwidth]{Figures/distrust_table}
        \caption{Trust--Distrust table from \cite{Lewicki1998-ox}}
        \label{fig:distrust_table}
    \end{figure}

    In \citet{McKnight2001-gz} models for trust and distrust are proposed, and in later studies empirical evidence to validate some of the dimensions of both models was performed. \citet{McKnight2002-qx} is an empirical study that validates the model and quantifies the relationships between the model dimensions, and \citet{McKnight2004-vv} investigates and quantifies the relationship between dispositional distrust and web-site usage. Finally, \citet{McKnight2006-ce} revisits the position of \citet{McKnight1998-ty} and concludes that research had largely confirmed the model.

    % \textbf{add a small mention of McKnight's definition that distrust is the total opposite of trust?}

    In this survey distrust was not considered; this was in a effort to reduce the scope, which is already at risk of being too large. However, it must be made clear that any \emph{complete} treatment of trust relationships, and for our purposes, designed assurances, must consider the dimensions of distrust as well as those of trust. 
    
    Distrust has been shown to vary with perceived risk \cite{McKnight2004-vv}. This means that the implicit assumption of this paper is that risk in 1-on-1 human-AIA relationships is held constant. We claim that this is a realistic assumption for many practical applications.

