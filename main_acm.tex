\listfiles
% \RequirePackage{rotating}
% \documentclass[format=manuscript]{acmart}
\documentclass[format=manuscript, screen,review=true]{acmart}
%\documentclass[format=acmsmall, screen]{acmart}

% bibliography
% using biblatex isn't very smart because journals are pretty picky. Instead use the default natbib package
% \usepackage[backend=biber,style=trad-abbrv,citecounter=true]{biblatex}
% \addbibresource{References.bib}
%
% \renewcommand{\finentrypunct}{%
  % \addperiod\space
  % (Cited \arabic{citecounter}~time\ifnumequal{\value{citecounter}}{1}{}{s})%
% }

%\setcitestyle{super,sort&compress}
% \citestyle{acmauthoryear}
\usepackage{booktabs} % For formal tables
\usepackage[ruled]{algorithm2e} % For algorithms
\usepackage{subcaption}
\usepackage[printwatermark]{xwatermark}
\usepackage{xcolor}
\usepackage{graphicx}
\usepackage{tikz}
\usepackage{nameref}

% \newsavebox\mybox
% \savebox\mybox{\tikz[color=red,opacity=0.3]\node{1st Round Revisions};}
% \newwatermark*[
  % allpages,
  % angle=45,
  % scale=6,
  % xpos=-35,
  % ypos=30
% ]{\usebox\mybox}

% Metadata Information
\acmJournal{CSUR}
\acmVolume{01}
\acmNumber{01}
\acmArticle{01}
\acmYear{2018}
\acmMonth{01}

%\acmBadgeL[http://ctuning.org/ae/ppopp2016.html]{ae-logo}
% \acmBadgeR[http://ctuning.org/ae/ppopp2016.html]{ae-logo}

% Copyright
% \setcopyright{acmcopyright}
\setcopyright{acmlicensed}
% \setcopyright{rightsretained}
%\setcopyright{usgov}
%\setcopyright{usgovmixed}
%\setcopyright{cagov}
%\setcopyright{cagovmixed}
\begin{CCSXML}
<ccs2012>
<concept>
<concept_id>10002944.10011122.10002945</concept_id>
<concept_desc>General and reference~Surveys and overviews</concept_desc>
<concept_significance>500</concept_significance>
</concept>
<concept>
<concept_id>10003120</concept_id>
<concept_desc>Human-centered computing</concept_desc>
<concept_significance>500</concept_significance>
</concept>
<concept>
<concept_id>10003120.10003121.10003122.10003332</concept_id>
<concept_desc>Human-centered computing~User models</concept_desc>
<concept_significance>300</concept_significance>
</concept>
<concept>
<concept_id>10003120.10003121.10003122.10003334</concept_id>
<concept_desc>Human-centered computing~User studies</concept_desc>
<concept_significance>300</concept_significance>
</concept>
<concept>
<concept_id>10003120.10003121.10003129</concept_id>
<concept_desc>Human-centered computing~Interactive systems and tools</concept_desc>
<concept_significance>300</concept_significance>
</concept>
<concept>
<concept_id>10003120.10003123.10011758</concept_id>
<concept_desc>Human-centered computing~Interaction design theory, concepts and paradigms</concept_desc>
<concept_significance>300</concept_significance>
</concept>
<concept>
<concept_id>10003120.10003123.10011760</concept_id>
<concept_desc>Human-centered computing~Systems and tools for interaction design</concept_desc>
<concept_significance>300</concept_significance>
</concept>
<concept>
<concept_id>10003120.10003130.10003131</concept_id>
<concept_desc>Human-centered computing~Collaborative and social computing theory, concepts and paradigms</concept_desc>
<concept_significance>300</concept_significance>
</concept>
<concept>
<concept_id>10003120.10003121.10003124</concept_id>
<concept_desc>Human-centered computing~Interaction paradigms</concept_desc>
<concept_significance>100</concept_significance>
</concept>
<concept>
<concept_id>10003120.10003121.10003128</concept_id>
<concept_desc>Human-centered computing~Interaction techniques</concept_desc>
<concept_significance>100</concept_significance>
</concept>
<concept>
<concept_id>10003120.10003145.10003147</concept_id>
<concept_desc>Human-centered computing~Visualization application domains</concept_desc>
<concept_significance>100</concept_significance>
</concept>
<concept>
<concept_id>10010147</concept_id>
<concept_desc>Computing methodologies</concept_desc>
<concept_significance>300</concept_significance>
</concept>
<concept>
<concept_id>10010147.10010178</concept_id>
<concept_desc>Computing methodologies~Artificial intelligence</concept_desc>
<concept_significance>300</concept_significance>
</concept>
<concept>
<concept_id>10010147.10010257</concept_id>
<concept_desc>Computing methodologies~Machine learning</concept_desc>
<concept_significance>300</concept_significance>
</concept>
<concept>
<concept_id>10010405</concept_id>
<concept_desc>Applied computing</concept_desc>
<concept_significance>300</concept_significance>
</concept>
<concept>
<concept_id>10010405.10010432.10010439</concept_id>
<concept_desc>Applied computing~Engineering</concept_desc>
<concept_significance>300</concept_significance>
</concept>
<concept>
<concept_id>10010405.10010455.10010459</concept_id>
<concept_desc>Applied computing~Psychology</concept_desc>
<concept_significance>300</concept_significance>
</concept>
<concept>
<concept_id>10010405.10003550.10003555</concept_id>
<concept_desc>Applied computing~Online shopping</concept_desc>
<concept_significance>100</concept_significance>
</concept>
<concept>
<concept_id>10010405.10010455.10010458</concept_id>
<concept_desc>Applied computing~Law</concept_desc>
<concept_significance>100</concept_significance>
</concept>
<concept>
<concept_id>10010405.10010476.10010936</concept_id>
<concept_desc>Applied computing~Computing in government</concept_desc>
<concept_significance>100</concept_significance>
</concept>
</ccs2012>
\end{CCSXML}

\ccsdesc[500]{General and reference~Surveys and overviews}
\ccsdesc[500]{Human-centered computing}
\ccsdesc[300]{Human-centered computing~User models}
\ccsdesc[300]{Human-centered computing~User studies}
\ccsdesc[300]{Human-centered computing~Interactive systems and tools}
\ccsdesc[300]{Human-centered computing~Interaction design theory, concepts and paradigms}
\ccsdesc[300]{Human-centered computing~Systems and tools for interaction design}
\ccsdesc[300]{Human-centered computing~Collaborative and social computing theory, concepts and paradigms}
\ccsdesc[100]{Human-centered computing~Interaction paradigms}
\ccsdesc[100]{Human-centered computing~Interaction techniques}
\ccsdesc[100]{Human-centered computing~Visualization application domains}
\ccsdesc[300]{Computing methodologies}
\ccsdesc[300]{Computing methodologies~Artificial intelligence}
\ccsdesc[300]{Computing methodologies~Machine learning}
\ccsdesc[300]{Applied computing}
\ccsdesc[300]{Applied computing~Engineering}
\ccsdesc[300]{Applied computing~Psychology}
\ccsdesc[100]{Applied computing~Online shopping}
\ccsdesc[100]{Applied computing~Law}
\ccsdesc[100]{Applied computing~Computing in government}


% DOI
\acmDOI{0000001.0000001}

% some useful shortcut commands
\newcommand{\Pl}{\textbf{Pl}}
\newcommand{\R}{\textbf{R}}
\newcommand{\nisarcomm}[1]{{\color{red} (NRA: #1)}}
\newcommand{\brettcomm}[1]{{\color{blue} (BWI: \textbf{#1})}}
\newcommand{\edit}[1]{{\color{blue} #1}}
\newcommand{\hlr}[1]{{\color{red} #1}}

\received{November 2017}
%% 35 pages with references!!!
% Document starts
\begin{document}
% Title portion
\title{``Dave\ldots I can assure you \ldots that it's going to be all right \ldots''} 
\titlenote{HAL 9000, \textit{2001 A Space Odyssey}, full quote: ``Just what do you think you're doing, Dave? Dave, I really think I'm entitled to an answer to that question. I know everything hasn't been quite right with me, but I can assure you now, very confidently, that it's going to be all right again.''
 }
\subtitle{A definition, case for, and survey of algorithmic assurances in human-autonomy trust relationships}
% \subtitlenote{Subtitle note}
\author{Brett W. Israelsen}
\authornote{Corresponding author}
    \orcid{0000-0003-1602-1685}
    \email{brett.israelsen@colorado.edu}
\author{Nisar R. Ahmed}
    \email{nisar.ahmed@colorado.edu}
    \affiliation{%
        \institution{University of Colorado Boulder}
        % \department{Ann and H.J. Smead Aerospace Engineering Sciences}
        \city{Boulder}
        \state{CO}
        \country{USA}
    }
    \affiliation{%
        \institution{%\href{https://cohrint.info/}
        {Cooperative Human-Robot Intelligence  Laboratory (COHRINT)}}
    }
    \affiliation{%
        \institution{%\href{http://www.colorado.edu/recuv/}
        {Research and Engineering Center for Unmanned Vehicles (RECUV)}}
    }

\begin{abstract}
    People who design, use, and are affected by autonomous artificially intelligent agents want to be able to \emph{trust} such agents -- that is, to know that these agents will perform correctly, to understand the reasoning behind their actions, and to know how to use them appropriately. 
    Many techniques have been devised to assess and influence human trust in artificially intelligent agents. However, these approaches are typically ad hoc, and have not been formally related to each other or to formal trust models. This paper presents a survey of \emph{algorithmic assurances}, i.e. programmed components of agent operation that are expressly designed to calibrate user trust in artificially intelligent agents. 
    Algorithmic assurances are first formally defined and classified from the perspective of formally modeled human-artificially intelligent agent trust relationships. Building on these definitions, a synthesis of research across communities such as machine learning, human-computer interaction, robotics, e-commerce, and others reveals that assurance algorithms naturally fall along a spectrum in terms of their impact on an agent's core functionality, with seven notable classes ranging from integral assurances (which impact an agent's core functionality) to supplemental assurances (which have no direct effect on agent performance). Common approaches within each of these classes are identified and discussed; benefits and drawbacks of different approaches are also investigated. %%This spectrum and set of classifications can be directly useful to designers of advanced artificially intelligent agents, who are concerned about trust and appropriate use. %%Remaining questions and opportunities for future research are also presented.
\end{abstract}

\keywords{human-computer trust, interpretable machine learning, explainable artificial intelligence, transparency, accountability, fairness, algorithmic assurances}

\thanks{The authors acknowledge the helpful feedback of Michael Mozer and Eric Frew, as well as that of the reviewers. This work was funded by a research gift from Northrop-Grumman Aerospace Systems and by the Center for Unmanned Aircraft Systems (C-UAS), a National Science Foundation Industry/University Cooperative Research Center (I/UCRC) under NSF Award No. CNS-1650468 along with significant contributions from C-UAS industry members.}

\maketitle

%%%%%%%%%%%%%%%%%%%%%%%%%%%%%%%%%%%%%%%%%%
\section{Introduction}\label{sec:introduction}
\section{Introduction}
    As technology becomes more advanced, those who design, use, and are affected by it in other ways want to know that it will perform correctly, and understand why it does what is does, and how to use it appropriately. In essence, people who interact with advanced technology want to be able to trust it appropriately, and then act on that trust.

    In interpersonal relationships, and otherwise, humans act largely based on trust. For example, a supervisor asks a subordinate to accomplish a task based on several factors that indicate they can trust them to accomplish that task. When consumers make purchases, they do so with trust that the product will perform as promised. Likewise, when using something like an autonomous vehicle, the user must be able to trust it appropriately in order to use it properly.

    With the rapid advancement of the capabilities of intelligent computing technology to do tasks that were previously assumed to be too complicated for computers, there has been much recent discussion regarding how humans can trust this technology -- although the connection to trust is not always made explicit, per se. This discussion has taken place both in public \cite{Spectrum2016-jv,DeSteno2014-cq,Cranz2017-yh,Cassel2017-tn,Danks2017-sb,Wagner2016-ck}, business \cite{Banavar2016-nm, Khosravi2016-ke,Moody2017-vd,Rudnitsky2017-in,Benioff2016-tc,Tankard2016-rk}, and academic \cite{Groom2007-bz,Lloyd2014-bb,Goodrum_2016-fm,Foley2017-qj,Ghahramani2015-yq,Castelvecchi2016-mr} settings.

    Those who discuss \emph{how} to trust a specific technology are really referring to the need for some indication of the appropriate level of trust to give said technology. In other words, it is desirable to \emph{design} capabilities and methods for intelligent technology which help us achieve appropriate levels of trust in that technology. These capabilities and methods are collectively referred to as \emph{assurances}.
    
    Specifically, this survey investigates what assurances an Artificially Intelligent Agent (AIA) can provide to a human user in order to affect their trust. The colloquial definitions of `appropriate use', `assurance', `AIA', and `trust' should suffice for now to give the reader a general idea of the motivation; more formal definitions will be presented in section \ref{sec:background}. It is the author's position that there are many researchers, from different disciplines, who will potentially be interested in this work. This group includes those who are interested in working with, trusting, interpreting, understanding, and/or regulating AIAs.

    Figure \ref{fig:SimpleTrust_one_way} is a simple diagram of the trust cycle that exists between a human user and an AIA (justification for the existence of this cycle will be presented later). Simply, the user's trust is affected by assurances that in turn affect the user's behaviors in interacting with the AIA (e.g. to trust AIA with responsibilities, or not). To fully understand and appreciate the importance of assurances, one must have a more formal understanding of each component in figure \ref{fig:SimpleTrust_one_way}.

    This paper provides an overview of the components of figure \ref{fig:SimpleTrust_one_way}. It then turns a more focused attention to assurances, and investigates some of the related research that has been done to date. From this survey of literature, properties and classifications of assurances are created, and directions and considerations for further research are presented.

    \begin{figure}
        \centering
        \includegraphics[width=0.7\textwidth]{Figures/SimpleTrust_one_way}
        \caption{Diagram depicting the simple one-way trust development relationship between a human user and an AIA. Based on a user's level of trust they take certain actions (e.g. give AIA commands), these commands can lead the AIA to certain actions and/or to provide assurances to the user in order to affect their trust.}
        \label{fig:SimpleTrust_one_way}
    \end{figure}

    Some of the novel contributions of this paper include: a detailed description and definition of assurances in general human-AIA relationships; an argument that trust-related behaviors should be used to measure the effect of assurances on user trust; presenting the idea that assurances can be either explicit or implicit; suggestions for promising research directions for explicit assurances. To this end, section \ref{sec:background} provides definitions for each of the terms. In section \ref{sec:methodology} we discuss the methodology used when compiling this survey. Afterwards, section \ref{sec:survey} will discuss the current landscape of assurances that exist in the literature. Section \ref{sec:synthesis} discusses some important trends of the literature, and then attempts to use those to inform considerations when classifying and designing assurances. Finally, conclusions are presented in section \ref{sec:conclusions}.


%%%%%%%%%%%%%%%%%%%%%%%%%%%%%%%%%%%%%%%%%%
\section{Motivation and Background} \label{sec:background}
%background.tex

%%\nisarcomm{my TODOs: 1. Re-Merge Sec 2 and 3; 2. trim down text/figures to `bare essentials' and definitions required for survey, leave enough material for sensible  transition, refer to deeper dive into definitions/background for later on in discussion/future work; 3. move other material/definitions not needed for survey to new section 4 (currently 5) -- need transition to revisit and expound on these...}
%As per the title, in this paper we present a case for the necessity of understanding assurances, and in doing so also present a definition. Finally we perform a survey of methods by which assurances have been calculated.

%%\subsection{Motivation} \label{sec:motivation}
    %What do people who talk about `comprehensible systems', `interpretable learning', `opaque systems', and `explainable AI' really, fundamentally, care about?
    %Whether formally acknowledged or not, human designers and users want \emph{assurances} to help them appropriately `trust' autonomous and artificially intelligent systems. 
    %%\nisarcomm{for me todo: some of this might be redundant given the intro, basically repeating the same question/points made above?...might be worth trimming down a bit or moving to back for `recap'...include foglights here for this section}
    The need for designed assurances has grown considerably in recent years, as the advanced capabilities of intelligent systems have become more difficult to comprehend and predict \cite{Doshi-Velez2017-xy, Weller2017-zx, Lipton2016-ug, Gunning2017-ih}. 
    When researchers discuss concepts like `comprehensible systems', `interpretable learning', `transparent systems', and `explainable AI', they are looking for deliberately designed mechanisms to help designers and users appropriately `trust' autonomous and artificially intelligent systems as they perform their tasks. 
    For example, many systems are designed to learn from extremely large amounts of data and are expected to regularly perform on never before seen data -- yet, it is rarely obvious if such data conforms to assumptions made at design time. 
    Other systems are designed to perform tasks that are too `dirty, dull, and dangerous' for humans; the separation of users from these tasks often makes it difficult for them to understand whether these systems are performing as desired. 
    The authors, for instance, are interested in the design of unmanned robotic vehicle systems that operate in concert with remote human operators in uncertain dynamic environments. 
    Since operators will generally not be computer scientists or roboticists, it is desirable for such systems to communicate in ways that help operators properly use their abilities in scenarios featuring unexpected or incomplete information, time-critical decisions, and risky outcomes \cite{Hutchins2015-if, Sweet2016-tz}. 
    %%The hope is that, in doing so, the performance of the team can be improved by using the capabilities of both human and unmanned vehicles for intelligent reasoning and decision making \cite{Hutchins2015-if}. 
    This application is explained in more detail later in relation to Figure~\ref{fig:SimpleTrust_one_way}. 
    These issues also have relevance and analogues in other applications of autonomous artificial intelligence, robotics, machine learning and decision making/support systems ~\cite{Garcia2015-rs,Otte2013-oo,Sugiyama2013-ci,Amodei2016-xi}, e.g. for scientific data analysis ~\cite{Faghmous2014-og}, public policy and medicine ~\cite{Wagner2016-ck,Jovanovic2016-gw} and cognitive assistance \cite{Gutfreund2016-xe}.
    
    %To this end, 
    Many studies, models, metrics and methods have been created around these systems to address these issues. 
    Some fields have formally and explicitly considered trust between humans and specific forms of intelligent technology, e.g. e-commerce, automation, and human-robot interaction. However, these research efforts have focused largely on developing formal cognitive and psychological models of trust, rather than system behaviors or algorithms that designers can exploit as assurances. %to calibrate trust.  
    %mainly focused on implicit `uncontrolled' properties of the systems that affect trust. 
    Other fields that have explored assurance design only provide an informal connection to trust and applications to other disciplines, so it is unknown how effective their developed assurances might be in practice, or what principles ought to be considered for other kinds of autonomous and artificially intelligent systems. 
    This paper surveys assurance metrics and methods across relevant application domains, with the goal of %bridging gaps between them by 
    identifying common principles, approaches and questions related to trust-based interaction. %opportunities where each might benefit from another. 
    %Generally, humans want to `trust' 
    %\footnote{Of course, at this point `trust' is quite an imprecise term, and will be defined more formally in Section~\ref{sec:trust}}
    %the tools and systems that they create. 
    To begin with, definitions for the trust cycle in Fig. ~\ref{fig:SimpleTrust_one_way} are given to ground the concept of assurances. A running application example is then provided to compare/contrast technical ideas and implementations of algorithmic assurances throughout the survey in Section \ref{sec:synthesis}. 
     
     
    
    %%Other systems are designed to perform tasks that might take humans entire lifetimes to complete \nisarcomm{...so...and? This is only half an argument as written...also seems orthogonal to what we want to focus on, i.e. robotics and autonomous vehicles} 
    %Below is a sample of some application areas and possible reasons why they have an interest in creating trustworthy systems:

%\nisarcomm{moved to discussion part at end of paper...}
%    \begin{description}
%        \item [Artificial Intelligence/Machine Learning:] There is a need to interpret how and why theoretical AIA models function, in order to know they are being applied correctly and to design new approaches to overcome weaknesses in the existing methods~\cite{Garcia2015-rs,Otte2013-oo}.    
%        \item [Interpretable Science:] Scientists need to be able to trust the models created using data analysis, and be able to draw insights from them. Scientific discoveries cannot depend on methods that are not understood~\cite{Faghmous2014-og}.
%        \item [Reliable Machine Learning:] It is critical to have safety guarantees for AIAs that have been deployed in the real world. Failing to do so can result in serious accidents that cause loss of life, or significant damage~\cite{Sugiyama2013-ci,Amodei2016-xi}.       
%        \item [Public Policy:] Governments are beginning to enforce regulations on the interpretability of certain algorithms in order to ensure that citizens can understand why AIA driven services make the decisions and predictions that they do. A specific example are the algorithms deployed by credit agencies to approve/reject loans~\cite{Wagner2016-ck}.
%        \item [Medicine:] Medical professionals need to understand why data-driven models give predictions so that they can choose whether or not to follow the recommendations. While AIAs can be a very useful tool, ultimately doctors are liable for the decisions they make and treatments they administer~\cite{Jovanovic2016-gw}.
%        \item [Cognitive Assistance:] Systems are being designed as aids for humans to make complex decisions, e.g. searching and assimilating information from databases of legal proceedings. When an AIA presents perspectives and conclusions as data summaries, it must be able to also present justifying evidence and logic~\cite{Gutfreund2016-xe}.
%    \end{description}


%    \brettcomm{Important---Whether formally acknowledged or not, human AIA system designers and users want \emph{assurances} to help them appropriately trust AIAs. There are a few research fields that have formally and explicitly considered trust between humans and technology, e.g. e-commerce, automation, and human-robot interaction. However, these research efforts have mainly focused on implicit properties of the systems that affect trust. Conversely, there are other fields that have informally considered how to affect the trust of designers and users via explicitly designed assurances. However, due to their informal treatment of trust, it is unknown and unclear how effective these designed assurances might be in practice, or what principles ought to be considered when designing assurances for general AIAs. The goal of this paper is to survey these areas and, in doing so, help bridge the gap between them by identifying common goals and approaches, as well as highlighting where the different disciplines might benefit from each other.}

%\subsection{Motivating Application and Basic Definitions} \label{sec:mot_example}

%\nisarcomm{split and move example to after the definitions?? need to shrink, and note that this will be used as a running example throughout the paper to illustrate/ground/relate key ideas...}


%%%%%%%%%%%%%%%%%%%%%%%%%%%%%%%%%%%%%%%%%%%%%%%%%%%%%%%%%%%%%%%%

\subsection{Trust Cycle Definitions}
%definitions.tex
%%%%%%%%%%%%%%%%%%%%%%%%%%%%%
\subsubsection*{Artificially Intelligent Agents} %%\label{sec:aias}
%aias.tex
%%%\nisarcomm{merge, trim...move before example?}

    %Intelligent technology spans a wide spectrum of capabilities. With regards to autonomous systems, these 
    %Autonomous systems can describe anything from 
    %%a thermostat
    %a simple assembly line robot to the fabled HAL 9000. 
    While the main interest of the authors is geared towards human trust in `advanced' technology for dynamic robotic decision making under uncertainty, this survey we will take a more holistic view and use the term Artificially Intelligent Agent (AIA) to encompass a broad range of technologies that can be considered `autonomous'. 
    %%, to gather generally applicable insight. 
    An AIA is defined here as an agent that acts on an internally/externally generated goal, and possesses, to some extent, at least one of the capabilities shown in Fig.~\ref{fig:AIcapabilities} ~\cite{Russell2010-wv,Nilsson2009-rp,Luger2008-vf}. 
    While an AIA can describe anything from a simple assembly line robot to the fabled HAL 9000, this definition underscores the idea that many assurances that exist for one set of AIAs can be adapted and generalized for use in other AIAs. 
    %%For instance, chi-square consistency tests for Kalman filter state estimation algorithms for autonomous vehicle perception, target tracking and control systems \cite{Bar-Shalom2001-tg}. 
    In other words, this definition sets a scope for the bodies of research that are likely to have investigated assurances and assurance principles, which can be extended to any intelligent computing system. 
    The range of AIA capabilities also helps establish what kinds of assurances might be needed in future systems. 
    For example, assurances for an AIA that only carry out planning tasks will probably differ in design or implementation from assurances for an AIA that only carry out perception tasks. 
    
    %\nisarcomm{for me todo: one more important idea that hasn't been articulated here, but that we rely on from very first sentence of paper, is that AIA is seen as subordinate delegate to human -- this is key idea since it bridges definition of AIA to need for understanding user trust. can cite Chris Miller's work?}
    
    It should be noted that an AIA is assumed to operate with a degree of autonomy that is \emph{delegated} by a user. That is, an AIA is self-directed and self-sufficient in its task to the extent that the user's `intent frame' (desired goals, plans, constraints, stipulations and/or value statements) can be met by the AIA, regardless of how it actually accomplishes these. %or what intermediate decisions it needs to make to achieve these ends consistently. 
    Following \citet{Miller2014-av}, this view of autonomy as a delegation relationship refines need for `transparent AIAs' by avoiding a contradiction of purpose that stems from an otherwise naive interpretation. From a naive standpoint, one could argue that if AIAs are developed primarily in the first place to alleviate the burden of complex reasoning and other undesirable workloads by removing users from the task at hand entirely, then this purpose is undercut by exposure and explanation of sophisticated AIA inner workings to the user. 
    However, if AIAs are subordinates that are delegated tasks by users (who must still act as supervisors), the meaning of `transparency' shifts away from concern over how exactly an AIA accomplishes a task, towards concern over whether or not an AIA can execute the task as per the user's intent frame. 
    This delegation-based view naturally sets up the question of user trust in AIAs. 
    %
    %To this end, an artificially intelligent system needs to possess at least some of the capabilities shown in Figure~\ref{fig:AIcapabilities}~\cite{Russell2010-wv,Nilsson2009-rp,Luger2008-vf}. 
    %Some might argue that it is also necessary to add other categories like creativity and social intelligence~\cite{Tao2005-kh}. 
    %\brettcomm{SEEMS TO DETRACT---}Some of these categories are also not clearly separable; for instance, where does the capability to `plan' end, and `reasoning' begin? Nevertheless, these capabilities are conceptually useful in defining an AIA:     
%    \begin{description}
%        \item[Artificially Intelligent Agent (AIA):] an agent that acts on an internally/externally generated goal, and possesses, to some extent, at least one of the capabilities shown in Fig.~\ref{fig:AIcapabilities} ~\cite{Russell2010-wv,Nilsson2009-rp,Luger2008-vf}. .
%    \end{description}

	\begin{figure}[t!]%[thbp]
    	\centering
     	\includegraphics[width=0.55\textwidth]{Figures/AI_capabilities}
    	\caption{Set of possible AIA capabilities.}
        \label{fig:AIcapabilities}
    \end{figure}

%    The broad range of AIAs implied by this definition is most usefully viewed in terms of scope and adaptability. Scope refers to the range of possible applications for an AIA: does it have a small number of specialized application, or can it be used in many different applications? Adaptability refers to the ability of the AIA to become better at executing its goal over time. Low adaptability has often been associated with `weak AI' whereas high adaptability is often associated with `strong AI'.  Figure~\ref{fig:StrongWeak} depicts these axes for some (real and fictitious) AIAs.
%
%	\begin{figure}[htbp]
%    	\centering
%     	\includegraphics[width=0.7\textwidth]{Figures/strong_weak_narrow_broad.pdf}
%    	\caption{Illustration of the range of systems encompassed by the AIA definition. Horizontal axis reflects the scope of the AIA, the vertical axis reflects the adaptability of the AIA. \nisarcomm{todo: REMOVE}}
%        \label{fig:StrongWeak}
%    \end{figure}

%    Arguably, we might instead have used the term `artificial intelligence' (AI) instead of AIA. However, `AI' carries too much ambiguity (in its fullest meaning, it would possess all capabilities from Figure~\ref{fig:AIcapabilities}, and more). AIA allows the broad inclusion of \emph{any} system in the adaptability/scope plane. The research discipline of machine learning (ML) is a subset of the AI research landscape. Individual ML algorithms might be thought of as being a narrowly scoped AI that is contained within only one of the AIA capabilities. 
%
    %One might also question the need to define AIAs in the first place. 
    %This is to aid in the search for and understanding of assurances. As will be shown later, different methods of assurance can be found over the entire range of AIAs, so that an automation system such as a factory robot might be able to use similar assurances -- or more generally, similar principles of assurance -- as might a self-driving car, and vice-versa. The capabilities of AIAs (Fig.~\ref{fig:AIcapabilities}) are the sources of assurances; in other words, assurances cannot exist without some grounding set of AIA capabilities. 
%
 %   This definition %, while broad, is still useful because it 
 %   encompasses many systems that are typically described as `artificially intelligent'. 
%%%%%%%%%%%%%%%%%%%%%%%%%%%%%
\subsubsection*{User Trust} %%\label{sec:trust}
\subsection{User Trust} \label{sec:trust}
    In designing assurances that affect trust-based user behaviors, it is critical to know what drives those behaviors. Because of this, some time must be spent to understand what trust is. 

    Trust is critical in interpersonal relationships, and it affects the dynamics of intelligent multi-agent systems as simple as one-on-one personal interactions  \cite{Lewicki2006-hj}, to more complicated ones such as financial markets and governments \cite{Fukuyama1995-un}. Consequently, researchers in psychology, sociology, and economics have historically sought to understand the fundamental principles of trust, each with the aim of understanding their field better \cite{Gambetta1988-pi}. Moral philosophers have also thought intently about the topic \cite{Baier1986-im}.

    Due to wide interest spanning many disciplines it is difficult, if not impossible, to write a succinct definition of trust that would appease all interested parties. Besides that, trust is actually a very broad concept that evades precise definitions at a high level. However the following definition, adapted from \cite{McKnight2004-vv}, is broad enough to avoid too much contention:

    \begin{description}
        \item [Trust:] a psychological state in which an agent willingly and securely becomes vulnerable, or depends on, a trustee (e.g., another person, institution, or an AIA), having taken into consideration the characteristics (e.g., benevolence, integrity, competence) of the trustee.
    \end{description}

    \subsubsection{Trust between AIAs and humans?}
        Trust is generally understood to exist between people. Is it possible for a human to enter into a trusting relationship with an AIA?
        % In their paper regarding important human factors that should be considered when designing autonomous machines \cite{Sheridan1984-kx} are seemingly the first to discuss the idea that trust relationships between humans and autonomous systems are important, and to suggest that humans need some assurance that the ``commands will be carried out properly''. They also mention the idea that ``there needs to be an accurate perception of [the autonomous system's] trustworthiness''. Finally they suggest that ``appropriate criteria for trust need to be studied to develop a theory of trust in supervisory control''.
        % Perhaps motivated by \citeauthor{Sheridan1984-kx}\cite{Sheridan1984-kx}, a few years later \cite{Muir1987-mk}, and later in more detail \cite{Muir1994-ow}, create a psychologically based model of trust that considered the ``component expectations of trust'' of \cite{Barber1983-yc} and the dynamic evolution of trust from \cite{Rempel1985-sg}, to make a framework for studying trust in human-machine relationships.
        % \citet{Muir1996-gt} reported the results of two experimental studies to investigate the validity of her proposed model. She claims that these were the first experiments to explicitly ask "operators to rate their trust in automated equipment", and to see if they could do so under normal operating conditions. She found that operators were able to rate their trust in the automation, and that the level of trust changed based on different performance characteristics of the automated system. In her own words: "These results suggest that operators' subjective ratings of trust and the properties of the automation which determine their trust, can be used to predict and optimize the dynamic allocation of functions in automated systems".
        That humans actually do feel trust towards machines has been experimentally confirmed several times in research using common subjective psychological questionnaires. Some examples include: \citet{Muir1996-gt,Reeves1997-ad,Groom2007-bz,Mcknight2011-gv,Riley1996-qm,Bainbridge2011-pl,Kaniarasu2012-mo,Salem2015-md,Desai2012-rc, Freedy2007-sg, Wang2016-id, Inagaki1998-cl, Kaniarasu2013-ho}. 

        Several academic experiments have investigated the possibility of trust existing between humans and (according to the terminology of this survey) AIAs. All found that some level of trust can be formed in such relationships. For instance, \citet{Lacher2014-yc} points out that people trust an AIA at different levels. For example, an operator would have different perspectives on trust based on their level of interaction with the AIA. The designer of an AIA would also trust the AIA differently than an end user, due to the differing nature of the trust relationship from one to the other. 
        % \citet{Lankton2008-ct} claims, and finds some support for the idea, that trust in technology is fundamentally different from interpersonal trust between humans. They demonstrate the validity of the hypothesis by using a survey of 427 college students regarding Facebook. However, the authors point out that this study was based on a single set of survey data about facebook, and may not be unbiased or apply to other technologies. Beyond this, it is the author's opinion that the `fundamental differences' they point out are not that divergent from the human-human trust model.

        \citet{Tripp2011-cq} investigate the variation of trust between humans and different levels of technology. They run experiments with three different levels of technology: Microsoft Access, a recommender system, and Faceobook. They found that `human-like' trust applied more to Facebook, while `system-like' trust applied more to MS Access. They conclude that if the system is `human enough', then a human trust model is appropriate.

        Given this research, we will take the position of presenting a human-human trust model and use it as a basis for human-AIA trust -- with the understanding that the strength of the model varies with the complexity of the AIA. In other words, some features of the model will have varying level of significance over the range of adaptability and capability of AIAs.

\subsubsection{A Model of Human-AIA Trust}
        We now present a model of human-AIA trust, which will cast insights on assurances that will be discussed later. It should be noted that this model is being presented as \emph{one possible model} that can be helpful in understanding assurances -- it is neither the only model, or a perfect model. As research advances, such models will likely continue to evolve, and the ideas of assurances will naturally evolve as well.
            % This has been recently attempted in the context of human-AIA relationships \cite{Lahijanian2016-nd}, but with an overly simplistic reduction of trust. The reality is that trust is extremely complex, and so dealing with it in the setting of human-AIA relationships is going to be complex.

        In work relating to business management, \citet{McKnight1998-ty}, and later \citet{McKnight2001-fa}, performed what is, arguably, the first multi-disciplinary survey and unification of trust literature, which also condensed it into a single typology. The resulting model is shown, with some minor adaptations, in Figure \ref{fig:UserTrust}. The figure illustrates the three categories that make up a human's trust. There are causal arrows that connect the different components. The `Dispositional Trust' block is generally considered by psychologists, and deals with long-term psychological traits that develop in a person from childhood. The `Instututional Trust' block is generally studied by sociologists, and represents the level to which a person trusts social/commercial structures. Finally, the `Interpersonal Trust' block is deals directly with one-on-one relationships and can generally fluctuate more quickly than the other two.

        \begin{figure}[htbp]
            \centering
            \includegraphics[width=0.9\textwidth]{Figures/UserTrust}
            \caption{Interdisciplinary trust model proposed by \citet{McKnight2001-fa}. The three main categories are delineated, and corresponding disciplines that are interested are listed within parentheses. Connections indicate a causal relationship. The suggestion regarding time scales of the development of trust is the author's addition, trust development is discussed more in \cite{Lewicki2006-gp}, and \cite{Lewicki2006-hj}}
            \label{fig:UserTrust}
        \end{figure}

        In the context of AIAs the components of the three categories from figure \ref{fig:UserTrust} are defined as follows:

        \begin{description}
            \item [Disposition to Trust:] The extent to which one displays a consistent tendency to be willing to depend on AIAs in general across a broad spectrum of situations and persons
            \item [Institution-Based Trust:] One believes that regulations are in place that are conducive to situational success in an endeavor
            \item [Trusting Beliefs:] One believes that the AIA has one or more characteristics beneficial to oneself
            \item [Trusting Intentions:] One is willing to depend on, or intends to depend on, the AIA even though one cannot control its every action
        \end{description}

        Each of these main categories of trust has components defined in Figure \ref{fig:Assurance_classes}. These components were defined through the compilation of many research studies across research disciplines, and because of this represent the most accurate notion of the components of trust available. It is asserted here that these are the principal drivers of TRBs, the targets at which assurances must be directed.

        \begin{sidewaysfigure}[htbp]
            \includegraphics[width=8in]{Figures/Assurances.pdf}%
            \caption{Assurance targets based on the component definitions of the main categories of trust: `Disposition to Trust', `Institution-Based Trust', `Trusting Beliefs', and `Trusting Intentions'}
            \label{fig:Assurance_classes}
        \end{sidewaysfigure}

%%%%%%%%%%%%%%%%%%%%%%%%%%%%%
\subsubsection*{Trust-Related Behaviors} %%\label{sec:trbs}
\subsection{Trust-Related Behaviors} \label{sec:trbs}
Something that is well accepted among researchers of all disciplines is that trust ultimately leads to some kind of behavior or action; this idea was highlighted by \citet{Lewis1985-pr}.  \citet{McKnight2001-fa} call these `trust-related behaviors` (TRBs), which is the term that will be used in this survey. In the case of a human-AIA relationship, the author is concerned with TRBs could include the kinds of tasks the human user assigns to the UGV such as accepting and following through on its plan, or directing that a new plan be made.

\subsubsection{Calibration of Trust-Related Behaviors}
    Trust is not a quantity that can be objectively measured. Rather, its relative magnitude must be observed through changes in TRBs, or qualitative surveys \cite{Muir1996-gt}. Of these two approaches, TRBs are the more objective measure due to the fact that people are not always consistent in their ratings, and may sincerely feel different levels of trust while performing similar TRBs. \citet{Parasuraman1997-co} were interested in understanding the use of automation by humans, and defined terms related to  along these lines. Here it is proposed that, by extension, those terms also apply to the relationship between humans and AIAs. Within this scope the definitions are as follows:%\footnote{a third term `Abuse' was left out because it concerns the decision by humans to install automation in an environment that was inappropriate. It is my belief that this can be wrapped under the umbrella of `Misuse'},
    
    \begin{description}
        \item [Misuse:] The over-reliance on an AIA -- which could manifest itself in expecting too much accuracy from and AIA, or believing that it can be applied in an application that it was not designed to function in
        \item [Disuse:] The under-utilization of and AIA -- which could be manifest in a user turning off the AIA, or failing to use all of its capabilities 
        % \item [\textbf{Abuse}:] I don't agree that abuse should be included -- we should talk about it
        \item [Abuse:] Inappropriate application of automation (where \emph{application} in this case means the choice to deploy an AIA in a certain context, such as the choice to use a quad-copter underwater).
        % \nisarcomm{I don't think you should leave it out -- there is an important and subtle difference between over-reliance and misapplication of a capability/technology -- assurances to prevent the former are thus not going to be the same as assurances that try to avoid the latter. Also, the installation of automation in an environment that is inappropriate doesn't necessarily translate the same way to autonomous systems -- in particular, since autonomous systems often have to be self-directed and make decisions under uncertainty on their own. i.e. using the Google car to tow another vehicle behind it would not necessarily count as over-reliance/misuse, but could possibly be construed as abuse -- it is in the right environment, and it is still carrying out a function that any rational user might possibly conceive as a capability it ought to possess, since the car is still essentially carrying out the same function as before and doing what any car might be expected to do -- however, a user may not be aware of the fact that placing another vehicle in tow behind the car might introduce perception errors and planning errors, since the vehicle may not be able to fully account for the altered dynamics or changed sensor field, etc. -- it was not really designed for towing tasks, but someone might get lucky and get away with pulling it off a few times. In contrast, using the Tesla autopilot on the highway (even though it results in a crash while driver is watching a movie) is not abuse but clearly a case of misuse/overreliance -- the system encountered a known failure condition in a scenario in which it was expected to be deployed (highway driving), but the operator/user was not prepared to either anticipate or deal with the situation -- here the driver was habituated into overtrusting the system, since nothing bad happened before, i.e. no failures were encountered for the system in its intended operating condition and environment. }
    \end{description}

    Again referring to the diagram in Figure \ref{fig:SimpleTrust_one_way} the AIA must influence the user's TRBs by way of assurances. We propose that the goal of an AIA should be that the user should not misuse, disuse, or abuse it. In other words, the space of all trust-related behaviors can be made up of misuse, disuse, abuse, and appropriate behaviors (all behaviors that aren't misuse, disuse, or abuse).

    % \nisarcomm{in light of previous comments above, it is worth spelling out in just a few sentences what difference in translation of misuse, abuse, and misuse is when going  from automation to AIAs -- i.e. to what extent to the concepts carry over directly, and which parts or considerations might have to be modified to accommodate the more autonomous nature of AIAs? This really is where some of the insights on the relationship between trust and assurances start to emerge.}
    
    % To be more formal, let the total set of TRBs as $\mathcal{T}$. Then as subsets of $\mathcal{T}$ define the set of misuse actions as $\mathcal{M}$, the set of disuse actions as $\mathcal{D}$, and the set of abuse actions as $\mathcal{A}$. Next, define the total set of inappropriate TRBs $\mathcal{I}$ as the union of $\mathcal{I} = \mathcal{M}\cup \mathcal{D}\cup\mathcal{A}$. Having defined the set of inappropriate actions, the set of appropriate TRBs can be defined as $\mathcal{U}$, the compliment of the set of inappropriate TRBs $\mathcal{U} = \mathcal{I}^\prime$. This is illustrated in Figure \ref{fig:appropriate_use}, where the set of appropriate actions $\mathcal{U}$ is the gray colored area (i.e. all TRBs \emph{not} in either of the three sets of inappropriate TRBs). \textbf{This is probably too complicated, I could probably just say that $\mathcal{U}$ is the set of all actions not included in either of M,A, or D}.
    %
	% \begin{figure}[htbp]
        % \centering
         % \includegraphics[width=0.4\textwidth]{Figures/misuse_disuse_abuse}
        % \caption{Graphic representing the total space of user actions, in which the inappropriate uses $\mathcal{M}$, $\mathcal{D}$, and $\mathcal{A}$ lie. The set of inappropriate uses $\mathcal{I}$ is the union of $\mathcal{M}$, $\mathcal{D}$, and $\mathcal{A}$. The appropriate set of actions $\mathcal{U}$ is the compliment of $\mathcal{I}$, or the part of $\mathcal{T}$ that does not include $\mathcal{I}$.}
        % \label{fig:appropriate_use}
    % \end{figure}
    
    Given this definition, in order to ensure that humans use AIAs appropriately, it is critical that the user TRBs be calibrated to elicit behaviors that are within the set of appropriate behaviors, which can only be done by influencing the user trust. This is a point that, to some extent, has been informally mentioned in \citet{Muir1994-ow,Muir1987-mk,Lillard2016-yg,Lee2004-pv,Hutchins2015-if}.

    A critical oversight of other researchers who mention `calibration' is that they suggest calibrating \emph{trust} as opposed to TRBs. \citet{Dzindolet2003-ts} studied the effect of performance feedback on user's self-reported trust, and found that it increased; however the appropriate TRBs toward the system did not reflect the level of self-reported trust. This shows the danger of calibrating ``trust'', as opposed to calibrating the TRBs.

    Calibrating TRBs focuses on concrete and measurable behaviors that are universally applicable. In contrast, calibrating trust involves influencing a quantify that is directly immeasurable, and that, when measured indirectly, is subject to the biases and uncertainties of humans, along with inherent differences between different users. Viewing the task from this point of view, the findings of \citeauthor{Dzindolet2003-ts} are not surprising.

    % \textbf{I'd like to go off on a little rant about this, but I'm not sure if it is appropriate. There is a TON of literature that talks about calibrating trust. Calibrating trust is asking for trouble, when we actually care about TRBs. VERY LITTLE RESEARCH  (none?) HAS BEEN DONE CONSIDERING ONLY TRBS, IT IS MOSTLY JUST SELF-REPORTED TRUST, WHICH DZINDOLET HAS SHOWN TO BE SHAKY GROUND} \nisarcomm{This is a very interesting point -- definitely worth mentioning as a conclusion/takeaway of this survey and worth restating/discussing in a bit more detail at the end for open opportunities and future work, etc.}

    It is desirable for AIAs to be designed in order to encourage appropriate TRBs, as opposed to the alternative of purposefully misleading users misuse or abuse. There is a valid argument that many of today's AIAs that ignore (or whose designers ignored) TRBs and assurances can be `unwittingly malicious' in that they do not actively attempt to guide user's TRBs to lay within the space of appropriate TRBs.
    % A benevolent AIA the user's TRBs should be appropriate \nisarcomm{I am not sure what you mean here -- when I read this sentence out loud, it doesn't really come together: what do you mean by the `perspective of a benevolent AIA'?? Also, `benevolent AIA' seems to ascribe some sort of intentionality to an AIA, which was not really discussed or mentioned at all before in your AIA definition}.
    % This in contrast to a malicious AIA that tries to manipulate the TRBs of a human to overlap with misuse, \edit{abuse} or disuse to some extent \nisarcomm{again, unclear what precisely you mean here when reading this out loud -- are you trying to make the point that simply getting a user to trust a system more is not meaningful or beneficial on its own, since it's not hard to dupe people into trusting something that's actually bad for them? (note this doesn't necessarily square with the idea you seem to be proposing of `tricking' people into abusing, misusing, or disusing AIAs -- the connotation of abuse, misuse and disuse is more that these are tendencies that people tend to converge to on their own; misuse is arguably the only one that people might actually get actively tricked into in terms of having their trust levels being actively manipulated by an AIA; interestingly, disuse can also arise in `neutral/benevolent' systems, e.g. the Mars Rovers: disuse of the advanced features onboard the rovers arises due to cultural influences at NASA, i.e. don't let rovers make decisions that can't be backtracked or accounted for fully, since the cost of failure is too high and thus the engineers are highly risk averse -- the rover autonomy itself arguably plays no role in shaping this disuse}. There is a valid argument that many of today's systems that ignore TRBs and assurances are unwittingly malicious in that they do not actively attempt to guide user's TRBs to lay within the space of appropriate TRBs .

    % Generally trust between a human and AI could be depicted as in Figure \ref{fig:SimpleTrust_two_way}, where each has TRBs that must be calibrated, and each provides certain feedback, which will be called assurances, in order to do so. In a more simple scenario, where the AI implicitly trusts the human user the trust relationship can be depicted as shown in Figure \ref{fig:SimpleTrust_one_way}, where only the user has TRBs that are being calibrated.

	% \begin{figure*}[htbp]
        % \centering
        % \begin{subfigure}[t]{0.48\textwidth}
            % \centering
            % \includegraphics[width=0.95\textwidth]{Figures/SimpleTrust_two_way}
            % \caption{Diagram showing a general case of a two-way trust relationship between an AI and a human. Arrows that are not connected to boxes represent some action outside of the trust loop.}
            % \label{fig:SimpleTrust_two_way}%
        % \end{subfigure}
        % \hfill
        % \begin{subfigure}[t]{0.48\textwidth}
            % \centering
            % \includegraphics[width=0.95\textwidth]{Figures/SimpleTrust_one_way}
            % \caption{Diagram illustrating a general one-way trust relationship between a human and an AI. In this case the AI has, what could be considered perfect trust in the user.}
            % \label{fig:SimpleTrust_one_way}%
        % \end{subfigure}
        % \caption{Feedback Loops For One and Two-way human-AI Trust Relationships}
        % \label{fig:SimpleTrust}
    % \end{figure}
    
    % Figure \ref{fig:SimpleTrust_dist} serves as a simple example illustrating the possible disparity between the user TRB distribution and the appropriate TRB distribution. In this case assurances would be used to minimize the difference between the two distributions.

	% \begin{figure}[htbp]
        % \centering
         % \includegraphics[width=0.4\textwidth]{Figures/SimpleTrust_dist.png}
        % \caption{Diagram illustrating the point that a hypothetical user TRB distribution might not match the appropriate TRB distribution. In this case the AI should provide assurances in order to minimize the difference between the two.}
        % \label{fig:SimpleTrust_dist}
    % \end{figure}
%

%%%%%%%%%%%%%%%%%%%%%%%%%%%%%
\subsubsection*{Assurances} \label{sec:assurances}
An assurance is an AIA property or behavior that can either increase or decrease user trust. The term `assurances' is perhaps earliest used in the context of human-AIA relationships by \citet{Sheridan1984-kx}. \citet{McKnight2001-fa} allude to this kind of feedback in e-commerce relationships as `Web Vendor Interventions'. \citet{Corritore2003-gx} refer to assurances as `trust cues' that can influence how online users trust e-commerce vendors. \citet{Lee2004-pv} discuss `display characteristics', which are methods by which an autonomous systems can communicate information to an operator. More recently, \citet{Lillard2016-yg} provided a formal definition of assurances for autonomous systems that is similar to the one used here. 

\begin{figure}[t]%[htpb]
    \centering
    \includegraphics[width=0.6\linewidth]{Figures/RefinedTrust_one_way.pdf}
    \caption{Figure depicting the details of the human-AIA trust cycle.}
    \label{fig:refined_trust}
\end{figure}

Assurances can be classified in several different ways. One way to classify an assurance is by its \emph{Information Source:}. Assurances must be informed by some kind of information, whether that means real-time observation of TRBs in order to have feedback, or well accepted concepts of cognitive science as guiding principles of design.
Another approach is to identify the \emph{Source/Target} pair: In a human-AIA trust relationship, assurances link the AIA to the user. The user has multi-dimensional trust in the AIA (see Fig.~\ref{fig:Assurance_classes}), and each AIA capability has multiple dimensions of `trustworthiness'. In designing an assurance it is useful to explicitly identify the source capability, and the target trust dimension (i.e. a certain assurance may have been designed as a `planning-competence').
An assurances can be considered \emph{Component} or \emph{Composite:} A component assurance stems from one AIA capability to one trust dimension. A composite assurance originates from multiple AIA capabilities to one trust dimension.
Another consideration is whether the assurance is \emph{Tutoring} or \emph{Telling:} An assurance that is dynamic to the different characteristics, and experience of users is a `tutoring' assurance. It is designed to help a user learn, over time, to trust appropriately. Conversely, all other assurances are `telling' in that they are static in regards to separate users.
\emph{Mode of Expression:} Assurances can also be classified by their mode of expression. This includes the method and medium by which the assurance is expressed.
There are many open questions regarding each of these categories; they are discussed further in Sec.~\ref{sec:future_work}, regarding future work.

\emph{Level of Integration:} Herein the `level of integration' of assurances are surveyed. This is useful because it addresses a natural consideration in the design process of AIAs; it also encapsulates well the key approaches that are in use. In this context `integration' refers to the level of effect the assurance has on the core functions of the AIA. As an example: an assurance that, if missing, greatly effects the AIA functionality is considered integral to the AIA. Conversely, a missing assurance that has no effect on the AIA functionality is not integral; we also call this `supplemental'. Between these two extremes there is a natural continuum of integration on which we can classify the different algorithmic approaches to designing assurances; we do so in Sec.~\ref{sec:synthesis}.


\subsubsection*{Summary}
Each of the elements of Fig.~\ref{fig:SimpleTrust_one_way} has been defined in this Section (\ref{sec:trust_definitions}). Figure~\ref{fig:refined_trust} illustrates how these concepts fit together. In this document algorithmic assurances are surveyed through the lens of their `Level of Integration'; more detailed discussion of the other elements are found in Sec.~\ref{sec:future_work}.
 %%re-merging and heavily editing this subsection

%%%%%%%%%%%%%%%%%%%%%%%%%%%%%%%%%%%%%%%%%%%%%%%%%%%%%%%%%%%%%%%%

\subsection{Recurring Example Application}
    \nisarcomm{for me todo: edit, trim down a bit; what makes this dull, dirty, dangerous task?}
    It is useful to have a concrete grounding example. For instance, consider a toy application motivated by the ``VIP escort'' problem \cite{Humphrey2012-lr}, which also serves as a useful analog for surveillance and reconnaissance operations. 
    A robotic unmanned ground vehicle (UGV) acts as the lead vehicle of a small convoy attempting to navigate its way through a road network that is monitored by accessible unattended ground sensors (UGS). The road network also contains a hostile pursuer that the UGV is trying to evade while exiting the road network. The pursuer's location is unknown but can be estimated using intermittent data from unattended ground sensors (UGS), which only sense portions of the network and can produce false alarms. The UGV's decision space involves selecting a sequence of actions (i.e. go straight, left, right, back, stay in place). The UGS data, the UGV's motion, and the pursuer's behavior are all stochastic, and the problems of decision making and sensing are strongly coupled: some trajectories through the road network allow the UGV to localize the pursuer before heading to the exit (but incur a high time penalty); other trajectories afford rapid exit with high pursuer location uncertainty (increasing the risk of getting caught by the pursuer, which can follow multiple paths). 

    One of the many approaches to modeling and solving this decision making problem for the UGV (the AIA in this example) is to discretize time and vehicle spatial variables, in order to construct a partially observable Markov decision process (POMDP) model of the task. The ideal POMDP solution is an optimal UGV action selection policy that will, \emph{on average}, maximize some utility function whose optimum value coincides with desirable UGV behaviors (i.e. avoiding the pursuer and reaching the exit quickly). Although analytically and computationally intractable to find exactly, POMDP policies can be approximated by any number of sophisticated approaches.

    A human supervisor monitors and interfaces with the UGV during operation. The supervisor does not have a detailed knowledge of how the UGV functions or makes decisions (e.g. according to the POMDP), but can interrogate the system, modify the decision making stance (such as `aggressive' or `conservative'), and provide information and advice to the UGV. In this situation, the supervisor could benefit from the UGVs capability to express confidence in its ability to escape given the current sensor information, and work with the AIA to modify behavior if necessary. 
    
	\begin{figure}[t]%[htbp]
    	\centering
     	\includegraphics[width=0.5\textwidth]{Figures/RoadNet}
    	\caption{Application example of unmanned ground vehicle (UGV) in a road network, trying to evade a pursuer, using information from unmanned ground sensors (UGSs), as well as information and decision making advice from a human operator.} %The operator's actions towards the UGV are trust-based.}
        \label{fig:RoadNet}
    \end{figure}

%%\nisarcomm{for me todo: garnish with some pics of UGV/operator? should UGV being doing something else like gathering intel from UGS, etc.? what makes this task so dull, dirty, dangerous?}

    In this scenario the trust-cycle terms can be defined as follows: \textit{Artificially Intelligent Agent:} the UGV, which must make decisions under uncertainty with little information (the pursuer is only observed sporadically and does not follow a known course); \textit{Trust:} The operator's willingness to rely on the UGV when the outcome of the mission is at stake, such as in a scenario where the UGV is carrying a VIP or some other valuable payload; \textit{Trust-Related Behaviors:} operator's behaviors that indicate trust (or lack thereof) in the UGV, including the information and commands they give to the UGV. This might take the form of approving/rejecting the UGV's decisions, or real-time communication and adjustments of the UGV's information based on what the operator receives from other intelligence sources; \textit{Assurances:} properties and/or behaviors of the UGV that have an effect on the operator's trust. These can include communicating the probability of success for a given policy, or communicating that the mission is not proceeding as expected.

%%\subsection{Related Work}\label{sec:rel_work}
    
%%    \nisarcomm{shrink down...move earlier and merge? too much detail here...}
    
%    The issues of interpretability, explainability, and transparency of AI has garnered considerable recent attention \cite{Doshi-Velez2017-xy, Weller2017-zx, Lipton2016-ug, Gunning2017-ih}. The related body of work has many interesting and important insights regarding the need for transparency, but does not formally acknowledge the role of trust in human decision making, or how interpretability and transparency affect the trust of those who use AIAs. Yet, this work is beneficial because it draws the attention of researchers to this critical area, and it initially formalizes the problem for those actively researching assurance design.

%    \citet{Lillard2016-yg} addressed the role of assurances in the relationship between humans and AIAs, and provides much of the foundation used here for describing the relationships between assurances and trust in human-AIA interaction. Here, the framework for analyzing assurances is presented in a way that is both more general and more detailed, albeit with the same end goal of being applied in a very similar end application. For instance, we consider the full trust model presented by \citet{McKnight2001-fa}, whereas \citeauthor{Lillard2016-yg} only consider a subset of the trust model.

%    Relating to the work of \citet{McKnight2001-fa}, who constructed a typology of interpersonal trust, we adopt the position that (besides being applicable to the e-commerce industry as originally intended) their trust model also applies to relationships between humans and AIAs (as in \cite{Lillard2016-yg}). They refer to something called `vendor interventions' that are related to assurances in this paper. One small, but important distinction from vendor interventions is that assurances cannot directly have an affect on the user trust-related behaviors (TRBs). This is an important point, since we are considering scenarios in which the human and AIA are working together, and not ones where the human is strictly dependent or steered/guided by an AIA. 
    
%    While assurances are defined by \citeauthor{Lillard2016-yg}, and mentioned by \citeauthor{McKnight2001-fa}, and also \citeauthor{Corritore2003-gx}, we investigate in detail how assurances fit within the trust cycle (from Figure~\ref{fig:SimpleTrust_one_way}), survey what methods of assurance have been and are currently being used, then present a refined definition and classification of assurances. In essence, whereas others have noted the \textit{existence} of assurances, we now directly consider the question: What, exactly, \textit{are} assurances, and how can they be \textit{designed}? 
    %To that end we survey literature that formally considers trust between humans and AIAs, as well as literature that informally investigates trust through topics like transparency, explainability, and interpretability, and begin to distill ideas for practically designing assurances in human-AIA trust relationships. \brettcomm{I DON'T THINK THERE'S MUCH DISCUSSION OF THIS ANYWHERE\ldots}


%%%%%%%%%%%%%%%%%%%%%%%%%%%%%%%%%%%%%%%%%%
\section{Survey of Algorithmic Assurances} \label{sec:synthesis}
\section{Survey of Methods for Calculating Assurances} \label{sec:synthesis}
    \brettcomm{BRING IN SOME MATERIAL FROM LIPTON, GUNNING, WELLER, AND DOSHI-VELEZ}
    The definition given in Section~\ref{sec:assurances} gives a good definition and classification of assurances, but doesn't give very many practical insights into how assurances might be designed. In this section we review the related literature to understand what approaches have been used in order to calculate assurances.
    
    Figure~\ref{fig:refined_assurances} combines material from Section~\ref{sec:definitions} in order to make a more detailed version of Figure~\ref{fig:SimpleTrust_one_way}. In this section we perform a survey of the methods of calculation that have been used in literature. We have distilled the common concepts down to the following different approaches that will be discussed further.

    \begin{itemize}
        \item Integral Assurance
        \begin{itemize}
            \item Human-like Behavior
            \item Value Alignment
            \item Interpretable Models
            \item User Interaction
        \end{itemize}
        \item Supplemental Assurance
        \begin{itemize}
            \item User Assessment
            \item AIA Self-Assessment
            \begin{itemize}
                \item Quantify Uncertainty
                \item Reduce Complexity
            \end{itemize}
        \end{itemize}
    \end{itemize}

    \begin{figure}[htbp]
        \centering
        \includegraphics[width=0.85\textwidth]{Figures/RefinedTrust_one_way}
        \caption{Detailed extension of Figure~\ref{fig:SimpleTrust_one_way}. The AIA, User, and User TRBs blocks are defined as in Section~\ref{sec:background} (with the exception of the `Perception' blocks added to the AIA and User boxes). The AIA Assurances box has been filled using insights from the survey. The grey shaded boxes will not be discussed in this work; we focus on elements that a designer can address given an AIA with a fixed set of capabilities.}
        \label{fig:refined_assurances}
    \end{figure}

\section{Methodology} \label{sec:methodology}
    In this survey, I attempt to look at research of those who are formally and informally addressing the idea of human-AIA trust. In particular, I focus on a ideas that might be applicable to the trust relationship between a single human user (User), and a single autonomous vehicle. While theoretically a two-way trust model could be considered, I will only be considering a one-way trust relationship, that is that the autonomy has perfect trust towards a user.

    It should be noted that it is almost impossible to perform a comprehensive survey of all assurances due to the broad nature of assurances in general. One could rightly argue that metrics like gain and phase margins are assurances for control engineers, as are training and test accuracy for machine learning practioncioners. However, it is my opinion that the somewhat narrow view of the surveyed literature does not significantly hinder the definition or classification of assurances.

    In order to find applicable research I first looked at papers that formally addressed trust and tried to create models of it; this with the aim of trying to understand how it might be influenced. Secondly, I looked at some historical research regarding trust between humans and some form of non-human entity. This mainly lead to e-commerce literature, automation literature, and human-robot interactions. Third, I investigated work regarding `interpretable', `comprehensible', `transparent', `explainable', \ldots and other types of learning and modeling methods. Finally, I searched for research disciplines that are investigating methods that would be useful as assurances, but of which trust is not the main focus.

    With this information, I try to make an informed definition and classification of assurances based off of empirical information of methods that are currently in use or being investigated. In doing so I was able to identify several areas that are open for further research. 

%%(i) assurance argument: what specifically is the assurance signal/rationale?  -- both value alignment and interpretable model approaches have the implicit aim of grounding user trust in an understanding of rational decision making processes; however, cognitive scientists and many AIA experts will point out, humans are not entirely rational actors. As such, what accounts for human-human trust?  Like interpretability, a precise definition is hard to pin down -- but perhaps essence can be best summed up as encoding AIA behaviors that avoid `inhuman-like' characteristics. This can be viewed as the algorithmic equivalent of avoiding the so-called `uncanny valley' in humanoid and social robotics and VR/AR domains. Indeed, the mere association of `human-like' behavior to an AIA can be enough to endow levels of trust in AIAs that are comparable to human-human relationships [Tripp, et al]. Basic idea then is to endow AIA with capabilities that can be interpreted by user as though they could have come from another human. 
%One example is provided by Wink Bennett's work for LVC-based simulation training of Air Force fighter pilots -- in the NotSoGrandChallenge, basic idea was to develop realistic enemy fighter plane AI that would keep trainee pilots honest, i.e. from finding loop holes in computer logic that would allow them to easily game the simulation and win predictably -- this requires not only intelligent mechanisms for adapting to different human trainees and exploring/exploiting their weaknesses through simulated combat interactions, but also requires AI to maintain some level of human credibility --- since trainees will eventually be up against real human fighter opponents, this application requires realistic human behavior for both trainers and trainees to trust the AIA is providing useful sim experience...
%%(ii) what is mechanism for generating appropriate TRBs? 
%Key idea here is to interpret/acknowledge the user's intent frame by producing suitable set of `expected' human behaviors. Assurances generated along these line tend towards improving an understanding of an AIA's predictability and situation normality, most often via verbal or non-verbal human interaction cues to express intent or uncertainty. This can be especially useful in complex problem settings where a utility-based or interpretable model-based understanding of the problem is insufficient for completely quantifying user preferences that can be used to inform `trustworthy' AIA responses (e.g. dog-fighting, or even highway driving [slowing down and coming to complete stop to yield right of way for pedestrians cross, using blinkers and horns to signal other drivers, etc.]...). This is also especially useful for interacting with users who are not privy to access `innards' of AIA, i.e. who cannot potentially act as all-seeing/all-knowing supervisors or operators, per se, but as in situ participants in task/environment that AIA must operate in... Caveat here is that users tend to fill in a lot of blanks on their own, so anthropomorphizing AIA can lead to other unintended consequences that are not present in value alignment or interpretable modeling -- namely, users might extrapolate other assurances or capabilities from AIA that it does not actually mean to signal or possess (e.g. eyes on quadcopter for signaling can't actually see anything...self-driving car cannot hear sounds or horns, even though it looks like a human-driven car and behaves as if it is being driven by a human...)
%%(iii) how can designers build/exploit for AIA assurances, i.e. what techniques available for human-like behavior?: wide variety of heuristics, but two salient categories pertaining to trust and assurances are 
%%(a) nonverbal communication of intent (e.g. physical gesturing, legible planning, turn signaling, etc.) 
%%(b) mannerisms: establishing implicit cues for interaction

\subsection{Human-Like Behavior} \label{sec:human_behavior}
In trying to enable AIAs to better participate in trust-relationships with human users, many have been inspired by the natural analog of human-human. Humans are used to forming trusting relationships with other humans. In essence they ask: what kinds of communication do humans use during interaction that helps them to understand and trust each other? This idea was investigated by \citet{Tripp2011-rx} who compared human trust in other humans against human trust in intelligent interactive technology, which in this case was represented by Microsoft Access, an intelligent recommendation assistant, and Facebook. They found that, as the technology becomes more `human-like', self-reported levels of trust in technology become more similar to levels of trust in other humans.
\brettcomm{picks up with (possibly) irrational behavior, not included in `value alignment'}

\citet{De_Visser2018-kd} specifically discusses different methods by which AIAs can be more human-like in order to `repair trust' with humans (here trust repair is roughly analogous to assurances, but focusing on re-building trust after it is lost). Among several other possibilities, they suggest that an AIA might repair trust by anthropomorphizing (responding using a human communication channel), or by explaining their actions. \brettcomm{Perhaps discuss other methods?}

\subsubsection{Common Approaches}
Generally, we do not have algorithms that describe how humans interact, and must settle for heuristics, or best attempts to create human-like behavior via algorithms. From a high level, there are two main ways that researchers have been addressing this challenge: Direct communication, and mannerisims.

\paragraph{Direct Communication:} \nisarcomm{this is a misnomer -- `nonverbal' communication is what you mean to say??}
In designing assurances that use direct communication, the communication can take many different forms. One popular approach is to use motion or gestures.\citet{Szafir2014-ok} investigated how to enable `Assisted Free Flyer' robots (quad-copters that are made to interact with humans in close spaces) to communicate by using gestures. In doing so they use `motion primitives' (a basic vocabulary of movements) that were inspired by animation \cite{Van_Breemen2004-rz}. In their evaluations of these primitives with human participants, they found that human users significantly found the AFFs to be more natural, and felt safer around them. Later \citet{Szafir2015-iy} also experimentally showed the effectiveness of using a quad-copter's `turn signal' lighting to help users more easily interpret the intended movements and actions. These works provide strong support for `natural communication' assurances aimed at predictability.

\citet{Dragan2013-wd} investigate `legible motion planning' (LMP). LMP is, in essence, planned robotic physical movements and gestures that, by themselves, convey intended actions and goals. They are meant to improve a user's ability to understand and predict where the robot is trying to move. Their insight is that legible motion is used by humans, and is important for situations in which a robot and person are collaboratively working in close proximity to each other. Similarly, in more recent work \citet{Kwon2018-xt} investigates calculating trajectories that convey `incapability', which is \emph{what} the AIA is trying to do, and \emph{why} it is unable to do so. \cite{Dragan2013-wd} calculated sub-optimal paths in order to communicate \emph{intent} of a gesture, by gathering data from human participants they were able to find the parameters for their cost function that best matched the expectations. Likewise \cite{Kwon2018-xt} tested several different objectives using user studies to guide the selection. See also \cite{Admoni2016-db} for related work.

As good example in the domain of natural language communication, at Google/IO 2018, `Google Duplex' (\cite{Google2018-eb}) was introduced through a demo where it called a business establishment to make a reservation. One of the striking characteristics of the system is that it was very difficult (if not impossible) to detect whether the Duplex voice was a human or not. This applied both for the words that is spoke and the accent of the voice. In order to achieve this impressive result, they trained a recurrent neural network (RNN) on anonymized phone conversation data.



\paragraph{Mannerisms:}
\citet{Salem2015-md} investigated the effects of autonomous task errors, task types, and `system personality' on cooperation and trust for humans who observed a domestic robot performing house tasks, such that the robot implicitly showed competence by its mannerisms and successes/failures during tasks. In this case the mannerisms and competency of the robot were completely under control and hard-coded into the system. Regardless, when participants were asked to cooperate with the robot on certain other tasks, the faulty operation of the robot was found to affect the self-reported trust levels of the participants.

\citet{Wu2016-ei} investigated how a person's decisions in a coin entrustment game are affected by their belief in whether they are competing against an AIA or another human player (which, unbeknownst to participants, was in fact an AI with some programmed human-like idiosyncrasies, e.g. variable wait times between turns). Trust in this context was measured directly by the number of coins a participant was willing to lose by putting them at risk to the other player. The experiment found that the participants trusted the AI opponent more than they trusted the `human' opponent; the authors suggest that this may be due to the perception that the AI opponent did not have feelings and operated in a more predictable and consistent `machine-like' way. Given that the `human' was an AI as well, this experiment illustrates that `machine-like' behavioral consistency can lead to implicit positive effects the trust of the participant in certain contexts.

Returning to the case of Google Duplex, another striking characteristic was that Duplex used mannerisims of speech such as saying `um\ldots', including pauses, and sometimes using shortened sentences, which made it even more difficult to distinguish between it and a real human. And during the demo calls, the human on the other end of the line was none the wiser, and trusted that they were in fact speaking to a human.
\brettcomm{Add something about etiquette?} \nisarcomm{this belongs in mannerisms}

\subsubsection{Grounding Example:}
In the case of the `VIP Escort' problem (described in Section~\ref{sec:mot_example}), human-like behavior might be used as an assurance in the following way:

We make the following assumptions

\begin{itemize}
    \item The UGV is about to begin an attempt at escaping the road-network
    \item The operator can observe all the actions of the UGV via video feeds at intersections
    \item The UGV has been designed with the ability to use gestures in order to indicate its `incapability' as in \cite{Kwon2018-xt}
\end{itemize}

As the UGV begins the escort problem, the human supervisor is monitoring progress. As the UGV reaches a certain intersection of the road network the supervisor expects the UGV to take a path $A$, but it does not. However, before choosing to take path $B$, the UGV made a movement that, to the operator, indicated that it considered attempting to traverse $A$. Due to the attempt the supervisor was able to surmise that the UGV wanted to take that path but couldn't due to some limitation.

\paragraph{\textbf{Discussion of Example:}} In this case the UGV is able to maintain appropriate trust of the supervisor because the supervisor was able to interpret the `gesture' that UGV was using. This highlights the assuring effects that human-like communication/behaviors can have on users.

\subsection{Value Alignment} \label{sec:value_alignment}


\subsubsection{Common Approaches:}

\paragraph{stuff:}


\subsubsection{Grounding Example:}
In the case of the `VIP Escort' problem (described in Section~\ref{sec:mot_example}), value alignment might be used as an assurance in the following way:

We make the following assumptions

\begin{itemize}
    \item The UGV has just begun an attempt to escape the road-network
    \item 
    \item 
\end{itemize}

\paragraph{\textbf{Discussion of Example:}} 

\subsection{User Interaction} \label{sec:user_interaction}
To date one of the more common approaches to engender trust in users has been to put the users `in-the-loop'. This has been, and still is, modus operandi in the automation industry and others. While some think that more advanced AIAs will `soon' be able to operate with little human involvement, in general those who have more practical experience with AIAs are more reserved.

In this section we review some methods by which engineers have made AIA assurances by making the performance of the system highly dependent on the user's participation. This includes work from disciplines such as human-robot collaboration, cooperative control, cooperative sensing, and others. Humans playing a significant role in the functionality of an AIA is analogous to a supervisor working `in the trenches' with those they supervise; in doing so they are able to provide feedback in real-time, lend their expertise, and better appreciate the decisions and outcomes of the team's work.

\subsubsection{Common Approaches:}
\nisarcomm{what are the main ideas to cover beyond the individual papers? first talk about what *could* be done in terms of human roles (humans as planners/controllers, sensors/perception augments, tutors/trainers, etc.) -- then discuss how each approach contributes to acting as an assurance for an AIA, using refs as specific examples -- otherwise a laundry list of papers doesn't really help organize or convey any part of your argument here }\brettcomm{In what ways can users interact? make a list. Humans as sensors, humans as controllers,\ldots} \brettcomm{In what ways can users interact? make a list. Humans as sensors, humans as controllers,\ldots}

In general, a human and AIA can interact on many different levels. At the extreme of the most fundamental extremes, the human might fully replace a system or sub-system of the AIA. For example, the human might act as the sensors for an AIA. On the other extreme, the human might have a very weak involvement in the core functionality of the AIA. There are bodies of literature that address humans as sensors, and humans as controllers of AIAs. There is also research in which humans and robots \emph{share} responsibility for different functions.

Enabling the robot and human to share or `fuse' information can have an effect on trust. \citet{Sweet2016-dw} investigate how to enable using humans as `soft' sensors, and then fuse that information into that of the `hard' robot sensors in order to improve and augment the robot's Bayesian state estimation capabilities. They apply their approach in a scenario called `cops and robots' where a single `cop' robot tries to locate `robber' robots. In this case the human acts as a deputy that remotely interacts with the system. The human can see security camera footage of the building in which the cop is searching, and can offer natural language feedback to the cop robot when appropriate. If the human offers information it can be fused into the cop robot's estimation model, but in the meantime the cop robot operates autonomously without assistance. Similarly, \citet{Tse2015-tz} consider a framework for robots and humans to share and fuse information in a cooperative context.
Similarly \citet{Tse2015-tz} consider a framework for robots and humans to share and fuse information in a cooperative context. \nisarcomm{so...? there's a lot more to be said here...}

In order to be able to design a system that can be useful to a human operator, \citet{Kaupp2008-yr,Kaupp2005-pk} empirically identify the appropriate level of automation for a system while taking into account the amount of interaction required by a human operator. In this case the robot has sensors of its own, but can also ask for user input when the value of information (VOI) is high enough (i.e. is it worth asking a human for information given that there is a cost?); they define the threshold VOI by human trials before deployment of the system in order to optimize the involvement of the human user.

\citet{Tellex2014-uc} consider an autonomous assembly robot that can detect when it has failures (conditions that don't match expectations based on internal models). When this occurs the robot requests help from the human user to resolve the problem. In this way the human and robot are dependent on each other to accomplish a task. Since the user knows that, if needed, the robot will ask for help they can more appropriately trust that unknown problems won't occur without them being informed.

\citet{Freedy2007-sg} studied how mixed-initiative human-AIA teams might have their performance measured, and examined the extent to which such teams can only be successful if ``humans know how to appropriately trust and hence appropriately rely on the automation''. They explore this idea by using a tactical reconnaissance scenario where human participants supervised an unmanned ground vehicle (UGV)  platoon with three levels of autonomous targeting/firing capability (low, medium, high); these levels were dependent on the experimental conditions. The operator needed to monitor the UGV in case it couldn't perform as desired; in such cases the operator could intervene to resolve the problem. Operators were trained to recognize signs of task failure, and to only intervene if they thought the mission completion time would suffer.

\subsubsection{Grounding Example:}
In the case of the `VIP Escort' problem (described in Section~\ref{sec:mot_example}), operator interaction might be used as an assurance in the following way:

We make the following assumptions

\begin{itemize}
    \item The UGV has just begun an attempt to escape the road-network
    \item An interface system exists by which the operator can receive and provide information to the UGV
\end{itemize}

The UGV is capable of operating autonomously, but also has the ability to ask for assistance or information when necessary. In this way the functionality of the UGV can be greatly improved via interaction with the user. As the user interfaces with the UGV and is able to provide feedback and information about the best known location of the pursuer based on information unavailable to the UGV they have more trust in the competence, predictability, and situational normality of the UGV.

\paragraph{\textbf{Discussion of Example:}} In this scenario the user is more immersed in the functioning of the UGV. Not only are they able to respond to queries from the UGV, but they can also provide direct observations as well. Subsequently, the user feels more immersed in the functioning of the UGV and is more cognizant of appropriate TRBs.

%%for NRA TODO: edit/rewrite as follows:
%%(i) assurance argument: what specifically is the assurance signal? as opposed to value alignment, the signal here is some visualization or access to the decision-making process itself...the user is not just looking at the end result or utility of the AIA decision making process, or at behaviors that try to align the AIA's behavior/utility with the user's intent frame. Rather, the user gets some form of access to dig deeper into the AIA behavior to see what informs the utility function, or more generally the different steps along the way that generate different parts of the AIA's actual behavior (as opposed to a simplified or approximate explanation of its behavior, which is covered later)...in almost all cases, the model or intermediate model outputs are generally displayed to or accessed by users in some graphical manner, with some accompanying text/dialog/annotations...
%%(ii) what is mechanism for generating appropriate TRBs? if user gets to dig in, then they get to assess competency and predictability components of trust ; if models are interpretable, then user can in principle understand how to `do the task themselves in exactly the same way an AIA would do it' (i.e. by programatically imitating AIA's behavior on task)-- this follows a theory of mind argument, in that it should allow user to build a better `mental model' of what the AIA would consider to be `situational normality' and how AIA would handle different situations. Caveat here is that this could go south if the user mis-interprets or only understands part of what the AIA is actually doing, either in terms of interpretable model or the nature of the task itself -- this is a significant risk in highly complex or specialized problems where user may not have sufficient training or expertise. This also poses concerns for how/when user can access such models -- unlike value alignment (where user accesses assurances only through behavior of AIA itself), user has more freedom in deciding when and how to peek under the hood -- relates somewhat to problem of information visualization discussed later, except that here the information being given to the user are the actual AIA algorithms, as opposed to by products or after effects of those capabilities
%%(iii) how can designers build/exploit interpretable models for AIAs, i.e. what techniques available for interpretable model building? This gets to summarizing what Brett has written below -- can keep papers clustered more/less as they are, but need to repeat the assurance arguments above: 
%%(a) assessment of interp: post hoc approaches that map models to domain knowledge, place constraints on model inputs, or examine handling of known nonlinear effects;  
%%(b) creating interpretable models: compress this section into key ideas with citations of relevant papers/methods: model could be interpretable from outset like a decision tree (at the expense of performance), or could be less interpretable from outset and then post-fit with another interpretable model. Important thing to note is that most of these are designed around machine learning and pattern recognition problems -- and as such require that the tasks being performed by the AIA to be mappable to these kinds of problems; this generally means that some form of training data is required for supervised or unsupervised learning. ALSO: second approach of building a more interpretable model around an intial model technically counts as a `supplementary' method, in that this interpretable -- DOES THIS ALREADY COME LATER IN SURVEY???
%%(c) human-in-loop learning -- very similar to value alignment, except that not necessarily only concerned with learning utility, but also other kinds of features or useful tidbits that aid in AIA decision making -- follows same theory of mind argument from before: user will have same mental grounding as AIA, so trust can be calibrated appropriately to get better understanding of predictability, competence, and situation normality...Julie Shah's work on learning LTL rules from data?? can maybe leave (c) out for time being....
\subsection{Interpretable Models and Processes} \label{sec:interp_models}
\nisarcomm{add `processes' to label of this category...}
Another way to provide assurances about AIA conformance to user intent frames is to expose the models and algorithmic processes governing its actions directly to the user. If these models and processes also happen to be easy for users to interpret, then the user can (ideally) acquire a well-formed and highly predictive `theory of mind' for the AIA's behavior, with little or no effort . 
\citet{Doshi-Velez2017-xy} give an argument for why interpretability is critical in AIA systems since interpretability `is used to confirm other important desiderata of [machine learning] systems'. 
Yet, perhaps unsurprisingly, `interpretability' and the attendant desiderata still elude formal universally accepted definitions. 
%%Based on our survey, and building on the `easily accessible theory of mind' notion: we define an interpretable model or process here as one which can be fully comprehended by a typical human user, to the degree that the user herself could (given access to the model/process, and enough time and resources for performing required intermediate computations) precisely predict the results that an AIA would obtain on a given task. In other words: a model or process made so easy to understand that a human could do it. \nisarcomm{need to think carefully about implications of this definition: in principle any human could execute any algorithm given enough time/computation, so this isn't saying much...something else needed here to refine this idea...}. 
\cite{Doshi-Velez2017-xy} use the words `interpretable' and `explainable' interchangeably. In contrast, we treat them as distinct descriptors. We discuss models that are inherently interpretable here, and models that can be understood by explanation in Section~\ref{sec:reduce_complexity}.  The difference is that interpretability (in our view) implies that the actual process/model used by an AIA is self-explanatory, whereas explainable models can be made interpretable by post hoc operations but do not necessarily explain the actual model/process used by an AIA. 
Being able to interpret the actual model/process used by an AIA helps human users to more appropriately understand their behaviors, and thus exhibit appropriate TRBs in turn. This approach to assurance also captures broader AIA processes and models that rely on rules, heuristics, etc., rather than just those that rely on optimization of some particular utility. 
%%\nisarcomm{need to unpack this still further -- see tex comments above }

\subsubsection{Common Approaches:}
Two main approaches to designing interpretable AIA models and processes are considered here. 
The first is to \emph{assess an existing set of candidate models/processes} in order to evaluate their interpretability for a particular problem, and then select the best candidate. 
This is typically done with certain classes of models or solution processes, e.g. whether to use decision trees vs. decision tables for a given planning task. 
The second is to \emph{synthesize interpretable models/processes} by leveraging human designer input during the model/solution-building process. 
The first approach requires pre-defined measures of interpretability, and thus some mechanism for capturing ability to gain insights into competence, predictability, and situation normality. This also presupposes that the model/process candidates are inherently interpretable along these lines to begin with, which may rule out certain model/process families that perform well on certain tasks. 
The second approach allows designers to apply domain knowledge to determine metrics for interpretability, although this can lead to solutions that do not perform as well as those that are less interpretable. 
%%\nisarcomm{where and hwo do (intrinsic) assurances fall out of these?}

What is the assurance mechanism that potentially leads to proper TRBs in either approach? 
Essentially, allowing the user to access and examine an interpretable model/process also allows them to simultaneously assess competency, predictability, and situational normality components of trust. If the models/proceses are interpretable, then a user would understand exactly how the AIA would perform its task (i.e. down to a mechanical/programmatic level). 
This gives the user a `mental model' of what the AIA would consider to be situational normality and how AIA would respond in different situations (predictability and competence). 
The caveat here is that incorrect TRBs may arise if the user mis-interprets or only understands part of the model/process. 
This is a significant risk in highly complex or specialized problems, where users may not actually have sufficient training or expertise. This also poses concerns for how/when user can access interpretable AIA components. 
Unlike value alignment (where user accesses assurances only through behavior of AIA itself), the user has more freedom in deciding when and how to `peek under the hood'. This relates to assurances based on information visualization discussed later, except that here the information being given to the user are the actual AIA algorithms themselves, as opposed to by products or after effects of those algorithms. \nisarcomm{I need to trim this down a bit?}

\paragraph{Assessing Interpretability:}
\citet{Van_Belle2013-ph} suggested three ways to ascertain the level of interpretability and potential utility of learned models (compare to categories proposed by \citet{Lipton2016-ug}): 1) Map them to domain knowledge; 2) Ensure safe operation across the full operational range of model inputs; and 3) Assess whether important non-linear effects are accurately accounted for. This work identifies certain strengths and weaknesses of different techniques, but ultimately concludes that no method is clearly best in all situations. 
%%%\nisarcomm{need to say a little bit more about the actual methods proposed -- how does one, for instance, go about doing 1 or 2 or 3? don't need to give specific details, but mention what techniques available}
%
Along similar lines, \citet{Huysmans2011-th} compared decision trees, decision tables, propositional if-then rules, and oblique rule sets to understand which set of methods is `most interpretable'. It was experimentally determined that decision trees and tables tend to be easier to interpret, but it is noted that each method could perform better than others in different applications. For example decision trees and tables are typically better suited for answering a symbolic question (which requires a local understanding of a model) like: \emph{how does the model classify observation $X$'?}. This is in contrast to a spatial question (which requires a global understanding of the model) like: \emph{is it correct that applicants with a high income are more likely to be accepted than applicants with a low income?}. 
%
Having quantified the interpretability of a model given different classes of problems, and different requirements of users the appropriate model can then be selected during design to fit the needs of a specific application.

\paragraph{Interpretable Model Synthesis:}
\citet{Ruping2006-xj} asks how classification results and the accuracy-interpretability trade-off can be made more transparent to those who design and use classifiers. They explore one approach by combining simpler global models with more complex local models that are built around learning results (\citet{Otte2013-oo} and \citet{Ribeiro2016-uc} implement similar ideas as well). Figure \ref{fig:ruping} illustrates this idea.

%%%%%%%%%%%%%%%%%%%%%%%%%
\begin{figure}[htbp]
    \centering
    \includegraphics[width=0.5\textwidth]{Figures/global_local}
    \caption{Example of simple global interpretable learning model on the left, and on the right a more complex locally interpretable learning model that can be used when more precise understanding of a specific decision made by the learner is required. }
    \label{fig:ruping}
\end{figure}
%%%%%%%%%%%%%%%%%%%%%%%%%

Considerable effort has also gone into endowing `grey box' and `black box' models with interpretable features. 
For instance, \citet{Abdollahi2016-vn} investigate making collaborative filtering models more interpretable by using a conditional restricted Boltzmann machine (RBM). \citet{Ridgeway1998-lv} use `weight of evidence' (WoE) as a boosting method that is more amenable to interpretation, and show that WoE is on par with AdaBoost. \citet{Choi2016-by} construct a recursive attention neural network to remove recurrence on the hidden state vector, and instead add recurrence on the visits of patients to doctors, as well as on different diagnoses during those visits. In this way the model is able to predict possible diagnoses in time, and a visualization can be that that indicates the critical visits and diagnoses that lead to that prediction.

Learning of human-understandable representations for data and feature selection also provides another avenue for developing assurances  \cite{Bengio2013-uv, Guyon2003-fj}. For instance, \citet{Mikolov2013-lt} studied how to represent words and phrases in a vector space for natural language text learning; this enables simple vector operations for understanding word sense similarity and relative relationships learned from text corpora. For example, the vector addition operation $airlines+German$ yields similar entries that include $Lufthansa$. Such representations encodes knowledge that can be easily checked and understood by humans, and thus implicitly facilitate interaction and calibration of trust (see \cite{Haury2011-zi} for another example). The problem of discovering human understandable features and representations in more general settings still remains an open question. Currently, the main question for representation learning is how to find the `best representations' for a particular application -- not necessarily the representations and features that are `most humanly understandable'. This is not surprising, since human-understandable representations and features are not necessarily optimal for the criteria that AIAs are typically designed against. 

Contrary to the belief that interpretable models are necessarily worse performing than their less interpretable counterparts, several researchers have shown that this is not always the case (at least in the context of machine learning). However, the real trade-off is the amount of work that goes in to crafting the interpretable model from the start; these methods are often custom designed for certain tasks and are not easily transferable to other problems. Because of this, AIA designers must strike a balance between \emph{interpretable models}, \emph{explainable models}, and \emph{black-box models}.

\citet{Park2016-ld} point out that real interpretability in complex tasks still requires expert knowledge to make sense of complicated features; in essence: \emph{interpretable models require people}. For instance, \citet{Jovanovic2016-gw} use `Tree-Lasso' (TL) logistic regression with domain knowledge (i.e. medical diagnostic codes) to group similar conditions, and then use TL regression again on that information to develop a sparser model. \citet{Zycinski2012-jj} also use domain knowledge to structure a data matrix before feature selection and classification. See also \citet{Zhang2018-no,Khoa2018-gh} for other related examples. 
%
This kind of approach is also illustrated by those who use those who use `theory guided data science' (TGDS~\cite{Kumar2016-yw,Faghmous2014-og}). As one example \citet{Morrison2016-fz} address the situation where an imperfect analytical model is available for chemical reaction kinetics: the theoretical reaction equations are well known, but a `stochastic operator' is added on top of this to account for uncertainties and modeling errors. In adopting this approach the model becomes interpretable (to experts).

\subsubsection{Grounding Example:}
In the case of the `VIP Escort' problem (described in Section~\ref{sec:mot_example}), interpretable models might be used as an assurance in the following way:

We make the following assumptions

\begin{itemize}
    \item The UGV has just begun an attempt to escape the road-network
    \item The UGV is using a decision-tree for selecting different movements
    \item The operator is able to view the decision-tree model the UGV is using
\end{itemize}

While the operator is monitoring the progress of the UGV in its attempt to escape the road-network they are able to consult the decision-tree model. In this case the operator chose to consult the table when they saw the UGV make an unexpected turn at a given intersection. The operator identified the conditions that led to the decision and found that the UGV was not well equipped to execute the decision the operator thought was best.
\paragraph{\textbf{Discussion of Example:}} In this example the use of a decision-tree as a model enabled the operator to investigate unexpected behavior. During inspection they identified certain conditions that led to a decision, and they found that the UGV was not \emph{competent} to perform what the operator thought was a better decision. Because of this the operator better understood the decision the UGV made.

%%(i) assurance argument: what specifically is the assurance signal? 
%%(ii) what is mechanism for generating appropriate TRBs? 
%%(iii) how can designers build/exploit for AIA assurances, i.e. what techniques available for ...?: 
%%(a) ...  
%%(b) ...
%%(c) ...

\subsection{Information Visualization} \label{sec:vis_dr}
We define `information visualization' as the act of displaying information in such a way as to communicate to one of the trust dimensions of a human user. Specifically we consider the `competence', and `predictability' of the AIA and the `situational normality' of the task at hand.

\subsubsection{Common Approaches:}
\citet{Liu2017-xw} review several of the current methods that exist for visualizing ML models. They identified three main purposes for which visualizations are useful in this context: 1) understanding (why model behave how they do), 2) diagnosis (failures, or unexpected behavior), and 3) refinement (ability to improve performance). We focus on \emph{techniques} that assist in that process.

Two of the main tools in creating visualizations are and reducing the dimensionality, and treating uncertainty in creating the visualizations to assist users in understanding more easily.

\paragraph{Dimensionality Reduction:}
Dimensionality reduction (DR) is one of the key methods used in creating visualizations. \cite{Sacha2017-hf} identify seven different methods by which users interact with DR techniques. They use this to make the human-in-the-loop process model for interactive DR that is shown in Figure~\ref{fig:sacha_fig}.

\begin{figure}[htpb]
    \centering
    \includegraphics[width=0.9\linewidth]{Figures/dimred_framework.png}
    \caption{Human-in-the-loop process model proposed by \cite{Sacha2017-hf}. Included by permission.}
    \label{fig:sacha_fig}
\end{figure}

\citet{Venna2007-yj} discusses DR for ML and reviews many linear and non-linear projection methods. \citet{Vellido2012-nm} also discusses the importance of DR for making ML models interpretable. As one example, \citet{Chipman2005-om} applied this idea by constraining principle component analysis (PCA) in an attempt to make the resulting linear combinations of variables be more interpretable (more homogeneous, or more sparse).

At times a simple visualization is the most efficient way to communicate the results of decision making, planning. For example: \citet{Chadalavada2015-wx} enable a robot to project its path onto the ground so users can see.

\paragraph{Treatment of Uncertainty:}
In the previous section we have already visited the importance of an AIA being able to quantify its uncertainty. Visualization researchers are concerned with how to \emph{convey} that uncertainty to human users (and quantify uncertainty inherent in making visualizations). \cite{Sacha2016-tu} discuss how the propagation of uncertainty through visual analytics systems can affect the trust of human users (see also \cite{Correa2009-hi}).

One excellent example of this is the work by \citet{Wu2012-qi}, who create a tool to visualize the flow of uncertainty in the visualization process. In this way users can understand where uncertainty enters the visualization process.

\begin{figure}[htpb]
    \centering
    \includegraphics[width=0.6\linewidth]{example-image-golden}
    \caption{Example visualization of the flow of uncertainty in the creation of a visualization \cite{Wu2012-qi}. Included by permission.}
    \label{fig:hutchins_fig}
\end{figure}

The relationship between system uncertainty and the effects of uncertainty on the performance of the system can be very complex to understand. \citet{Hutchins2015-if} address this by using expert knowledge, and a `trust annunciator panel' (TAP) that has several `uncertainty level indicators' in order to display how uncertainties in sensors will effect the output quality, and the mission impact; and the same for the planning algorithm (see Figure~\ref{fig:hutchins_fig}).

\begin{figure}[htpb]
    \centering
    \includegraphics[width=0.9\linewidth]{Figures/Hutchins_fig.pdf}
    \caption{Proposed `trust annunciator panel' \cite{Hutchins2015-if}. Included by permission.}
    \label{fig:hutchins_fig}
\end{figure}

\subsubsection{Grounding Example:}
In the case of the `VIP Escort' problem (described in Section~\ref{sec:mot_example}), information visualization might be used as an assurance in the following way:

We make the following assumptions

\begin{itemize}
    \item The UGV has just begun an attempt to escape the road-network
    \item The user has access to an interface like that proposed in \cite{Hutchins2015-if}
\end{itemize}

During the attempt the user is able to see how the sensor uncertainty might possibly effect the outcome of the mission. In this case, the user is assured that the sensors will have little negative impact on the outcome of the mission given the current weather conditions.

\paragraph{\textbf{Discussion of Example:}} Here we see how a visualization is able to assist the user in correlating the effects between sensor uncertainty and mission outcome. This is not a simple relationship for operators (especially untrained) to learn on their own; even if they were able to learn the time required to do so can be very detrimental.

%%(i) assurance argument: what specifically is the assurance signal/rationale? -- whereas user interaction techniques of previous section generally tend to provide integral assurances (i.e. designed as part of core functionality of AIA capabilities) that introspectively compensate for shortcomings in AIA capabilities, similar introspective assurances can also be generated to determine and inform users of competency limits/boundaries of AIA capabilities *without requiring* user interaction or intervention -- and thus can be considered as supplementary to or separate from core AIA functionality. That is:  assurances can be provided to introspectively determine and explain/interpret limits of what AIA can do or knows, etc. for the sake of the user, without requiring modification of underlying AIA design.  In this sense, self-assessments can provide users with insights on one or both of the following related concerns: what can the AIA actually do and what does it know? and, what is required of/by the AIA to actually do the assigned task?  The first concerns identifying set of tasks in `reach set' of AIA; the second concerns figuring out what would be needed to do current task [need to refine this...trying to distinguish between questions that lead to insights about competency and situation normality via complexity reduction, vs. insights that inform predictability via uncertainty]... but basic analogy [can be mapped to UGV] is subordinate/delegate telling supervisor what it is/is not capable of, vs telling supervisor what it would need to be able to carry out specific task at hand or what the possible outcomes would be for that specific task (so, self-assessment can have contrasting focus on AIA itself on general capabilities vs. on the task at hand in relation to how AIA would perform on it specifically)...
%%(ii) what is mechanism for generating appropriate TRBs? -- Key targets are competency and situational normality dimensions in terms of explaining AIA functionality, whereas predictability is key target for task oriented assurances. Assurances from self-assessment paradigms are typified by frameworks like `machine self-confidence'[] and explainable Bayesian inference[] -- these operate on the results of `black box' AIA component outcomes in a post hoc manner, unlike interpretable models discussed earlier (which force AIA components to be inherently `understandable' to users). Viewed differently: whereas interpretable models are more `bottom up', self-assessment is more of a `top down'/drill down process: latter is less constrained in choice of models/techniques for AIA capabilities, but also must rely on ability to suitably decompose these same AIA functions across different task contexts and 
%%(iii) how can designers build/exploit for AIA assurances, i.e. what techniques available for getting assurances from self-assessment?: 
%%(a) complexity reduction...  
%%(b) uncertainty...


\subsection{AIA Self-Assessment} \label{sec:aia_self_assessment}

The techniques of previous section generally tend to provide integral assurances (i.e. designed as part of core functionality of AIA capabilities) that are artifacts of interactive algorithms designed to compensate for shortcomings in AIA capabilities. This section focuses on introspective assurances that inform users of competency limits and boundaries of AIA capabilities without requiring user interaction, and that can generally be separated from core AIA functionality (i.e. without requiring modification of underlying AIA design).  These self-assessments can provide users with insights regarding either or both of the following related issues: (i) what information and tasks are actually within the AIA's reach?, and (ii) what is required by the AIA to actually do its assigned task? 
In contrast to user interaction techniques: the analogy here is of a subordinate telling a supervisor what she is/is not capable of, or telling the supervisor what she would need to be able to carry out specific task at hand to achieve a specific outcome, or what the possible outcomes actually would be for that specific task. 

%%Such assurances provide windows into the competency and situation normality via complexity reduction, vs. insights that inform predictability via uncertainty]... but basic analogy [can be mapped to UGV] is subordinate/delegate telling supervisor what it is/is not capable of, vs telling supervisor what it would need to be able to carry out specific task at hand or what the possible outcomes would be for that specific task

\subsubsection{Common Approaches:}
%\nisarcomm{Need to say a bit more about what the motivation/general idea here is, to continue flow from other previous sections...}
The literature in this section can be split into two high-level categories. 
The first set deals with how an AIA can algorithmically account for its uncertainties in its models of its task, environment, operating context, and capabilities. 
These kinds of assurances help inform the predictability and situation normality aspects of trust. 
The second set of methods attempt to algorithmically reduce complex `uninterpretable' models or processes that underlie AIA capabilities into more interpretable ones by providing explanations. 
Here the AIA makes an active attempt at processing data and making information available to the user to inform the competency aspect of trust. %%This is done in a post-hoc manner, or in a way such that the quantification of uncertainty is more supplemental, rather than integral, to the main functions of the AIA.

\paragraph{Quantify Uncertainty} \label{sec:QU}

Although active learning does not explicitly consider safety, the underlying approaches can be useful because active learners need to be able to search the problem space to reduce uncertainty; this requires an internal representation of uncertainty. The applications surveyed here are all mainly related to image classification and robotics. In the context of image classification, \citet{Paul2011-vr} introduced `perplexity' as a metric that represents uncertainty in predicting a single class and is used to select the `most perplexing' images for further learning. There have also been several attempts to use Gaussian processes (GPs) to actively learn and assign probabilistic classifications \cite{MacKay1992-sp,Triebel2016-kj,Triebel2013-ow,Triebel2013-ku,Grimmett2013-gj,Grimmett2016-yc,Berczi2015-rd,Dequaire2016-kh}. As with perplexity-based classifiers, the key insight is that if a classifier possesses a measure of uncertainty, then that uncertainty can be used for efficient instance searching, comparison, and learning, as well as reporting a measure of confidence to users. The key property of GPs to this end is their ability to produce output confidence/uncertainty estimates that grow more uncertain away from the training data. This information can be readily assessed and conveyed to users, even in high-dimensional problems. This property has also found much use in other AIA active learning problems, e.g. Bayesian optimization \cite{Snoek2012-tt, Brochu2010-tj,Israelsen2017-zb}. 

\citet{Choi2017-th} investigates how mixture density networks (MDNs)---neural networks that learn parameters of a Gaussian mixture distributions---can be used to help a controller switch modes based on the MDN's prediction of 

Bayesian neural networks (BNNs) are a method by which we can have insight into the uncertainty of a neural network model. Using BNNs \citet{Kendall2017-ry}, in the context of computer vision, also use deep BNNs to help visualize epistemic (input) and aleatoric (model) uncertainty for each pixel of an image. 

Similarly \citet{Kahn2017-vy} use deep BNNs to learn about the probability (with uncertainty) of an autonomous vehicle colliding in an environment given its current state, observations, and sequence of controls. Using this model they formulate a `velocity-dependent collision cost' that is used for model-based reinforcement learning. With this approach the vehicle naturally proceeds slowly when there is an elevated risk of collision. \brettcom{not sure if this goes here, or in the `value alignment' section\ldots it goes here if i downplay the built-in nature of the behavior, and instead focus on the ability to quantify uncertainty}

An AIA that can predict its performance on different tasks can provide assurances about competence, predictability, and the situational normality of a given task. Several authors have worked to improve this ability in visual classification \cite{Zhang2014-he,Gurau2016-hs,Churchill2015-ei,Kaipa2015-hy}. 
For example, to ensure that visual classifiers don't fail silently in novel scenarios, 
\citet{Zhang2014-he} learned models of errors on training images to predict errors on test images. 
\citet{Kaipa2015-hy} consider 3D visual classification of assembly line parts for robotic pick and place tasks, and develop statistical goodness-of-fit tests to estimate the likelihood that robots can use their sensors to find parts matching desired ones. %To accomplish this they apply the `Iterative Closest Point' (ICP) method, to match a point cloud measurement of the part with a ground-truth 3D model of the part. 
These approaches allow the AIA to assess capability and present appropriate assurances to users, though without any formal notions of trust. 

Models and logic are not trustworthy by themselves; they may be flawed to begin with, or become invalid when certain assumptions or specifications are violated. Thus, there is great interest in providing assurances that the models and assumptions underlying different AIA processes are in fact sound. \citet{Laskey1991-mf} -- with the intention of communicating model validity to users of `probability-based decision aids' -- notes that it is infeasible to perform a decision-theoretic calculation to determine if model revision is necessary. 
She presents a class of theoretically justified model revision indicators which are based on the idea of constructing a computationally simple alternate model and then initiating model revision if the likelihood ratio of alternate model becomes too large (see also \citet{Zagorecki2015-qy,Habbema1976-xd} --these ideas also provide a potential basis for the `model validity' machine self-confidence factors from Quadrant II).
\citet{Ghosh2016-dl}  present `model repair' and `data repair' strategies that can be used when the current model doesn't match the observed data, at which point the model and data can be repaired, and control actions can be replanned in order to conform with the formal method specifications. One challenge is how the `trustable' constraints should be identified, as this places a strong burden on the certifying authorities and system designer to foresee all possible failures.


\paragraph{Reduce Complexity} \label{sec:reduce_complexity}
Main ideas: 1) make explanation, 2) operate on black-box model, 3) styles of explanation, 4) 

\citet{Olah2018-rp} talk about `intepretability' but are really thinking about explanation, or making something interpretable.

\citet{Abdollahi2018-uw} talk about different kinds of explanations that can be made. `neighbor style', `influence style', and `keyword style'

\citet{Huang2017-lk} using `algorithmic teaching'~\cite{Balbach2009-jw} as inspiration. Algorithmic teaching involves having a model of a students learning algorithm, and then presenting training examples to allow the student to learn a target model. In this case the student is the human user, and the teacher is the robot that is trying to teach the human its own objective function by presenting a set of (optimal) training examples. We consider these training examples to be assurances herein.

\citet{Hayes2017-nt} put an abstraction with `communicable predicates' over the state space, and are able to explain controller policies in natural language.

In some cases it is desirable to maintain a complex, less interpretable model and then apply `first principles' to explain the results to the user. \citet{Lacave2002-cu} address this from the perspective of explaining probabilistic inference in Bayesian networks -- specifically, \emph{how} and \emph{why} a Bayesian network reaches a conclusion given some imputed evidence. 
They present three properties of explanation: 1) content (what to explain), 2) communication (how to explain), and 3) adaptation (how to adapt based on who the user is). %%It is not possible to cover all of the ideas that they present in their paper, but they are key to the idea of designing assurances. 
Several key points for designing assurances arise from considering the differences between explaining evidence (i.e. data), the model (i.e. the Bayesian network itself), or the reasoning (i.e. the inference process). 
%These are three key considerations in making assurances. 
Also important is whether an explanation is meant to be descriptive or aimed at ensuring comprehension, as well as whether explanations need to be on a macro or micro scale relative for parts of the Bayesian network (similar to globally/locally interpretable learned models \cite{Ruping2006-xj}). 
The authors also consider whether explanations should occur by two-way interaction between system and user, by natural language interaction, or by probabilities. Finally, considering adaptation, another key point for designing assurances in general applications and contexts is that not all users will require (or desire) the same kinds of assurances. This paper points out many challenges and considerations in designing assurances for probabilistic algorithms, and illustrates that %(as with the `No free lunch' theorem) 
there is no single `best' assurance that will address every possible situation. 
Other discussion regarding how probabilistic and statistical explanations can be presented is found in \cite{Rouse1986-dz,Wallace2001-fm,Kuhn1997-qc,Lomas2012-ie,Swartout1983-ko}; 
aforementioned works like \cite{Kuhn1997-qc} highlight the importance of framing effects and other cognitive biases for these methods. 



\subsubsection{Grounding Example:}
In the case of the `VIP Escort' problem (described in Section~\ref{sec:mot_example}), self-assessment might be used as an assurance in the following way, starting with the assumptions that:

\begin{itemize}
    \item The UGV is about to being an attempt to escape the road-network
    \item The UGV is using the `solver quality' metric mentioned by \citet{Aitken2016-fb}
    \item The operator has access to an interface where they can view the self-confidence metric calculated by the UGV
\end{itemize}

Before the UGV begins its attempt it is able to assess its `solver quality' given the specific, unseen road-network, based on similarities between the current network and ones that it has encountered before (i.e. problem features that are important to determining the quality of the approximate solution produced by the policy). The UGV reports that it has high confidence in its solver quality, and the operator is assured that they can trust the solver in this situation.

\paragraph{\textbf{Discussion of Example:}} In this case the UGV is able to assure the operator of the quality of the solver in the specific road-network. Generally, the UGV reduced what could be a very complex analysis into a simple format for the operator to interpret. This is in contrast to the operator viewing policies, models, algorithms, and complex probability distributions.

\subsection{User Assessment} \label{sec:user_assessment}
talk about these: \citet{Yu2018-qw, Wickens1999-la, Riley1996-qm, Muir1996-gt, Desai2012-rc}
\brettcomm{this is communicating data (unprocessed), rely on human to do processing (as opposed to vis and DR). human limitations on cognition.}
\subsubsection{Common Approaches:}

\paragraph{stuff:}


\subsubsection{Grounding Example:}
In the case of the `VIP Escort' problem (described in Section~\ref{sec:mot_example}), value alignment might be used as an assurance in the following way:

We make the following assumptions

\begin{itemize}
    \item The UGV has just begun an attempt to escape the road-network
    \item 
    \item 
\end{itemize}

\paragraph{\textbf{Discussion of Example:}} 

% \subsection{Calculating, Designing, and Planning Explicit Assurances}
    Recall that an assurance is defined as \emph{any} behavior or property of an AIA that affects a user's trust; an explicit assurance is any assurance that was consciously implemented/applied by the designer with the express intention of influencing a user's TRBs/trust (whether or not the means for doing so conforms to a formal trust concept). As such, it is possible to design assuring properties into the system a priori. It is likewise possible to design assuring behaviors into an AIA. From the literature, several key high-level issues emerge pertaining to assurance implementation: (1) uncertainty quantification; (2) complexity reduction; (3) assurance by design; and (4) planning strategies for assurances. 

    \brettcomm{DISCUSS IMPLICIT VS EXPLICIT ASSURANCES HERE---PROBABLY SOMETHING LIKE A PARAGRAPH OR SO}

    \begin{itemize}
        \item Quantifying Uncertainty
        \item Reducing Complexity
        \item Assurance by Design
        \item Planning Strategies of Assurance
    \end{itemize}

\subsubsection{Quantifying Uncertainty} %%nra: awkward sentence: Being able to quantify the different kinds of uncertainty in the AIA is necessary before attempting to express that uncertainty to a human user. 
    An AIA must have some way to assess and quantify uncertainty before attempting to express those uncertainties to a human user. 
    There are several different kinds of uncertainty that might be considered, e.g. uncertainty in sensing, uncertainty in planning, and uncertainty in locomotion. 
    A model or method needs to be incorporated in the AIA that will represent the different kinds of uncertainty to the human user in some way. 
    A human user could use such information to inform their trust in the `situational normality', `competence', and/or `predictability' of the AIA. 
    One might imagine that, in the UGV road-network problem, the UGV expressing high uncertainty in its plan would influence the competence component of the user's trust. Conversely, if an uncertainty measure is not available, the user might take this as an implicit assurance that the AIA is perfectly confident, or (based on the user's experience)  might conclude that since all AIA plans have been flawed in the past, the plan of this AIA must be flawed as well. 

    The surveyed literature  featured several different uncertainty quantification approaches. In some cases uncertainty is already represented intrinsically by the algorithms and/or models being used in the AIA. Some use the built-in statistical representations of state transitions and sensor observations, e.g. for POMDPs, as the basis of quantifying uncertainty. This approach has straightforward analogs in other systems that use algorithms and/or models that inherently consider uncertainty. Models and methods that intrinsically represent uncertainty are frequently available. However, even when that it is the case, there are types of uncertainties that may still not be considered. For example, there are additional aspects of uncertainty beyond those intrinsically captured by some optimal planning approaches \cite{Aitken2016-cv,Kuter2015-qh}. 
    Using the UGV road network problem as an example, what is the total probability distribution over all possible total cumulative reward outcomes (i.e. beyond just the expected value maximized by the optimal policy)? 
    Or, given a certain road network, how closely can a Monte Carlo-based POMDP solver for the UGV approximate the true optimal policy? 
    These describe `uncertainties in application', which arise in applying certain algorithms and models in different contexts.

    Independent of the algorithms or representations used by an AIA, uncertainty quantification for assurance design also often focuses on assessing the probability of error for decision making as it pertains to any of the AIA capabilities from Fig. \ref{fig:AIcapabilities}. 
    %(regression methods have analogous approaches). 
    For instance, the problem of assessing the probability of error for classification and regression tasks for learning and perception are well-studied. 
    In addition to standard training/validation data-based assessment techniques, some have approached this problem by learning models of errors for underlying learning models, e.g. by training a GP error `meta-model' for classifier errors  from empirical training data and using the meta-model to quantify uncertainty in different test scenarios \cite{Gurau2016-hs}. 
    In contrast, we might quantify the uncertainty of a classifier based solely on the input data itself \cite{Zhang2014-he}. Of course, these methods share common drawbacks of being solely supervised learning approaches. However even in domains such as reinforcement learning, methods for avoiding highly uncertain (un-safe) rewards must use some form of external expert knowledge (see \cite{Garcia2015-rs, Lipton2016-dq}). Hence, the problem of defining a sensible `ground truth' against which uncertainties can be `correctly' assessed remains a fundamental challenge. %for developing AIA assurances.  
    %
    %\edit{...merge this paragraph with end of previous somehow...}
    %    Uncertainty can be easier to assess if some kind of oracle, or reference is available for comparison. 

    Some works have proposed quantifying the similarity between empirical experience and available reference data from a truth model as a measure of uncertainty \cite{Kaipa2015-hy}. Of course, this approach loses its appeal when a `truth model' isn't available. This shouldn't detract from the intent of finding some kind of reference (truth or otherwise) in which the reasoning, sensing, and other processes of an AIA can be compared to evaluate uncertainty. The evaluation of statistical models involves a very similar concept: given a statistical model as a reference does current empirical experience support or detract from the hypothesis that the model is still valid \cite{Laskey1995-jp,Ghosh2016-dl}? These approaches can quantify the degree to which the statistical models are still true, and this measurement can be used as an indication of uncertainty. However, these techniques can only provide binary indications that a model is statistically valid/invalid, without providing specific human-understandable causes or justifications as to why. Human expertise and designer judgment is thus required to carefully interpret such assurances, although a universal set of `best practices' for reporting such assurances and coping with cognitive biases in probabilistic/statistical data interpretation (framing effects, etc.) have yet to be established. 
    %
    %%This paragraph repeats the first one, so merging in the last part with that and commenting out the rest:
    %%Generally, uncertainty quantification allows AIAs to express assurances related to the `situational normality', `competence', and `predictability' of the system in a given situation. One might imagine that, in the UGV road-network problem, the UGV expressing high uncertainty in its plan would influence the competence component of the user's trust. Conversely, if an uncertainty measure is not available, the user might take this as an implicit assurance that the AIA is perfectly confident, or (based on the user's experience)  might conclude that since all AIA plans have been flawed in the past, the plan of this AIA must be flawed as well.
    %%
        \subsubsection{Reducing Complexity} Many researchers have attempted to remove complexity from the models and logic of the AIA to make the methods more interpretable (or comprehensible, or explainable, \ldots) to human users. 
    As with uncertainty quantification, interpretability can also inform a user's trust regarding the AIA's `situational normality', `competence', and `predictability'. Of course, this presupposes that many of the methods used by AIAs are `complex' by some measure; we claim that the fact that highly trained and educated experts are required to understand and develop these methods %(and even then it may not be totally possible) 
    proves this supposition. Complexity only exists in the presence of some reference frame, i.e. the designer's or the user's in this case. 
    From the perspective of an assurance designer, the key challenges to making/discovering/learning assurances with scalable interpretability lie in defining and satisfying criteria for required depth of understanding on the part of the user, the level of user expertise, and the time required by the user to gain understanding. 

    Generally, the `complexity' of an algorithm or more general process used by an AIA is said to increase with the number of variables, steps of reasoning, the size of data, etc. in the formal sense that is understood by computer scientists. 
    In practice, \textit{reduction} of complexity has been addressed by approaches as simple as finding summary statistics, e.g. calculating averages \cite{Muir1994-ow,Muir1996-gt}, that help explain why/how an algorithm or AIA arrives at a particular result. 
    This can also be accomplished by computing heuristic measures which reduce many complex algorithmic components into more manageable pieces\cite{Aitken2016-cv}. Creating variable fidelity models is another way by which complexity can be reduced (and increased when necessary). One might also elect to create and use models to support AIA reasoning, decision, making, etc. capabilities which are inherently more interpretable to humans. This could include constraining the feature space to be more simple, reducing dimensionality, learning more understandable features, and using physical theories to ground learned models (e.g. for interpretable science).  %%\edit{...merge? feels a bit out of place...}
        
    It is possible for `inherently interpretable models and algorithms' to compete with other non-interpretable models in terms of performance. 
    However, this may not be the most viable long-term approach for tackling the problem of complexity reduction in assurance design, especially due to the amount of work required to design interpretable counterparts to the latest state of the art algorithms for implementing AIA capabilities. 
    On the other hand, approaches that seek `post hoc' explanations of complex models and algorithms offer more promise. 
    This is partly because the idea of `interpretability' is not universally well-defined, and thus does not offer a single tangible goal. Rather, there exists a continuum of interpretability based on the complexity of the problem, the time required for a user to interpret (e.g. a few seconds vs. a few months of study, depending on the application), the expertise of the user, and other factors. However, assurances that can be automatically generated to provide user-specific and model/method-agnostic explanations/interpretations of complex AIA processes would provide the best of both worlds. These ideas are much more aligned with the efforts of \cite{Ruping2006-xj} and others who seek models with scalable resolution and accuracy.

\subsubsection{Assurance by Design}
    No matter how much engineers and system designers like to think about automating everything, humans will realistically need to be involved at some level of AIA operation as well as assurance design for the foreseeable future, since the human-AIA trust cycle cannot be properly managed without taking human inputs into consideration. 
    The problems of uncertainty quantification and complexity reduction can largely leverage existing methods to generate explicit assurances `on the fly' as needed for human users. 
    We have also seen that several strategies are available to directly engineer AIAs to be as meaningful and transparent to humans as possible from the outset through direct `consequential' forms of interaction. 

    One approach is to make the human user and human interpretability an intrinsic part of an AIA's decision making and learning processes, in order to modify their associated objective functions according to human user preferences \cite{Freitas2006-qo,Dragan2013-wd}. 
    Poorly designed objective functions are an important source of discord between how human users expect AIAs to behave and how AIAs actually behave (thus making them less predictable and less trustworthy) \cite{Amodei2016-xi}. 
    `Myopic objectives' are said to be present when an AIA focuses on a specific objective to the extent that a human can no longer relate to the objectives of the AIA (so that the user will be correspondingly surprised by its actions). This suggests that significant time may be required to design objectives that align with those of humans, in order to make the AIA more predictable, and competent in the user's eyes. 
    %
    `Human-in-the-loop' (HITL) methods can be used to implicitly encode many human qualities that cannot be expressly quantified or explained. 
    It is interesting to note that using HITL can offer more interpretability to the result of a learning process, while at the same time making the learning process itself less complex and procedural. 
    However, there are trade-offs that can be undesirable in many situations as well, as decision making and learning algorithms are often used to circumvent human biases and imperfect human reasoning. 

    Aside from relying on HITL, it is also possible to modify standard decision making and learning approaches (e.g. as discussed in the previous section) to make the methods inherently more transparent and assuring to users. For example, one might restructure a neural net architecture or a decision tree to make it more interpretable \cite{Choi2016-by,Abdollahi2016-vn,Jovanovic2016-gw}. However, there are no universally established principles yet for how this can/should be done.

\subsubsection{Planning Explicit Assurances}
    Computational resources are needed for an AIA to formulate and communicate assurances, as well as assess whether such assurances are eliciting the desired TRBs from users. This naturally raises the question of whether/how AIAs can formulate plans to achieve the goal of effectively and appropriately expressing assurances to a human user. 

    As mentioned earlier, not all AIAs are capable of planning. 
    Such AIAs can be designed beforehand with some kind of static plan or policy to deliver assurances (e.g. using a fixed set of rules). 
    Otherwise, more advanced AIAs might use TRBs as feedback signals to come up with assurance strategies designed to steer users into enacting appropriate TRBs. 

    When planning assurances, the AIA must be able to account for limitations of users, and its own limitations in expressing assurances. For example a user may not be able to understand information needed to use the AIA more appropriately. Also, the AIA may need to take a longer-term strategy to teach or tutor the user, as opposed to only presenting the user with canned, static, assurances. Some important user considerations are discussed in Section~\ref{sec:express_assurances}.
    
    One must ask whether the human user can correctly process the information received. This is perhaps most easily illustrated by considering a non-expert user who cannot understand highly technical assurances regarding the AIA. However, less trivial manifestations, such as the existence of bias in the perception of assurances, or the inability of users to recognize or process important information due to cognitive overload, are also possible. This will also be addressed in Section~\ref{sec:express_assurances}.

    The problem of planning for assurances is nearly unexplored in the context of human-AIA trust relationships. However, there are several fairly new research programs that are interested in this question (i.e. DARPA's Explainable Artificial Intelligence (XAI) program \cite{Gunning2016-kb}) and related ones (e.g. DARPA's Assured Autonomy program \cite{Neema2017-bb}). Assuming an AIA can provide assurances, important questions arise, such as: what is the best way to present them? How can they be adapted for different kinds of users? How can the AIA teach or tutor the human to use the AIA appropriately over time and in different operational contexts? The problem of enabling AIAs to be aware that they must take action and formulate plans to deliver assurances is a large gap in the current assurances landscape; answers to these questions are critical to designing more robust and effective assurances.





%%%%%%%%%%%%%%%%%%%%%%%%%%%%%%%%%%%%%%%%%%
\section{Future Work} \label{sec:future_work}
% The body of research that is encapsulated by assurances is very large. As such, this document can barely scratch the surface. Having said that the concepts that have been discussed lead to many interesting lines of research that can be pursued in future work. Here we highlight some of those.
%
% \begin{description}
    % \item [Artificial Intelligence/Machine Learning:] There is a need to interpret how and why theoretical AIA models function, in order to know they are being applied correctly and to design new approaches to overcome weaknesses in the existing methods~\cite{Garcia2015-rs,Otte2013-oo}.
    % \item [Interpretable Science:] Scientists need to be able to trust the models created using data analysis, and be able to draw insights from them. Scientific discoveries cannot depend on methods that are not understood~\cite{Faghmous2014-og}.
    % \item [Reliable Machine Learning:] It is critical to have safety guarantees for AIAs that have been deployed in the real world. Failing to do so can result in serious accidents that cause loss of life, or significant damage~\cite{Sugiyama2013-ci,Amodei2016-xi}.
    % \item [Public Policy:] Governments are beginning to enforce regulations on the interpretability of certain algorithms in order to ensure that citizens can understand why AIA driven services make the decisions and predictions that they do. A specific example are the algorithms deployed by credit agencies to approve/reject loans~\cite{Wagner2016-ck}.
    % \item [Medicine:] Medical professionals need to understand why data-driven models give predictions so that they can choose whether or not to follow the recommendations. While AIAs can be a very useful tool, ultimately doctors are liable for the decisions they make and treatments they administer~\cite{Jovanovic2016-gw}.
    % \item [Cognitive Assistance:] Systems are being designed as aids for humans to make complex decisions, e.g. searching and assimilating information from databases of legal proceedings. When an AIA presents perspectives and conclusions as data summaries, it must be able to also present justifying evidence and logic~\cite{Gutfreund2016-xe}.
% \end{description}

\brettcomm{
\begin{itemize}
    \item How to know if assurances are effective. Are explicit assurances being perceived as desired? Are implicit assurances overpowering?
    \item component and composite assurances. What happens when assurances are combined vs. when they exist in isolation? How should assurances be combined?
    \item methods, and modes of expression. What human limitations must be considered
    \item from the perspective of AIAs are there assurances that exist for each of the capabilities that can communicate to the different trust targets?
    \item currently assurances are typically `displayed' as static information to a user, can a user be taught over time by planning assurances?
    \item what cognitive effects inhib assurances
    \item how to select the expression of an assurance? Easy to calculate hard to communicate\ldots
    \item measuring effects of assurances
    \item distrust
    \item two-way trust
\end{itemize}
}
This field is nascent, and there are many opportunities for further research along different lines. This section outlines some possible directions for future work that are promising.

\subsection{Properties of Assurances}
Probably the most straight forward approach could be to draw some guidance from Figure~\ref{fig:assurance_classification}, which shows how to classify an assurance. In this survey we investigate, in some detail, `Level of Integration'. However all of the other grayed-out boxes in that figure have open questions that can still be investigated. 
\subsubsection{Source-Target Classification}
    It is convenient to refer to assurances by way of their source and target. More specifically their source AIA behavior (see Figure \ref{fig:AIcapabilities}) and their user trust target component (see figure \ref{fig:Asurane_classes}. Intuitively, there may be a set of different algorithms that are useful for making assurances that convey information about planning to the competence dimension of the user's trust. It is easier to refer to these assurances in terms of their source and target. So, for this example that class of algorithms would be the `planning-competence' class.

    The source AIA capability for an assurance might be most easily thought of by the algorithm it has to operate on. If the assurance operates on, or comes from, a planning algorithm then it would be considered a `planning' source. If the algorithm operated on data for learning patterns, the it would be considered a `learning' source.

    The trust dimensions shown in Figure \ref{fig:Assurance_classes} are the possible targets for assurances. The categories mirror those of the trust model proposed by \citet{McKnight2001-fa}, but with the emphasis on what an AIA has the ability to most readily influence (and consequently where most research is found). The boxes with the beveled corner identify and define the different classes of assurances. All classes are included here for completeness and generality. Although, while it is hypothetically possible for an AIA to influence a persons general `Trusting stance' given enough time\footnote{One might imagine an AIA that specifically speaks to the human about the benefits or drawbacks about trusting even though there might not be evidence to do so, similar to the role a counselor might play}, the gray boxes are not considered further in this survey, as practically no direct research exists in the realm of human-AIA relationships.
    
    Not only is the source-target notation useful shorthand for communicating about the purpose of the assurance, but it is useful in classifying the range of assurance algorithms that exist. There may also be a class of algorithms that span multiple source-target capabilities. For example there may be a kind of algorithm that can give a `learning-competence' assurance, as well as a `planning-competence' assurance. This is especially true since many of the AIA capabilities can overlap. Also, the effects of assurances cannot be guaranteed to affect only one trust dimension (see section \ref{sec:imprecise}).

\subsubsection{Component and Composite Assurances}
Assurances can be either component or composite. This was seen a little through the survey. The definitions are as follows:

\begin{description}
    \item [Component Assurance:] An assurance that originates from a single AIA capability source, and targets a single trust dimension target.
    \item [Composite Assurance:] The combination of more than one component assurance into a single assurance. 
\end{description}

\begin{figure}[!htbp]
    \centering
    \includegraphics[width=0.9\textwidth]{Figures/Assurance_component_composite.pdf}
    \caption{Figure illustrating the difference between component and composite assurances. The existence of multiple assurances does not imply a composite assurances, rather the combination of multiple component assurances into a single assurance constitutes a composite assurance.}
    \label{fig:assurance_mapping}
\end{figure}

Figure \ref{fig:assurance_mapping} illustrates the concepts of component and composite assurances.

\paragraph{Component Assurances:} Component assurances are perhaps the most well researched in the existing literature. This is likely because several verified component assurances are the predecessors to composite ones. A component assurance might include displaying the confidence of a classification prediction, or visualizing a model as discussed in section \ref{sec:q2}.

\paragraph{Composite Assurances:} Composite assurances are assurances that are built of several components. A notable example is the work by \citet{Aitken2016-cv} who propose a measurement called `self-confidence', applicable to Partially Observable Markov Decision Processes (POMDPs). This metric combines five component assurances into a single composite assurance that is meant to distill the information into a value that a novice operator could understand easily. This paper was discussed in more detail in \ref{sec:q2}. 

\subsubsection{Explicit and Implicit Assurances}
\citet{Sheridan1984-kx} briefly alluded to the existence of explicit and implicit assurances when they discussed the nature of how humans behave when working with automated systems. They suggested that the operator's perception of the automated system can be effected by `performance' and its `reports on its own performance'. The terms are more formally defined as,

\begin{description}
    \item [Explicit:] Assurances that are purposefully given to affect the trust of a user.
    \begin{itemize}
        \item Legible motion \cite{Dragan2013-wd}, which is motion calculated with the intent of being more understandable by a human
        \item $R^2$ value, gives some indication of how well the regression accounts for the variance of the data
    \end{itemize}
    \item [Implicit:] All other assurances that aren't explicit.
    \begin{itemize}
        \item Reliability in completing a task. Generally, the object of success is not to affect the user's trust (although this is a nice side-effect).
        \item The way an autonomous vehicle appears. For example something that looks neat will have a different effect on trust, than an AIA with wires dragging on the ground. 
    \end{itemize}
\end{description}

It is important for designers of AIAs to be aware of both, but anything that is consciously designed with the goal of affecting trust is automatically an explicit assurance, from the perspective of the AIA. Another way of stating the ideas is that trust relationships between humans and AIAs will form, but all assurances will be implicit if designers do not consciously consider the trust relationship. 

\subsubsection{Tutoring vs Telling} \label{sec:teach_tell}
    This is also a point rarely seen in the literature surveyed, but critical in the context of different users and long-term human-AIA interaction. We suggest that assurances can also be classified by whether they consider tutoring (or leading) the user to help them understand, or whether they just produce a static and unchanging value regardless of the user, their experience, or their expertise. This is a point mentioned (in terms of explainability) by \citet{Lacave2002-cu}, and \citet{Lacher2014-yc}.

    A tutoring assurance would be a planned, dynamic, sequence of assurances that would change in time to adapt to the user's needs (as discussed in section \ref{sec:consider_human}. This might include modification of assurances to help a user avoid boredom, or to use the system differently in varying circumstances. For example the first time an autonomous vehicle encounters snow with a certain user, it might take time to give special assurances. Or a user that is so used to an AIA that its TRBs begin to drift to disuse, and the AIA gives a special assurance to correct that.

    It isn't surprising that, to our knowledge, no research has been done with respect to tutoring a user in a trust relationship. This is a complex problem to address that would involve understanding how different users learn, and what an appropriate strategy would be to teach them to have appropriate TRBs. However, a rich resource (not investigated in this paper) would be the work on tutoring systems. There is definitely an open area for research that investigates the advantages of tutoring assurances versus those of telling assurances, and how easy they are to implement in contrast to the added time and effort needed to design tutoring assurances.


\subsection{Trust vs. Distrust}
The treatment of assurances in this survey is based, in part, on a model of interpersonal trust. For completeness it will be important to further investigate \textit{distrust}, as reviewed and discussed by \citet{Lewicki1998-ox}, and formalized in \citet{McKnight2001-gz}. Low trust is not the same as distrust, and low distrust is not the same as trust. \citet{McKnight2001-gz} suggest that `the emotional intensity of distrust distinguishes it from trust', and they explain that distrust comes from emotions like: wariness, caution, and fear. Whereas, trust stems from emotions like: hope, safety, and confidence. Trust and distrust are orthogonal elements that define a person's TRB towards a trustee. In this survey, distrust was not considered. However any \emph{complete} treatment of trust relationships, and for our purposes, designed assurances, must consider the dimensions of distrust as well as those of trust. Based on the emotions that drive distrust it will be important to consider in higher-risk human-AIA relationships.

It is not clear to what extent the human-AIA trust model remains effective in the presence of user wariness, caution, or fear. To what extent can behaviors driven by distrust be isolated from those originating from trust? How can those behaviors be detected to begin with? In what circumstances is the extra effort necessary?

\subsection{Human Limitations}
Dealing with human users implies addressing their limitations as well. In \cite{Freedy2007-sg,Riley1996-qm} found evidence of `framing effects' influencing operator behavior. While, this isn't surprising to those familiar with cognitive science, it will likely be surprising to many who are trying to implement assurances. Other cognitive biases and limitations such as `recency effects' (being biased based on recent experience), `focusing effects' (being biased based on a single aspect of an event), or `normalcy biases' (refusal to consider situations which have never occurred before) are also important to consider. 

Besides cognitive biases, humans are also limited in their ability to understand certain kinds of information. Communities that investigate how probabilistic and statistical explanations can be presented to humans will be useful in addressing this question \cite{Rouse1986-dz,Wallace2001-fm,Kuhn1997-qc,Lomas2012-ie,Swartout1983-ko}. It is not immediately clear what methods are most appropriate for application in assurance design, or how they might be applied. Can the AIA detect when cognitive limitations are effecting TRBs? What other limitations of relationships with humans and AIAs need to be characterized? Surely there are equally important limitations identified in psychology, sociology, economics, and others.

\subsection{Expression and Perception of Assurances} \label{sec:express_assurances}
Expression and perception of assurances have been combined in this section because they share several critical aspects. The key points to be considered here in design of assurances are:
(1) Mediums; (2) Methods; and (3) Efficacy.     
%     \begin{itemize}
%         \item Mediums
%         \item Methods
%         \item Efficacy
%     \end{itemize}
    
For explicit assurance design, the medium and method of expression must account for the AIA's limitations. 
Here medium denotes the means by which an assurances is expressed; this could be through any of the senses by which humans perceive, such as sight, sound, touch, smell, and taste. The method of assurance is the way by which the assurance is expressed. For example: a time series plot may be conveyed visually in the typical way, or via a spoken or textual description; in this case the plot or text description is the method, and sight or sound are the different mediums through which it can be communicated. An AIA might be limited in methods of expression, e.g. because it does not have a display or a speaker. %%In such cases, how is the user supposed to receive assurances?

A designer must also consider whether a human can perceive the assurances being given. If so, to what extent is the information from the assurance transfered, or how efficacious is the assurance? A few examples include: an AIA giving an auditory assurance in a noisy room and the user not hearing it (such as an alert bell in a factory where the workers use ear-plugs), or an AIA attempting to display an assurance to a user that has obstructed vision. 
An AIA may also have the ability to store data about its performance, and compute a statistic regarding its reliability -- but if it cannot successfully express (or communicate) that information in some way, the information is useless. 
If an assurance is not expressed, or not perceived by the user, it is useless and has no effect. 

%%...need to connect this paragraph to the previous one, otherwise it's a bit of a non-sequitur
It is also important to bear in mind that users will always produce some kind of TRB when interacting with an AIA (even if this only means choosing to ignore the AIA and not use it), and in the absence of explicit assurances users will instead use implicit assurances to inform their TRBs. 
However, users generally will not know which assurances are implicit or explicit -- e.g. humans participating in research from Quadrants I and II were generally not aware which aspects of the AIAs they interacted with were/were not deliberately designed by the researchers as assurances. 
Hence, there is always a danger that users can latch onto the `wrong' assurances, i.e. AIA features that are not meant to be interepreted as assurances but are nevertheless easily perceived (possibly moreso than intended explicit assurances). There is also the danger of overwhelming the user with too many easily perceivable explicit assurances, e.g. sounding and displaying several alarm indicators at once in an aircraft cockpit. 
%%Recall from Section~\ref{sec:assurances} that, to a user, all assurances are the same, i.e. any property or behavior of an AIA that affects trust is an assurance to a user, and it doesn't matter whether the assurance was designed for that purpose (explicit) or not (implicit). 

\subsubsection{Mediums}
In general, assurances are most often expressed visually. For example an AIA might give visual performance feedback to display different performance characteristics \cite{Chadalavada2015-wx,Muir1996-gt}. Written or spoken natural language can also be used \cite{Wang2016-id} -- given the impressive strides made by NLP researchers and practitioners lately, it is nowadays a simple matter to convert between written natural language and spoken natural language. 
These can be used to augment or replace more conventional audio-visual assurance indicators traditionally used and studied for human-machine/human-automation interaction, e.g. blinking lights, colored boxes in graphical displays, ringing bells/buzzers, recorded voice alerts, etc.
    
Other senses (touch, smell, and taste) are not well explored in literature related to human-AIA trust. Generally, any human sense could be used as a medium. Besides sight and sound, tactile feedback has been used extensively in robotics for `haptic feedback' (where the user receives mechanical feedback through the robot controls). This medium is used to create a more immersive user interface in robotics, to help users feel more connected to the robot (especially important for telerobotics applications). 
While one can imagine smell and taste having obvious applications in designing assurances for a cooking robot, other applications very likely exist and are open to further research.

\subsubsection{Methods}
Assurances associated with displaying AIA performance variables sound banal (e.g. flow rate for an automated pump \cite{Muir1996-gt}), but actually involves a nuanced point: the displayed performance value actually serves to inform the user's own mental model of the trustworthiness of an AIA capability. That is, the user's trust in the AIA's capability does not change only in response to the instantaneous `goodness/badness' of the AIA's performance, but accounts for the past history of the AIA's performance as well as any observed discrepancies between the AIA's expected behavior and its actual behavior.  
The user's trust dimensions (`competence', 'predictability', etc) are then affected by their perception of trustworthiness according to the combined model and data delivered by the display. 
This approach (also noted and discussed by \cite{Wickens1999-la,Sheridan1984-kx,Hutchins2015-if}) is effective, but relies heavily on the implicit assumption that the user will create a `good enough statistical model' of the AIA's behavior from data presented by the AIA. With this in mind, one might train a user to recognize signs of failure/success in different interactions with an AIA as assurances \cite{Freedy2007-sg,Desai2012-rc,Salem2015-md}. 
The main drawback of this idea is that it still relies on users' ability to construct `good enough' mental models of AIA behavior and characteristics from noisy observations to avoid misinterpreting AIA behaviors. 
However, this training can require intensive and costly special effort for non-expert or non-specialist users. 
A more ideal approach in such cases would be to design explicit assurances that help users construct correct/consistent mental process models of AIA behavior and thus reduce the risk of misinterpretation.

More direct methods of expressing assurances include displaying the intended actions, e.g. to indicate movements via visual projection of a planned mobile robot path \cite{Chadalavada2015-wx}. This is subtly but significantly different from making the user infer the intended action. Analogously, natural language expressions (written or otherwise) attempt a more active method of assurance expression. One might also display plans and logic in different formats, e.g. tables, trees, radar charts  \cite{Van_Belle2013-ph, Huysmans2011-th, Hutchins2015-if}, to remove some uncertainty regarding the user's ability to create an adequate mental model. %%%As humans are fond of saying ``You can't assume that I can read your mind!'', in essence more passive expressions from AIAs are relying on humans to read AIA's `minds' (we can't even do that with other humans).

It is often assumed that making an AIA more `human-like' will make it more trustworthy. 
An algorithm may be human-like when it represents knowledge in a human-understandable way, or executes logic in a way that a human can follow. 
A robot that is humanoid becomes more human-like in appearance \cite{Bainbridge2011-pl}, and thus implicitly projects that it has certain physical (and possibly mental) capabilities in common with humans as well. 
A system that uses natural language becomes more human-like in communication \cite{Lacave2002-cu}, and again projects that it has certain capabilities to understand or possibly hold a conversation at some level with a human user. 
The human-AIA trust relationship depends on assurances that, in essence, are conversions from AIA capabilities to human-perceptible/human-understandable behaviors and properties.  
Since assurances are the means of communication by which humans develop trust in AIAs, it is expected that all assurances have to at least be made human-understandable in some way (otherwise assurances will be totally ineffective). Therefore, it can be argued that assurances must make AIAs `human-like' in some regard.  

Interestingly, however, the converse is not true: making an AIA more `human-like' in any arbitrary way does not imply that it automatically provides assurances that make it more trustworthy. 
In \cite{Dragan2013-wd} the AIA is made more trustworthy by making the robot motions more human-like, whereas in \cite{Wu2016-ei} making the AIA more human-like resulted in a decrease of trustworthiness. In this case the difference came from the type of task: in the first case, the AIA (a robot) was physically working in proximity to a human, while in the other case the user was playing a competitive game against the AIA (a computer program). 
It has been observed that humans trust more `human-like' AIAs in more human-like ways \citet{Tripp2011-rx}. 
It is thus plausible to suppose that the term `human-like' can be more formally defined in terms of the difficulty that a typical user would have in relating to the AIA. 
Following on this idea, the benefits/drawbacks of human-like characteristics would be influenced by a user's general impressions and feelings of how trustworthy humans are in similar situations. 
This would also involve aspects of psychology and sociology, and would be very difficult to control and account for. 
Nevertheless, the problem of coping with such factors is an open and important research question that will impact the design of assurances for AIAs. 

It is also worth considering, in more detail, what implications the existence of implicit and explicit assurances means practically for AIA system designers when it comes to considering and implementing assurances. 
Since it is unrealistic for designers to take all possible kinds assurances into account, they will need to focus their efforts on how to identify and focus on only the most important ones. 
The foremost consideration is that an analysis of the interaction between the human and user needs to be made in order to identify the critical assurances for a given scenario. 
For example, in the road network problem, an analysis might find that the most critical assurances are about the competence of the UGV's planner. 
In this case the designer must take time to design an explicit assurance that is directed at the user's perception of the AIA's competence -- let's call this a `planning-competence' assurance. 
One difficulty arises from this approach is that there doesn't seem to be a way to determine what other implicit assurances might drown out explicit assurances. 
Continuing the example, the system designer may come up with a well thought out planning-competence assurance, but failed to consider the effect of how the UGV appears -- it may be old, have loose panels, and rust holes. Generally, designers overlook implicit assurances (i.e. do not consider them explicitly in design) because they assume that they will have no effect (i.e. why does it matter if there are rust-holes if the UGV works?). This can stem from ignorance of human-AIA trust dynamics, or failure to identify which assurances are most important to users.
%
% \edit{...move to end...trim also -- not sure it's saying much...and feels out of place given next paragraphs}
% Any of these methods can be more or less effective based on the task and context in which they are used. 
% How should uncertainly be displayed (i.e. as a distribution, summary statistics, fractions or decimals)?  Unsurprisingly we find that the answer is `it depends' \cite{Chen2014-dk,Wallace2001-fm,Kuhn1997-qc,Lacave2002-cu}. 

    While it might be desirable, it is generally unreasonable and practically inefficient to attempt a study of \emph{every possible} assurance from an AIA to a user and then select the most important. Perhaps one way a designer might try to identify which assurances are important is to perform user studies, to obtain feedback about which characteristics of the AIA most affected user trust. An approach like this would help determine if explicit assurances are being picked up, and if there are implicit assurances that are overly influential or that overwhelm explicitly designed assurances. With such feedback, designers would have a realistic idea about whether their explicitly designed assurances are having the desired effect on user TRBs. We use the UGV road-network problem to illustrate: after designing an explicit assurance, the supervisor-UGV team could work together in a training mission. Afterwards, the supervisor could rank the different behaviors/properties of the AIA affected their trust in it. In this way, the critical implicit and explicit assurances will be identified. If the explicit assurance is near the top of the list of influencing assurances, then it is working; if not a re-design may need to occur. 
Of course, even this approach has its own caveats, as factors such as the experience of the user, or the nature of the information being displayed, must be taken into account, as these will affect the user's ability to interpret explicit assurances or extract implicit assurances on their own. Absent data for analyzing such considerations, the best that can be done is to select explicit assurances that will work for the largest group of typical users of the AIA. A sufficiently advanced AIA might also learn how best to communicate to individual users. %%, although such adaptability can also make it difficult to formally establish the efficacy of assurances via user studies. 

One final point is that several potential sources for explicit assurances lack well-established human-understandable expressions, and thus are not yet widely utilized as effective assurances. For example, it is unclear how an AIA can best express that it has been formally validated and verified for similar operational settings. 
Similarly, it is unclear how information related to random variables can be best communicated to users besides showing them histogram or probability distribution plots (only useful for 1 or 2 dimensional distributions), or displaying statistics such as means and variances. 
Investigating and understanding how such useful, but otherwise difficult to understand, types of information can be expressed as explicit assurances will be critical to enhancing human-AIA trust relationships. 

\subsubsection{Efficacy}
Some kinds of expression are very `one-dimensional' in that they only rely on one medium or method. This, again, has been seen in practice by the use of plotting a certain AIA performance variable value over time. Because of this, much of the research to date involves assurances that are not robust to loss in transfer, i.e. the approaches rely heavily on a specific medium and method to work, otherwise the whole assurance is rendered useless. 
Hence, the problem of robustly communicating assurances remains an open research question. 
An analogy can be made here to a person speaking to another person with their voice, while also making facial expressions and gestures with their hands, thus simultaneously utilizing several mediums/methods helps to ensure the effectiveness of an assurance. 
If the person instead tries to simply repeat the same message over and over many times to the other person in a monotone voice without changing facial expressions or making any other gestures, then this would be considered inefficient, especially if the message is not received or considered to be effective after the first attempt. %This raises the idea of efficiency in expression. 
    
    %...again this paragraph feels out of place and disjointed...going to try to rearrange the ideas here, since they don't make sense in current form, not clear what the point is...
    %%
%    Perhaps less obvious is a situation in which the user has to supplement an incomplete assurance. A user can create a mental model of the trustworthiness of an AIA capability based on repeated observations over time. Creating this mental model takes time/effort, and the model is prone to cognitive biases. 
 %   In this case the assurance is communicated slowly and indirectly. Generally, a highly effective assurance would have precise information communicated in a way that is easy for the user to perceive, with little loss. Whereas, an inefficient and ineffective assurance may be more vague and wasteful (i.e. repeating the same thing many differet times), and susceptible to loss in communication. The solutions to efficacy lay in selecting appropriate methods, and mediums for expression of the assurance, and by designing for appropriate levels of redundancy to ensure that the assurance is received.
We might also consider situations where a user has to supplement an incomplete assurance from an AIA on their own. 
For instance, in the UGV road network scenario, suppose the hypothetical planning-competence assurance discussed earlier is implemented in such a way that it is communicated slowly and indirectly back to the supervisor. 
The supervisor can create and use a mental model of the AIA's capabilities (based on repeated observations over time or previous interactions) to `fill in' what an AIA might mean when it tries to express this assurance (e.g. to anticipate what it might say or display next).  
Creating and leveraging this mental model takes time/effort on the part of the user, and the process of interpreting the assurance by guessing at its content or meaning opens becomes prone to cognitive biases. %(e.g. if the user becomes impatient and starts to incorrectly categorize/infer the meaning of assurances printed on a display, and then acting on the incorrect interpretations before they have even finished printing). 
Ideally, a highly effective assurance communicate in a precise and direct way that is easy for the user to recognize and understand, with little loss in content or ability to act correctly on the assurance in a given situation. 
The keys to efficacy lie in the selection of appropriate methods and mediums for assurance expression, and in the design of appropriate levels of redundancy to ensure that assurances are correctly received and interpreted in a timely manner.

\subsection{Observing Effects of Assurances} \label{sec:measuring_effects}
    Since assurances are meant to influence TRBs, it is important to quantify those effects so that:  1) the AIA system designer can understand how effective the assurances are; and 2) the AIA can observe and respond to/adjust the efficacy of its assurances. To our knowledge, there has not been any work that enables an AIA to observe user responses to assurances and then adapt behaviors appropriately (at least not in the trust cycle setting). 
    Yet, this capability is crucial for enabling AIAs to meet different user's needs. 
Theoretically, any method that is made for the designer to measure the effects of assurances could also be deployed by the AIA itself to assess the effects of assurances on user TRBs. 
The surveyed literature gives some insights into how that has been done to date; namely, there are two main approaches: (1)  Gather self-reported changes in trust from human users; and (2) Measure changes in user's TRBs. 
    
\subsubsection{Self-Reported Changes in Trust} Assessing self-reported changes in trust involves asking users to answer questions, such as `how trustworthy do you feel the system is?'; or `to what extent do you find the system to be interpretable?', either while using a system or afterwards \cite{Mcknight2011-gv,Muir1996-gt,Wickens1999-la,Salem2015-md,Kaniarasu2013-ho}. These kinds of questions are useful in verifying whether the assurances are having the expected effects. It is not unreasonable to imagine that an AIA might be equipped to ask users questions about their trust, process those responses, and modify assurances appropriately.

Self-reports are the most useful when trying to understand the true effects of an assurance. Does a certain assurance, assumed to affect `situational normality', actually do that? 
Does displaying a specific plot actually convey information about `predictability'? 
There is much room for research in this area, which can be used to inform the selection of the methods of assurance. 
However, changes in self-reported trust do not always result in changes in TRBs \cite{Dzindolet2003-ts}. From the AIAs perspective this means that --- unless the object of the assurances is to make the person's level of self-reported trust change --- the assurances may not be providing any tangible benefit. 
As previously discussed, a more concrete objective for designing assurances in human-AIA interaction is to elicit appropriate TRBs from the human user. 
From this perspective, measuring changes in TRBs is the more direct and objective approach to assessing effectiveness.% of assurances.

\subsubsection{Measuring Changes in TRBs} Researchers often measure how long AIAs are able to run under full autonomy, before the autonomy is turned off by users \cite{Freedy2007-sg,Desai2012-rc}. 
Other researchers assess user's willingness to cooperate with AIAs \cite{Salem2015-md,Wu2016-ei,Bainbridge2011-pl}. 
A more ideal metric is the likelihood that users will use certain AIA capabilities `appropriately'. 
However, this is more difficult to formally define/calculate in different situations. 
As a concrete example for the UGV road network problem, %%%there is not an option to `turn off' the UGV's autonomy --but the user could switch off the planning feature...
the remote supervisor can make decisions such as accepting a plan or policy formulated by the UGV, or switching off the autonomous planner to provide their own plan to be implemented by the UGV. 
In this situation, the effect of assurances might be measured by how likely the operator is to accept a generated plan, instead of overriding it (recall that the goal may not be to have the generated plan accepted 100\% of the time, but rather that it be accepted with respect to how appropriate it is in a given context).

In practical application, assurances designed to lead the user to believe that the AIA is more competent, predictable or reliable than the user initially believed do not achieve their objectives if the user doesn't treat the AIA any differently than before/without the assurances. 
This assumes that it is possible for appropriate TRBs to be defined and observable in the first place. 
If, for example, an appropriate TRB hypothetically involves user verification of a sensor reading, can the AIA perceive whether or not such behavior takes place? 
%\edit{...good following: need to revise/polish...}
If the user is queried about this, can the user always be trusted to provide an honest/correct response or behave appropriately? 
%Is there a way to verify the user behavior is actually appropriate? 
This issue has gained notoriety with the current generation of autonomous cars, where users still need to attentively sit in the driver's seat in case the vehicle cannot perform correctly. This underscores the importance of designing methods for perceiving (in)appropriate TRBs. 
%
% \subsection{The Imprecise Nature of Assurances} \label{sec:imprecise_nature}
    % Due to the nature of trust (and humans in general), a single assurance might be targeted at influencing the competence dimension of trust, but it may also have effects on other dimensions. As an example an assurance that targets predictability may also have an affect on the probability of depending.
%
    % Besides being difficult to separate effects on a single user, individual users are different as well. Thus no assurance will have an identical effect when given to two separate users. This makes it difficult to have precise effects on user trust behaviors.
%
    % One might attempt to mitigate this uncertainty by using expressions that are more precise than others, such as displaying a probability distribution rather than on a maximum likelihood. This gets into some considerations about how the presentation of information affects the ability of a human to understand.



%%%%%%%%%%%%%%%%%%%%%%%%%%%%%%%%%%%%%%%%%%
\vspace{-0.1 in}

\section{Conclusions}\label{sec:conclusions}
\section{Discussion}\label{sec:discussion} 
    From the review of Quadrants I. through IV. of the formal/informal, explicit/implicit plane (see figure \ref{fig:trust_assurance_intention}), several categories for assurances present themselves, and assurances can be described in a more comprehensive manner.

    \subsection{A Refined Perspective On Assurances}
Given the body of literature surveyed, assurances will now be classified in the following subsections.
\subsubsection{Explicit and Implicit Assurances}
\citet{Sheridan1984-kx} briefly alluded to the existence of explicit and implicit assurances when they discussed the nature of how humans behave when working with automated systems. They suggested that the operator's perception of the automated system can be effected by `performance' and its `reports on its own performance'. The terms are more formally defined as,

\begin{description}
    \item [Explicit:] Assurances that are purposefully given to affect the trust of a user.
    \begin{itemize}
        \item Legible motion \cite{Dragan2013-wd}, which is motion calculated with the intent of being more understandable by a human
        \item $R^2$ value, gives some indication of how well the regression accounts for the variance of the data
    \end{itemize}
    \item [Implicit:] All other assurances that aren't explicit.
    \begin{itemize}
        \item Reliability in completing a task. Generally, the object of success is not to affect the user's trust (although this is a nice side-effect).
        \item The way an autonomous vehicle appears. For example something that looks neat will have a different effect on trust, than an AIA with wires dragging on the ground. 
    \end{itemize}
\end{description}

It is important for designers of AIAs to be aware of both, but anything that is consciously designed with the goal of affecting trust is automatically an explicit assurance, from the perspective of the AIA. Another way of stating the ideas is that trust relationships between humans and AIAs will form, but all assurances will be implicit if designers do not consciously consider the trust relationship. 

\subsubsection{Source-Target Classification}
    It is convenient to refer to assurances by way of their source and target. More specifically their source AIA behavior (see Figure \ref{fig:AIcapabilities}) and their user trust target component (see figure \ref{fig:Asurane_classes}. Intuitively, there may be a set of different algorithms that are useful for making assurances that convey information about planning to the competence dimension of the user's trust. It is easier to refer to these assurances in terms of their source and target. So, for this example that class of algorithms would be the `planning-competence' class.

    The source AIA capability for an assurance might be most easily thought of by the algorithm it has to operate on. If the assurance operates on, or comes from, a planning algorithm then it would be considered a `planning' source. If the algorithm operated on data for learning patterns, the it would be considered a `learning' source.

    The trust dimensions shown in Figure \ref{fig:Assurance_classes} are the possible targets for assurances. The categories mirror those of the trust model proposed by \citet{McKnight2001-fa}, but with the emphasis on what an AIA has the ability to most readily influence (and consequently where most research is found). The boxes with the beveled corner identify and define the different classes of assurances. All classes are included here for completeness and generality. Although, while it is hypothetically possible for an AIA to influence a persons general `Trusting stance' given enough time\footnote{One might imagine an AIA that specifically speaks to the human about the benefits or drawbacks about trusting even though there might not be evidence to do so, similar to the role a counselor might play}, the gray boxes are not considered further in this survey, as practically no direct research exists in the realm of human-AIA relationships.
    
    Not only is the source-target notation useful shorthand for communicating about the purpose of the assurance, but it is useful in classifying the range of assurance algorithms that exist. There may also be a class of algorithms that span multiple source-target capabilities. For example there may be a kind of algorithm that can give a `learning-competence' assurance, as well as a `planning-competence' assurance. This is especially true since many of the AIA capabilities can overlap. Also, the effects of assurances cannot be guaranteed to affect only one trust dimension (see section \ref{sec:imprecise}).

\subsubsection{Component and Composite Assurances}
Assurances can be either component or composite. This was seen a little through the survey. The definitions are as follows:

\begin{description}
    \item [Component Assurance:] An assurance that originates from a single AIA capability source, and targets a single trust dimension target.
    \item [Composite Assurance:] The combination of more than one component assurance into a single assurance. 
\end{description}

\begin{figure}[!htbp]
    \centering
    \includegraphics[width=0.9\textwidth]{Figures/Assurance_component_composite.pdf}
    \caption{Figure illustrating the difference between component and composite assurances. The existence of multiple assurances does not imply a composite assurances, rather the combination of multiple component assurances into a single assurance constitutes a composite assurance.}
    \label{fig:assurance_mapping}
\end{figure}

Figure \ref{fig:assurance_mapping} illustrates the concepts of component and composite assurances.

\paragraph{Component Assurances:} Component assurances are perhaps the most well researched in the existing literature. This is likely because several verified component assurances are the predecessors to composite ones. A component assurance might include displaying the confidence of a classification prediction, or visualizing a model as discussed in section \ref{sec:q2}.

\paragraph{Composite Assurances:} Composite assurances are assurances that are built of several components. A notable example is the work by \citet{Aitken2016-cv} who propose a measurement called `self-confidence', applicable to Partially Observable Markov Decision Processes (POMDPs). This metric combines five component assurances into a single composite assurance that is meant to distill the information into a value that a novice operator could understand easily. This paper was discussed in more detail in \ref{sec:q2}. 

\subsubsection{Tutoring vs Telling} \label{sec:teach_tell}
    This is also a point rarely seen in the literature surveyed, but critical in the context of different users and long-term human-AIA interaction. We suggest that assurances can also be classified by whether they consider tutoring (or leading) the user to help them understand, or whether they just produce a static and unchanging value regardless of the user, their experience, or their expertise. This is a point mentioned (in terms of explainability) by \citet{Lacave2002-cu}, and \citet{Lacher2014-yc}.

    A tutoring assurance would be a planned, dynamic, sequence of assurances that would change in time to adapt to the user's needs (as discussed in section \ref{sec:consider_human}. This might include modification of assurances to help a user avoid boredom, or to use the system differently in varying circumstances. For example the first time an autonomous vehicle encounters snow with a certain user, it might take time to give special assurances. Or a user that is so used to an AIA that its TRBs begin to drift to disuse, and the AIA gives a special assurance to correct that.

    It isn't surprising that, to our knowledge, no research has been done with respect to tutoring a user in a trust relationship. This is a complex problem to address that would involve understanding how different users learn, and what an appropriate strategy would be to teach them to have appropriate TRBs. However, a rich resource (not investigated in this paper) would be the work on tutoring systems. There is definitely an open area for research that investigates the advantages of tutoring assurances versus those of telling assurances, and how easy they are to implement in contrast to the added time and effort needed to design tutoring assurances.


\subsubsection{The Imprecise Nature of Assurances}\label{sec:imprecise}
    Due to the nature of trust (and humans in general), a single assurance might be targeted at influencing the competence dimension of trust, but it may also have effects on other dimensions. As an example an assurance aimed at influencing Predictability may also have an affect on the Probability of Depending.

    Besides being difficult to separate, each user is different. Thus no assurance will have an identical effect when given to two separate users. This makes it difficult to have precise effects on user trust behaviors.

    \textbf{I am not sure how I want this argument to go, I want to highlight that it is theoretically possible to have some affect on each of these attributes, but that some are more practical. Two main things need to be considered 1) time-scale (how long will it take to make a noticeable change), and 2) what SHOULD be influenced in order to appropriately calibrate TRBs (it probably isn't acceptable to lie in order to manipulate a user's trust)?}


    \subsection{A Note Regarding Distrust}
    The treatment of assurances in this survey are based, in part, on a model of trust. For completeness it is important to mention distrust. As reviewed and discussed by \citet{Lewicki1998-ox}, and formalized in \cite{McKnight2001-hm,McKnight2001-gz} low trust is not the same as distrust, neither is low distrust the same as trust. \citet{McKnight2001-gz} suggest that ``the emotional intensity of distrust distinguishes it from trust''. They explain that distrust comes from emotions like: wariness, caution, and fear. Whereas, trust stems from emotions like: hope, safety, and confidence. Trust and distrust are orthogonal elements that define a person's TRB towards a trustee. \citet{Lewicki1998-ox} list several behaviors that stem from different combinations of trust and distrust; these are shown in Figure \ref{fig:distrust_table}.

    \begin{figure}[!htbp]
        \centering
        \includegraphics[width=0.6\textwidth]{Figures/distrust_table}
        \caption{Trust--Distrust table from \cite{Lewicki1998-ox}}
        \label{fig:distrust_table}
    \end{figure}

    In \citet{McKnight2001-gz} models for trust and distrust are proposed, and in later studies empirical evidence to validate some of the dimensions of both models was performed. \citet{McKnight2002-qx} is an empirical study that validates the model and quantifies the relationships between the model dimensions, and \citet{McKnight2004-vv} investigates and quantifies the relationship between dispositional distrust and web-site usage. Finally, \citet{McKnight2006-ce} revisits the position of \citet{McKnight1998-ty} and concludes that research had largely confirmed the model.

    % \textbf{add a small mention of McKnight's definition that distrust is the total opposite of trust?}

    In this survey distrust was not considered; this was in a effort to reduce the scope, which is already at risk of being too large. However, it must be made clear that any \emph{complete} treatment of trust relationships, and for our purposes, designed assurances, must consider the dimensions of distrust as well as those of trust. 
    
    Distrust has been shown to vary with perceived risk \cite{McKnight2004-vv}. This means that the implicit assumption of this paper is that risk in 1-on-1 human-AIA relationships is held constant. We claim that this is a realistic assumption for many practical applications.



\section{Conclusions}\label{sec:conclusions}
    Now, more than ever, there is a great need for humans to be able to trust the AIAs that we are creating. Assurances are the method by which AIAs can encourage humans to trust them appropriately, and to then use them appropriately. We have presented here a definition, case for, and survey of assurances in the context of human-AIA trust relationships. These assurances have been classified according to different properties.
    
    The survey was performed, to some extent, from a standpoint of designing an unmanned ground vehicle that is working in concert with a human supervisor. However, the theoretical framework, and classification of assurances is meant to be general in order to apply to a broad range of AIAs.

    While the basic definition of assurances (i.e. feedback to user trust, in the human-AIA trust cycle) is simple from a theoretical standpoint, the exercise of gathering related literature helped to illuminate some important considerations and details regarding the design of assurances. These include the observations that assurances can be implicit or explicit, component or composite, tutoring or telling, and that they can be classified by their source AIA capability and target user trust dimension.

    We propose that, when possible, TRBs should be calibrated instead of trust. This is due to the nature of subjective surveys and human psychology, which is that a human might rate their trust higher, but act no differently (as discussed in \cite{Dzindolet2003-ts}). 

    From the survey we learned that much of the formal research has focused on implicit properties of AIAs, which were not designed to affect trust, but that are a property of the AIA itself. We learn that trust can be measured by user's actions, which we call TRBs, or by self-reported measures in the form of questionnaires. Due to the inherent differences between people, TRBs are the more quantitative way to measure the effects of assurances, in some sense an AIA should care more about appropriate TRBs than appropriate trust as measuring trust is quite subjective.

    There is sometimes an understanding that AIAs should be more `human-like' to be more trustworthy, but we have seen that explicit assurances can use simulated attributes of a human to improve and degrade the user TRBs (adding human-like motions makes an AIA more trustworthy in a working scenario, while adding human-like reaction times in a coin entrustment game reduces trustworthiness). It is common, in order to have more interpretable models, to use a simple global model, and a more complex local model when accuracy is needed. There are many different situations in which assurances must be used (i.e. make models, evidence, or reasoning more predictable, or to show competence, or to display or communicate assurances in certain ways), and so there will never be a single perfect assurance that performs the best in all situations.

    There is a fairly large body of research that is focused, in some way, on influencing trust in human-AIA relationships. However, there is probably a larger portion of research that deals with techniques that would be useful in designing assurances, but that has not directly or consciously considered trust as a formal design goal. Research from these areas (such as V\&V, active learning, and safety) should provide a rich collection of methods to be studied and formally applied to human-AIA trust relationships.

\newpage



%%%%%%%%%%%%%%%%%%%%%%%%%%%%%%%%%%%%%%%%%%
\bibliographystyle{ACM-Reference-Format}
\bibliography{References}
\end{document}
