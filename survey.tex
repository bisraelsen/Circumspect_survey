Whereas other researchers have noted the \textit{existence} of assurances, we now directly consider the question: what, exactly, \textit{are} assurances, and how can they be \textit{practically designed} into AIAs? 
This section surveys the related literature to understand what algorithmic approaches can be used to design AIA assurances. 

As discussed in Sec.~\ref{sec:assurances} there are many different ways of classifying assurances. In evaluating different practical approaches to designing assurances we have found that it is easiest to consider the `level of integration' of the assurance in the AIA. The level of integration of an assurance refers to the extent to which the core functionality of the AIA is dependent on the existence of that assurance. Assurances naturally lie on a continuum between being totally integral to the core function of the AIA, and not being integral at all but being generated by artifacts of the underlying task or AIA functionality.  Figure~\ref{fig:assurance_continuum} illustrates this continuum. For simplicity we will sometimes refer generally to assurances as `integral' or `supplemental' based on whether they lie on the left or right side of the figure respectively~\footnote{Note that, while Fig.~\ref{fig:assurance_continuum} shows that the assurance classes occupy large spaces on the continuum, this is not referring to individual `component' assurances. An individual component assurance cannot be both integral and supplemental at the same time; it is located at a point on the continuum. This is not to say that an AIA cannot, simultaneously, have many assurances distributed over the assurance integration continuum, but that these assurances must be considered as separate.}. In the literature we have identified seven main categories for designed assurances that span this continuum.

Practically, understanding the level of integration of different assurances is useful because doing so can indicate at which point different assurances need to enter the design process. For example, an assurance that is integral to the AIA must necessarily be considered from early on in the design process, whereas one that is supplemental can feasibly be added much later. Also, assurances at different levels of integration have similarities in their affects; because of this designers may make different decisions regarding assurance design based on their specifications and goals.

While assurances cannot \emph{guarantee} appropriate TRBs from a user, integral assurances are generally built with the aim of intrinsically guaranteeing---as nearly as possible---certain effects on user trust and TRBs. In contrast, supplemental assurances are typically weaker, and \emph{encourage} appropriate TRBs; they rely much more on the uncertain relationships with human users. Problems can arise, for example, when a designer expects supplemental assurances to have the same effects on TRBs as those of integral assurances. This should generally not be expected. The remainder of this section is dedicated to discussing each of the seven categories in more detail.

\begin{figure}[!t]%[htbp]
    \centering
    \includegraphics[width=1.0\textwidth]{Figures/Assurance_Integration.pdf}
    \caption{Depiction of the continuum of the level of integration of algorithmic assurances. To the left are those assurances that are integral to the key functions of the AIA. On the right are assurances that are not integral to performance, also called `supplemental' assurances.}
    \label{fig:assurance_continuum}
\end{figure}

\section{Methodology} \label{sec:methodology}
    In this survey, I attempt to look at research of those who are formally and informally addressing the idea of human-AIA trust. In particular, I focus on a ideas that might be applicable to the trust relationship between a single human user (User), and a single autonomous vehicle. While theoretically a two-way trust model could be considered, I will only be considering a one-way trust relationship, that is that the autonomy has perfect trust towards a user.

    It should be noted that it is almost impossible to perform a comprehensive survey of all assurances due to the broad nature of assurances in general. One could rightly argue that metrics like gain and phase margins are assurances for control engineers, as are training and test accuracy for machine learning practioncioners. However, it is my opinion that the somewhat narrow view of the surveyed literature does not significantly hinder the definition or classification of assurances.

    In order to find applicable research I first looked at papers that formally addressed trust and tried to create models of it; this with the aim of trying to understand how it might be influenced. Secondly, I looked at some historical research regarding trust between humans and some form of non-human entity. This mainly lead to e-commerce literature, automation literature, and human-robot interactions. Third, I investigated work regarding `interpretable', `comprehensible', `transparent', `explainable', \ldots and other types of learning and modeling methods. Finally, I searched for research disciplines that are investigating methods that would be useful as assurances, but of which trust is not the main focus.

    With this information, I try to make an informed definition and classification of assurances based off of empirical information of methods that are currently in use or being investigated. In doing so I was able to identify several areas that are open for further research. 

\subsection{Value Alignment} \label{sec:value_alignment}


\subsubsection{Common Approaches:}

\paragraph{stuff:}


\subsubsection{Grounding Example:}
In the case of the `VIP Escort' problem (described in Section~\ref{sec:mot_example}), value alignment might be used as an assurance in the following way:

We make the following assumptions

\begin{itemize}
    \item The UGV has just begun an attempt to escape the road-network
    \item 
    \item 
\end{itemize}

\paragraph{\textbf{Discussion of Example:}} 

%%for NRA TODO: edit/rewrite as follows:
%%(i) assurance argument: what specifically is the assurance signal? as opposed to value alignment, the signal here is some visualization or access to the decision-making process itself...the user is not just looking at the end result or utility of the AIA decision making process, or at behaviors that try to align the AIA's behavior/utility with the user's intent frame. Rather, the user gets some form of access to dig deeper into the AIA behavior to see what informs the utility function, or more generally the different steps along the way that generate different parts of the AIA's actual behavior (as opposed to a simplified or approximate explanation of its behavior, which is covered later)...in almost all cases, the model or intermediate model outputs are generally displayed to or accessed by users in some graphical manner, with some accompanying text/dialog/annotations...
%%(ii) what is mechanism for generating appropriate TRBs? if user gets to dig in, then they get to assess competency and predictability components of trust ; if models are interpretable, then user can in principle understand how to `do the task themselves in exactly the same way an AIA would do it' (i.e. by programatically imitating AIA's behavior on task)-- this follows a theory of mind argument, in that it should allow user to build a better `mental model' of what the AIA would consider to be `situational normality' and how AIA would handle different situations. Caveat here is that this could go south if the user mis-interprets or only understands part of what the AIA is actually doing, either in terms of interpretable model or the nature of the task itself -- this is a significant risk in highly complex or specialized problems where user may not have sufficient training or expertise. This also poses concerns for how/when user can access such models -- unlike value alignment (where user accesses assurances only through behavior of AIA itself), user has more freedom in deciding when and how to peek under the hood -- relates somewhat to problem of information visualization discussed later, except that here the information being given to the user are the actual AIA algorithms, as opposed to by products or after effects of those capabilities
%%(iii) how can designers build/exploit interpretable models for AIAs, i.e. what techniques available for interpretable model building? This gets to summarizing what Brett has written below -- can keep papers clustered more/less as they are, but need to repeat the assurance arguments above: 
%%(a) assessment of interp: post hoc approaches that map models to domain knowledge, place constraints on model inputs, or examine handling of known nonlinear effects;  
%%(b) creating interpretable models: compress this section into key ideas with citations of relevant papers/methods: model could be interpretable from outset like a decision tree (at the expense of performance), or could be less interpretable from outset and then post-fit with another interpretable model. Important thing to note is that most of these are designed around machine learning and pattern recognition problems -- and as such require that the tasks being performed by the AIA to be mappable to these kinds of problems; this generally means that some form of training data is required for supervised or unsupervised learning. ALSO: second approach of building a more interpretable model around an intial model technically counts as a `supplementary' method, in that this interpretable -- DOES THIS ALREADY COME LATER IN SURVEY???
%%(c) human-in-loop learning -- very similar to value alignment, except that not necessarily only concerned with learning utility, but also other kinds of features or useful tidbits that aid in AIA decision making -- follows same theory of mind argument from before: user will have same mental grounding as AIA, so trust can be calibrated appropriately to get better understanding of predictability, competence, and situation normality...Julie Shah's work on learning LTL rules from data?? can maybe leave (c) out for time being....
\subsection{Interpretable Models and Processes} \label{sec:interp_models}
\nisarcomm{add `processes' to label of this category...}
Another way to provide assurances about AIA conformance to user intent frames is to expose the models and algorithmic processes governing its actions directly to the user. If these models and processes also happen to be easy for users to interpret, then the user can (ideally) acquire a well-formed and highly predictive `theory of mind' for the AIA's behavior, with little or no effort . 
\citet{Doshi-Velez2017-xy} give an argument for why interpretability is critical in AIA systems since interpretability `is used to confirm other important desiderata of [machine learning] systems'. 
Yet, perhaps unsurprisingly, `interpretability' and the attendant desiderata still elude formal universally accepted definitions. 
%%Based on our survey, and building on the `easily accessible theory of mind' notion: we define an interpretable model or process here as one which can be fully comprehended by a typical human user, to the degree that the user herself could (given access to the model/process, and enough time and resources for performing required intermediate computations) precisely predict the results that an AIA would obtain on a given task. In other words: a model or process made so easy to understand that a human could do it. \nisarcomm{need to think carefully about implications of this definition: in principle any human could execute any algorithm given enough time/computation, so this isn't saying much...something else needed here to refine this idea...}. 
\cite{Doshi-Velez2017-xy} use the words `interpretable' and `explainable' interchangeably. In contrast, we treat them as distinct descriptors. We discuss models that are inherently interpretable here, and models that can be understood by explanation in Section~\ref{sec:reduce_complexity}.  The difference is that interpretability (in our view) implies that the actual process/model used by an AIA is self-explanatory, whereas explainable models can be made interpretable by post hoc operations but do not necessarily explain the actual model/process used by an AIA. 
Being able to interpret the actual model/process used by an AIA helps human users to more appropriately understand their behaviors, and thus exhibit appropriate TRBs in turn. This approach to assurance also captures broader AIA processes and models that rely on rules, heuristics, etc., rather than just those that rely on optimization of some particular utility. 
%%\nisarcomm{need to unpack this still further -- see tex comments above }

\subsubsection{Common Approaches:}
Two main approaches to designing interpretable AIA models and processes are considered here. 
The first is to \emph{assess an existing set of candidate models/processes} in order to evaluate their interpretability for a particular problem, and then select the best candidate. 
This is typically done with certain classes of models or solution processes, e.g. whether to use decision trees vs. decision tables for a given planning task. 
The second is to \emph{synthesize interpretable models/processes} by leveraging human designer input during the model/solution-building process. 
The first approach requires pre-defined measures of interpretability, and thus some mechanism for capturing ability to gain insights into competence, predictability, and situation normality. This also presupposes that the model/process candidates are inherently interpretable along these lines to begin with, which may rule out certain model/process families that perform well on certain tasks. 
The second approach allows designers to apply domain knowledge to determine metrics for interpretability, although this can lead to solutions that do not perform as well as those that are less interpretable. 
%%\nisarcomm{where and hwo do (intrinsic) assurances fall out of these?}

What is the assurance mechanism that potentially leads to proper TRBs in either approach? 
Essentially, allowing the user to access and examine an interpretable model/process also allows them to simultaneously assess competency, predictability, and situational normality components of trust. If the models/proceses are interpretable, then a user would understand exactly how the AIA would perform its task (i.e. down to a mechanical/programmatic level). 
This gives the user a `mental model' of what the AIA would consider to be situational normality and how AIA would respond in different situations (predictability and competence). 
The caveat here is that incorrect TRBs may arise if the user mis-interprets or only understands part of the model/process. 
This is a significant risk in highly complex or specialized problems, where users may not actually have sufficient training or expertise. This also poses concerns for how/when user can access interpretable AIA components. 
Unlike value alignment (where user accesses assurances only through behavior of AIA itself), the user has more freedom in deciding when and how to `peek under the hood'. This relates to assurances based on information visualization discussed later, except that here the information being given to the user are the actual AIA algorithms themselves, as opposed to by products or after effects of those algorithms. \nisarcomm{I need to trim this down a bit?}

\paragraph{Assessing Interpretability:}
\citet{Van_Belle2013-ph} suggested three ways to ascertain the level of interpretability and potential utility of learned models (compare to categories proposed by \citet{Lipton2016-ug}): 1) Map them to domain knowledge; 2) Ensure safe operation across the full operational range of model inputs; and 3) Assess whether important non-linear effects are accurately accounted for. This work identifies certain strengths and weaknesses of different techniques, but ultimately concludes that no method is clearly best in all situations. 
%%%\nisarcomm{need to say a little bit more about the actual methods proposed -- how does one, for instance, go about doing 1 or 2 or 3? don't need to give specific details, but mention what techniques available}
%
Along similar lines, \citet{Huysmans2011-th} compared decision trees, decision tables, propositional if-then rules, and oblique rule sets to understand which set of methods is `most interpretable'. It was experimentally determined that decision trees and tables tend to be easier to interpret, but it is noted that each method could perform better than others in different applications. For example decision trees and tables are typically better suited for answering a symbolic question (which requires a local understanding of a model) like: \emph{how does the model classify observation $X$'?}. This is in contrast to a spatial question (which requires a global understanding of the model) like: \emph{is it correct that applicants with a high income are more likely to be accepted than applicants with a low income?}. 
%
Having quantified the interpretability of a model given different classes of problems, and different requirements of users the appropriate model can then be selected during design to fit the needs of a specific application.

\paragraph{Interpretable Model Synthesis:}
\citet{Ruping2006-xj} asks how classification results and the accuracy-interpretability trade-off can be made more transparent to those who design and use classifiers. They explore one approach by combining simpler global models with more complex local models that are built around learning results (\citet{Otte2013-oo} and \citet{Ribeiro2016-uc} implement similar ideas as well). Figure \ref{fig:ruping} illustrates this idea.

%%%%%%%%%%%%%%%%%%%%%%%%%
\begin{figure}[htbp]
    \centering
    \includegraphics[width=0.5\textwidth]{Figures/global_local}
    \caption{Example of simple global interpretable learning model on the left, and on the right a more complex locally interpretable learning model that can be used when more precise understanding of a specific decision made by the learner is required. }
    \label{fig:ruping}
\end{figure}
%%%%%%%%%%%%%%%%%%%%%%%%%

Considerable effort has also gone into endowing `grey box' and `black box' models with interpretable features. 
For instance, \citet{Abdollahi2016-vn} investigate making collaborative filtering models more interpretable by using a conditional restricted Boltzmann machine (RBM). \citet{Ridgeway1998-lv} use `weight of evidence' (WoE) as a boosting method that is more amenable to interpretation, and show that WoE is on par with AdaBoost. \citet{Choi2016-by} construct a recursive attention neural network to remove recurrence on the hidden state vector, and instead add recurrence on the visits of patients to doctors, as well as on different diagnoses during those visits. In this way the model is able to predict possible diagnoses in time, and a visualization can be that that indicates the critical visits and diagnoses that lead to that prediction.

Learning of human-understandable representations for data and feature selection also provides another avenue for developing assurances  \cite{Bengio2013-uv, Guyon2003-fj}. For instance, \citet{Mikolov2013-lt} studied how to represent words and phrases in a vector space for natural language text learning; this enables simple vector operations for understanding word sense similarity and relative relationships learned from text corpora. For example, the vector addition operation $airlines+German$ yields similar entries that include $Lufthansa$. Such representations encodes knowledge that can be easily checked and understood by humans, and thus implicitly facilitate interaction and calibration of trust (see \cite{Haury2011-zi} for another example). The problem of discovering human understandable features and representations in more general settings still remains an open question. Currently, the main question for representation learning is how to find the `best representations' for a particular application -- not necessarily the representations and features that are `most humanly understandable'. This is not surprising, since human-understandable representations and features are not necessarily optimal for the criteria that AIAs are typically designed against. 

Contrary to the belief that interpretable models are necessarily worse performing than their less interpretable counterparts, several researchers have shown that this is not always the case (at least in the context of machine learning). However, the real trade-off is the amount of work that goes in to crafting the interpretable model from the start; these methods are often custom designed for certain tasks and are not easily transferable to other problems. Because of this, AIA designers must strike a balance between \emph{interpretable models}, \emph{explainable models}, and \emph{black-box models}.

\citet{Park2016-ld} point out that real interpretability in complex tasks still requires expert knowledge to make sense of complicated features; in essence: \emph{interpretable models require people}. For instance, \citet{Jovanovic2016-gw} use `Tree-Lasso' (TL) logistic regression with domain knowledge (i.e. medical diagnostic codes) to group similar conditions, and then use TL regression again on that information to develop a sparser model. \citet{Zycinski2012-jj} also use domain knowledge to structure a data matrix before feature selection and classification. See also \citet{Zhang2018-no,Khoa2018-gh} for other related examples. 
%
This kind of approach is also illustrated by those who use those who use `theory guided data science' (TGDS~\cite{Kumar2016-yw,Faghmous2014-og}). As one example \citet{Morrison2016-fz} address the situation where an imperfect analytical model is available for chemical reaction kinetics: the theoretical reaction equations are well known, but a `stochastic operator' is added on top of this to account for uncertainties and modeling errors. In adopting this approach the model becomes interpretable (to experts).

\subsubsection{Grounding Example:}
In the case of the `VIP Escort' problem (described in Section~\ref{sec:mot_example}), interpretable models might be used as an assurance in the following way:

We make the following assumptions

\begin{itemize}
    \item The UGV has just begun an attempt to escape the road-network
    \item The UGV is using a decision-tree for selecting different movements
    \item The operator is able to view the decision-tree model the UGV is using
\end{itemize}

While the operator is monitoring the progress of the UGV in its attempt to escape the road-network they are able to consult the decision-tree model. In this case the operator chose to consult the table when they saw the UGV make an unexpected turn at a given intersection. The operator identified the conditions that led to the decision and found that the UGV was not well equipped to execute the decision the operator thought was best.
\paragraph{\textbf{Discussion of Example:}} In this example the use of a decision-tree as a model enabled the operator to investigate unexpected behavior. During inspection they identified certain conditions that led to a decision, and they found that the UGV was not \emph{competent} to perform what the operator thought was a better decision. Because of this the operator better understood the decision the UGV made.

%%(i) assurance argument: what specifically is the assurance signal/rationale?  -- both value alignment and interpretable model approaches have the implicit aim of grounding user trust in an understanding of rational decision making processes; however, cognitive scientists and many AIA experts will point out, humans are not entirely rational actors. As such, what accounts for human-human trust?  Like interpretability, a precise definition is hard to pin down -- but perhaps essence can be best summed up as encoding AIA behaviors that avoid `inhuman-like' characteristics. This can be viewed as the algorithmic equivalent of avoiding the so-called `uncanny valley' in humanoid and social robotics and VR/AR domains. Indeed, the mere association of `human-like' behavior to an AIA can be enough to endow levels of trust in AIAs that are comparable to human-human relationships [Tripp, et al]. Basic idea then is to endow AIA with capabilities that can be interpreted by user as though they could have come from another human. 
%One example is provided by Wink Bennett's work for LVC-based simulation training of Air Force fighter pilots -- in the NotSoGrandChallenge, basic idea was to develop realistic enemy fighter plane AI that would keep trainee pilots honest, i.e. from finding loop holes in computer logic that would allow them to easily game the simulation and win predictably -- this requires not only intelligent mechanisms for adapting to different human trainees and exploring/exploiting their weaknesses through simulated combat interactions, but also requires AI to maintain some level of human credibility --- since trainees will eventually be up against real human fighter opponents, this application requires realistic human behavior for both trainers and trainees to trust the AIA is providing useful sim experience...
%%(ii) what is mechanism for generating appropriate TRBs? 
%Key idea here is to interpret/acknowledge the user's intent frame by producing suitable set of `expected' human behaviors. Assurances generated along these line tend towards improving an understanding of an AIA's predictability and situation normality, most often via verbal or non-verbal human interaction cues to express intent or uncertainty. This can be especially useful in complex problem settings where a utility-based or interpretable model-based understanding of the problem is insufficient for completely quantifying user preferences that can be used to inform `trustworthy' AIA responses (e.g. dog-fighting, or even highway driving [slowing down and coming to complete stop to yield right of way for pedestrians cross, using blinkers and horns to signal other drivers, etc.]...). This is also especially useful for interacting with users who are not privy to access `innards' of AIA, i.e. who cannot potentially act as all-seeing/all-knowing supervisors or operators, per se, but as in situ participants in task/environment that AIA must operate in... Caveat here is that users tend to fill in a lot of blanks on their own, so anthropomorphizing AIA can lead to other unintended consequences that are not present in value alignment or interpretable modeling -- namely, users might extrapolate other assurances or capabilities from AIA that it does not actually mean to signal or possess (e.g. eyes on quadcopter for signaling can't actually see anything...self-driving car cannot hear sounds or horns, even though it looks like a human-driven car and behaves as if it is being driven by a human...)
%%(iii) how can designers build/exploit for AIA assurances, i.e. what techniques available for human-like behavior?: wide variety of heuristics, but two salient categories pertaining to trust and assurances are 
%%(a) nonverbal communication of intent (e.g. physical gesturing, legible planning, turn signaling, etc.) 
%%(b) mannerisms: establishing implicit cues for interaction

\subsection{Human-Like Behavior} \label{sec:human_behavior}
In trying to enable AIAs to better participate in trust-relationships with human users, many have been inspired by the natural analog of human-human. Humans are used to forming trusting relationships with other humans. In essence they ask: what kinds of communication do humans use during interaction that helps them to understand and trust each other? This idea was investigated by \citet{Tripp2011-rx} who compared human trust in other humans against human trust in intelligent interactive technology, which in this case was represented by Microsoft Access, an intelligent recommendation assistant, and Facebook. They found that, as the technology becomes more `human-like', self-reported levels of trust in technology become more similar to levels of trust in other humans.
\brettcomm{picks up with (possibly) irrational behavior, not included in `value alignment'}

\citet{De_Visser2018-kd} specifically discusses different methods by which AIAs can be more human-like in order to `repair trust' with humans (here trust repair is roughly analogous to assurances, but focusing on re-building trust after it is lost). Among several other possibilities, they suggest that an AIA might repair trust by anthropomorphizing (responding using a human communication channel), or by explaining their actions. \brettcomm{Perhaps discuss other methods?}

\subsubsection{Common Approaches}
Generally, we do not have algorithms that describe how humans interact, and must settle for heuristics, or best attempts to create human-like behavior via algorithms. From a high level, there are two main ways that researchers have been addressing this challenge: Direct communication, and mannerisims.

\paragraph{Direct Communication:} \nisarcomm{this is a misnomer -- `nonverbal' communication is what you mean to say??}
In designing assurances that use direct communication, the communication can take many different forms. One popular approach is to use motion or gestures.\citet{Szafir2014-ok} investigated how to enable `Assisted Free Flyer' robots (quad-copters that are made to interact with humans in close spaces) to communicate by using gestures. In doing so they use `motion primitives' (a basic vocabulary of movements) that were inspired by animation \cite{Van_Breemen2004-rz}. In their evaluations of these primitives with human participants, they found that human users significantly found the AFFs to be more natural, and felt safer around them. Later \citet{Szafir2015-iy} also experimentally showed the effectiveness of using a quad-copter's `turn signal' lighting to help users more easily interpret the intended movements and actions. These works provide strong support for `natural communication' assurances aimed at predictability.

\citet{Dragan2013-wd} investigate `legible motion planning' (LMP). LMP is, in essence, planned robotic physical movements and gestures that, by themselves, convey intended actions and goals. They are meant to improve a user's ability to understand and predict where the robot is trying to move. Their insight is that legible motion is used by humans, and is important for situations in which a robot and person are collaboratively working in close proximity to each other. Similarly, in more recent work \citet{Kwon2018-xt} investigates calculating trajectories that convey `incapability', which is \emph{what} the AIA is trying to do, and \emph{why} it is unable to do so. \cite{Dragan2013-wd} calculated sub-optimal paths in order to communicate \emph{intent} of a gesture, by gathering data from human participants they were able to find the parameters for their cost function that best matched the expectations. Likewise \cite{Kwon2018-xt} tested several different objectives using user studies to guide the selection. See also \cite{Admoni2016-db} for related work.

As good example in the domain of natural language communication, at Google/IO 2018, `Google Duplex' (\cite{Google2018-eb}) was introduced through a demo where it called a business establishment to make a reservation. One of the striking characteristics of the system is that it was very difficult (if not impossible) to detect whether the Duplex voice was a human or not. This applied both for the words that is spoke and the accent of the voice. In order to achieve this impressive result, they trained a recurrent neural network (RNN) on anonymized phone conversation data.



\paragraph{Mannerisms:}
\citet{Salem2015-md} investigated the effects of autonomous task errors, task types, and `system personality' on cooperation and trust for humans who observed a domestic robot performing house tasks, such that the robot implicitly showed competence by its mannerisms and successes/failures during tasks. In this case the mannerisms and competency of the robot were completely under control and hard-coded into the system. Regardless, when participants were asked to cooperate with the robot on certain other tasks, the faulty operation of the robot was found to affect the self-reported trust levels of the participants.

\citet{Wu2016-ei} investigated how a person's decisions in a coin entrustment game are affected by their belief in whether they are competing against an AIA or another human player (which, unbeknownst to participants, was in fact an AI with some programmed human-like idiosyncrasies, e.g. variable wait times between turns). Trust in this context was measured directly by the number of coins a participant was willing to lose by putting them at risk to the other player. The experiment found that the participants trusted the AI opponent more than they trusted the `human' opponent; the authors suggest that this may be due to the perception that the AI opponent did not have feelings and operated in a more predictable and consistent `machine-like' way. Given that the `human' was an AI as well, this experiment illustrates that `machine-like' behavioral consistency can lead to implicit positive effects the trust of the participant in certain contexts.

Returning to the case of Google Duplex, another striking characteristic was that Duplex used mannerisims of speech such as saying `um\ldots', including pauses, and sometimes using shortened sentences, which made it even more difficult to distinguish between it and a real human. And during the demo calls, the human on the other end of the line was none the wiser, and trusted that they were in fact speaking to a human.
\brettcomm{Add something about etiquette?} \nisarcomm{this belongs in mannerisms}

\subsubsection{Grounding Example:}
In the case of the `VIP Escort' problem (described in Section~\ref{sec:mot_example}), human-like behavior might be used as an assurance in the following way:

We make the following assumptions

\begin{itemize}
    \item The UGV is about to begin an attempt at escaping the road-network
    \item The operator can observe all the actions of the UGV via video feeds at intersections
    \item The UGV has been designed with the ability to use gestures in order to indicate its `incapability' as in \cite{Kwon2018-xt}
\end{itemize}

As the UGV begins the escort problem, the human supervisor is monitoring progress. As the UGV reaches a certain intersection of the road network the supervisor expects the UGV to take a path $A$, but it does not. However, before choosing to take path $B$, the UGV made a movement that, to the operator, indicated that it considered attempting to traverse $A$. Due to the attempt the supervisor was able to surmise that the UGV wanted to take that path but couldn't due to some limitation.

\paragraph{\textbf{Discussion of Example:}} In this case the UGV is able to maintain appropriate trust of the supervisor because the supervisor was able to interpret the `gesture' that UGV was using. This highlights the assuring effects that human-like communication/behaviors can have on users.

\subsection{User Interaction} \label{sec:user_interaction}
To date one of the more common approaches to engender trust in users has been to put the users `in-the-loop'. This has been, and still is, modus operandi in the automation industry and others. While some think that more advanced AIAs will `soon' be able to operate with little human involvement, in general those who have more practical experience with AIAs are more reserved.

In this section we review some methods by which engineers have made AIA assurances by making the performance of the system highly dependent on the user's participation. This includes work from disciplines such as human-robot collaboration, cooperative control, cooperative sensing, and others. Humans playing a significant role in the functionality of an AIA is analogous to a supervisor working `in the trenches' with those they supervise; in doing so they are able to provide feedback in real-time, lend their expertise, and better appreciate the decisions and outcomes of the team's work.

\subsubsection{Common Approaches:}
\nisarcomm{what are the main ideas to cover beyond the individual papers? first talk about what *could* be done in terms of human roles (humans as planners/controllers, sensors/perception augments, tutors/trainers, etc.) -- then discuss how each approach contributes to acting as an assurance for an AIA, using refs as specific examples -- otherwise a laundry list of papers doesn't really help organize or convey any part of your argument here }\brettcomm{In what ways can users interact? make a list. Humans as sensors, humans as controllers,\ldots} \brettcomm{In what ways can users interact? make a list. Humans as sensors, humans as controllers,\ldots}

In general, a human and AIA can interact on many different levels. At the extreme of the most fundamental extremes, the human might fully replace a system or sub-system of the AIA. For example, the human might act as the sensors for an AIA. On the other extreme, the human might have a very weak involvement in the core functionality of the AIA. There are bodies of literature that address humans as sensors, and humans as controllers of AIAs. There is also research in which humans and robots \emph{share} responsibility for different functions.

Enabling the robot and human to share or `fuse' information can have an effect on trust. \citet{Sweet2016-dw} investigate how to enable using humans as `soft' sensors, and then fuse that information into that of the `hard' robot sensors in order to improve and augment the robot's Bayesian state estimation capabilities. They apply their approach in a scenario called `cops and robots' where a single `cop' robot tries to locate `robber' robots. In this case the human acts as a deputy that remotely interacts with the system. The human can see security camera footage of the building in which the cop is searching, and can offer natural language feedback to the cop robot when appropriate. If the human offers information it can be fused into the cop robot's estimation model, but in the meantime the cop robot operates autonomously without assistance. Similarly, \citet{Tse2015-tz} consider a framework for robots and humans to share and fuse information in a cooperative context.
Similarly \citet{Tse2015-tz} consider a framework for robots and humans to share and fuse information in a cooperative context. \nisarcomm{so...? there's a lot more to be said here...}

In order to be able to design a system that can be useful to a human operator, \citet{Kaupp2008-yr,Kaupp2005-pk} empirically identify the appropriate level of automation for a system while taking into account the amount of interaction required by a human operator. In this case the robot has sensors of its own, but can also ask for user input when the value of information (VOI) is high enough (i.e. is it worth asking a human for information given that there is a cost?); they define the threshold VOI by human trials before deployment of the system in order to optimize the involvement of the human user.

\citet{Tellex2014-uc} consider an autonomous assembly robot that can detect when it has failures (conditions that don't match expectations based on internal models). When this occurs the robot requests help from the human user to resolve the problem. In this way the human and robot are dependent on each other to accomplish a task. Since the user knows that, if needed, the robot will ask for help they can more appropriately trust that unknown problems won't occur without them being informed.

\citet{Freedy2007-sg} studied how mixed-initiative human-AIA teams might have their performance measured, and examined the extent to which such teams can only be successful if ``humans know how to appropriately trust and hence appropriately rely on the automation''. They explore this idea by using a tactical reconnaissance scenario where human participants supervised an unmanned ground vehicle (UGV)  platoon with three levels of autonomous targeting/firing capability (low, medium, high); these levels were dependent on the experimental conditions. The operator needed to monitor the UGV in case it couldn't perform as desired; in such cases the operator could intervene to resolve the problem. Operators were trained to recognize signs of task failure, and to only intervene if they thought the mission completion time would suffer.

\subsubsection{Grounding Example:}
In the case of the `VIP Escort' problem (described in Section~\ref{sec:mot_example}), operator interaction might be used as an assurance in the following way:

We make the following assumptions

\begin{itemize}
    \item The UGV has just begun an attempt to escape the road-network
    \item An interface system exists by which the operator can receive and provide information to the UGV
\end{itemize}

The UGV is capable of operating autonomously, but also has the ability to ask for assistance or information when necessary. In this way the functionality of the UGV can be greatly improved via interaction with the user. As the user interfaces with the UGV and is able to provide feedback and information about the best known location of the pursuer based on information unavailable to the UGV they have more trust in the competence, predictability, and situational normality of the UGV.

\paragraph{\textbf{Discussion of Example:}} In this scenario the user is more immersed in the functioning of the UGV. Not only are they able to respond to queries from the UGV, but they can also provide direct observations as well. Subsequently, the user feels more immersed in the functioning of the UGV and is more cognizant of appropriate TRBs.

%%(i) assurance argument: what specifically is the assurance signal/rationale? -- whereas user interaction techniques of previous section generally tend to provide integral assurances (i.e. designed as part of core functionality of AIA capabilities) that introspectively compensate for shortcomings in AIA capabilities, similar introspective assurances can also be generated to determine and inform users of competency limits/boundaries of AIA capabilities *without requiring* user interaction or intervention -- and thus can be considered as supplementary to or separate from core AIA functionality. That is:  assurances can be provided to introspectively determine and explain/interpret limits of what AIA can do or knows, etc. for the sake of the user, without requiring modification of underlying AIA design.  In this sense, self-assessments can provide users with insights on one or both of the following related concerns: what can the AIA actually do and what does it know? and, what is required of/by the AIA to actually do the assigned task?  The first concerns identifying set of tasks in `reach set' of AIA; the second concerns figuring out what would be needed to do current task [need to refine this...trying to distinguish between questions that lead to insights about competency and situation normality via complexity reduction, vs. insights that inform predictability via uncertainty]... but basic analogy [can be mapped to UGV] is subordinate/delegate telling supervisor what it is/is not capable of, vs telling supervisor what it would need to be able to carry out specific task at hand or what the possible outcomes would be for that specific task (so, self-assessment can have contrasting focus on AIA itself on general capabilities vs. on the task at hand in relation to how AIA would perform on it specifically)...
%%(ii) what is mechanism for generating appropriate TRBs? -- Key targets are competency and situational normality dimensions in terms of explaining AIA functionality, whereas predictability is key target for task oriented assurances. Assurances from self-assessment paradigms are typified by frameworks like `machine self-confidence'[] and explainable Bayesian inference[] -- these operate on the results of `black box' AIA component outcomes in a post hoc manner, unlike interpretable models discussed earlier (which force AIA components to be inherently `understandable' to users). Viewed differently: whereas interpretable models are more `bottom up', self-assessment is more of a `top down'/drill down process: latter is less constrained in choice of models/techniques for AIA capabilities, but also must rely on ability to suitably decompose these same AIA functions across different task contexts and 
%%(iii) how can designers build/exploit for AIA assurances, i.e. what techniques available for getting assurances from self-assessment?: 
%%(a) complexity reduction...  
%%(b) uncertainty...


\subsection{AIA Self-Assessment} \label{sec:aia_self_assessment}

The techniques of previous section generally tend to provide integral assurances (i.e. designed as part of core functionality of AIA capabilities) that are artifacts of interactive algorithms designed to compensate for shortcomings in AIA capabilities. This section focuses on introspective assurances that inform users of competency limits and boundaries of AIA capabilities without requiring user interaction, and that can generally be separated from core AIA functionality (i.e. without requiring modification of underlying AIA design).  These self-assessments can provide users with insights regarding either or both of the following related issues: (i) what information and tasks are actually within the AIA's reach?, and (ii) what is required by the AIA to actually do its assigned task? 
In contrast to user interaction techniques: the analogy here is of a subordinate telling a supervisor what she is/is not capable of, or telling the supervisor what she would need to be able to carry out specific task at hand to achieve a specific outcome, or what the possible outcomes actually would be for that specific task. 

%%Such assurances provide windows into the competency and situation normality via complexity reduction, vs. insights that inform predictability via uncertainty]... but basic analogy [can be mapped to UGV] is subordinate/delegate telling supervisor what it is/is not capable of, vs telling supervisor what it would need to be able to carry out specific task at hand or what the possible outcomes would be for that specific task

\subsubsection{Common Approaches:}
%\nisarcomm{Need to say a bit more about what the motivation/general idea here is, to continue flow from other previous sections...}
The literature in this section can be split into two high-level categories. 
The first set deals with how an AIA can algorithmically account for its uncertainties in its models of its task, environment, operating context, and capabilities. 
These kinds of assurances help inform the predictability and situation normality aspects of trust. 
The second set of methods attempt to algorithmically reduce complex `uninterpretable' models or processes that underlie AIA capabilities into more interpretable ones by providing explanations. 
Here the AIA makes an active attempt at processing data and making information available to the user to inform the competency aspect of trust. %%This is done in a post-hoc manner, or in a way such that the quantification of uncertainty is more supplemental, rather than integral, to the main functions of the AIA.

\paragraph{Quantify Uncertainty} \label{sec:QU}

Although active learning does not explicitly consider safety, the underlying approaches can be useful because active learners need to be able to search the problem space to reduce uncertainty; this requires an internal representation of uncertainty. The applications surveyed here are all mainly related to image classification and robotics. In the context of image classification, \citet{Paul2011-vr} introduced `perplexity' as a metric that represents uncertainty in predicting a single class and is used to select the `most perplexing' images for further learning. There have also been several attempts to use Gaussian processes (GPs) to actively learn and assign probabilistic classifications \cite{MacKay1992-sp,Triebel2016-kj,Triebel2013-ow,Triebel2013-ku,Grimmett2013-gj,Grimmett2016-yc,Berczi2015-rd,Dequaire2016-kh}. As with perplexity-based classifiers, the key insight is that if a classifier possesses a measure of uncertainty, then that uncertainty can be used for efficient instance searching, comparison, and learning, as well as reporting a measure of confidence to users. The key property of GPs to this end is their ability to produce output confidence/uncertainty estimates that grow more uncertain away from the training data. This information can be readily assessed and conveyed to users, even in high-dimensional problems. This property has also found much use in other AIA active learning problems, e.g. Bayesian optimization \cite{Snoek2012-tt, Brochu2010-tj,Israelsen2017-zb}. 

\citet{Choi2017-th} investigates how mixture density networks (MDNs)---neural networks that learn parameters of a Gaussian mixture distributions---can be used to help a controller switch modes based on the MDN's prediction of 

Bayesian neural networks (BNNs) are a method by which we can have insight into the uncertainty of a neural network model. Using BNNs \citet{Kendall2017-ry}, in the context of computer vision, also use deep BNNs to help visualize epistemic (input) and aleatoric (model) uncertainty for each pixel of an image. 

Similarly \citet{Kahn2017-vy} use deep BNNs to learn about the probability (with uncertainty) of an autonomous vehicle colliding in an environment given its current state, observations, and sequence of controls. Using this model they formulate a `velocity-dependent collision cost' that is used for model-based reinforcement learning. With this approach the vehicle naturally proceeds slowly when there is an elevated risk of collision. \brettcom{not sure if this goes here, or in the `value alignment' section\ldots it goes here if i downplay the built-in nature of the behavior, and instead focus on the ability to quantify uncertainty}

An AIA that can predict its performance on different tasks can provide assurances about competence, predictability, and the situational normality of a given task. Several authors have worked to improve this ability in visual classification \cite{Zhang2014-he,Gurau2016-hs,Churchill2015-ei,Kaipa2015-hy}. 
For example, to ensure that visual classifiers don't fail silently in novel scenarios, 
\citet{Zhang2014-he} learned models of errors on training images to predict errors on test images. 
\citet{Kaipa2015-hy} consider 3D visual classification of assembly line parts for robotic pick and place tasks, and develop statistical goodness-of-fit tests to estimate the likelihood that robots can use their sensors to find parts matching desired ones. %To accomplish this they apply the `Iterative Closest Point' (ICP) method, to match a point cloud measurement of the part with a ground-truth 3D model of the part. 
These approaches allow the AIA to assess capability and present appropriate assurances to users, though without any formal notions of trust. 

Models and logic are not trustworthy by themselves; they may be flawed to begin with, or become invalid when certain assumptions or specifications are violated. Thus, there is great interest in providing assurances that the models and assumptions underlying different AIA processes are in fact sound. \citet{Laskey1991-mf} -- with the intention of communicating model validity to users of `probability-based decision aids' -- notes that it is infeasible to perform a decision-theoretic calculation to determine if model revision is necessary. 
She presents a class of theoretically justified model revision indicators which are based on the idea of constructing a computationally simple alternate model and then initiating model revision if the likelihood ratio of alternate model becomes too large (see also \citet{Zagorecki2015-qy,Habbema1976-xd} --these ideas also provide a potential basis for the `model validity' machine self-confidence factors from Quadrant II).
\citet{Ghosh2016-dl}  present `model repair' and `data repair' strategies that can be used when the current model doesn't match the observed data, at which point the model and data can be repaired, and control actions can be replanned in order to conform with the formal method specifications. One challenge is how the `trustable' constraints should be identified, as this places a strong burden on the certifying authorities and system designer to foresee all possible failures.


\paragraph{Reduce Complexity} \label{sec:reduce_complexity}
Main ideas: 1) make explanation, 2) operate on black-box model, 3) styles of explanation, 4) 

\citet{Olah2018-rp} talk about `intepretability' but are really thinking about explanation, or making something interpretable.

\citet{Abdollahi2018-uw} talk about different kinds of explanations that can be made. `neighbor style', `influence style', and `keyword style'

\citet{Huang2017-lk} using `algorithmic teaching'~\cite{Balbach2009-jw} as inspiration. Algorithmic teaching involves having a model of a students learning algorithm, and then presenting training examples to allow the student to learn a target model. In this case the student is the human user, and the teacher is the robot that is trying to teach the human its own objective function by presenting a set of (optimal) training examples. We consider these training examples to be assurances herein.

\citet{Hayes2017-nt} put an abstraction with `communicable predicates' over the state space, and are able to explain controller policies in natural language.

In some cases it is desirable to maintain a complex, less interpretable model and then apply `first principles' to explain the results to the user. \citet{Lacave2002-cu} address this from the perspective of explaining probabilistic inference in Bayesian networks -- specifically, \emph{how} and \emph{why} a Bayesian network reaches a conclusion given some imputed evidence. 
They present three properties of explanation: 1) content (what to explain), 2) communication (how to explain), and 3) adaptation (how to adapt based on who the user is). %%It is not possible to cover all of the ideas that they present in their paper, but they are key to the idea of designing assurances. 
Several key points for designing assurances arise from considering the differences between explaining evidence (i.e. data), the model (i.e. the Bayesian network itself), or the reasoning (i.e. the inference process). 
%These are three key considerations in making assurances. 
Also important is whether an explanation is meant to be descriptive or aimed at ensuring comprehension, as well as whether explanations need to be on a macro or micro scale relative for parts of the Bayesian network (similar to globally/locally interpretable learned models \cite{Ruping2006-xj}). 
The authors also consider whether explanations should occur by two-way interaction between system and user, by natural language interaction, or by probabilities. Finally, considering adaptation, another key point for designing assurances in general applications and contexts is that not all users will require (or desire) the same kinds of assurances. This paper points out many challenges and considerations in designing assurances for probabilistic algorithms, and illustrates that %(as with the `No free lunch' theorem) 
there is no single `best' assurance that will address every possible situation. 
Other discussion regarding how probabilistic and statistical explanations can be presented is found in \cite{Rouse1986-dz,Wallace2001-fm,Kuhn1997-qc,Lomas2012-ie,Swartout1983-ko}; 
aforementioned works like \cite{Kuhn1997-qc} highlight the importance of framing effects and other cognitive biases for these methods. 



\subsubsection{Grounding Example:}
In the case of the `VIP Escort' problem (described in Section~\ref{sec:mot_example}), self-assessment might be used as an assurance in the following way, starting with the assumptions that:

\begin{itemize}
    \item The UGV is about to being an attempt to escape the road-network
    \item The UGV is using the `solver quality' metric mentioned by \citet{Aitken2016-fb}
    \item The operator has access to an interface where they can view the self-confidence metric calculated by the UGV
\end{itemize}

Before the UGV begins its attempt it is able to assess its `solver quality' given the specific, unseen road-network, based on similarities between the current network and ones that it has encountered before (i.e. problem features that are important to determining the quality of the approximate solution produced by the policy). The UGV reports that it has high confidence in its solver quality, and the operator is assured that they can trust the solver in this situation.

\paragraph{\textbf{Discussion of Example:}} In this case the UGV is able to assure the operator of the quality of the solver in the specific road-network. Generally, the UGV reduced what could be a very complex analysis into a simple format for the operator to interpret. This is in contrast to the operator viewing policies, models, algorithms, and complex probability distributions.

%%(i) assurance argument: what specifically is the assurance signal? 
%%(ii) what is mechanism for generating appropriate TRBs? 
%%(iii) how can designers build/exploit for AIA assurances, i.e. what techniques available for ...?: 
%%(a) ...  
%%(b) ...
%%(c) ...

\subsection{Information Visualization} \label{sec:vis_dr}
We define `information visualization' as the act of displaying information in such a way as to communicate to one of the trust dimensions of a human user. Specifically we consider the `competence', and `predictability' of the AIA and the `situational normality' of the task at hand.

\subsubsection{Common Approaches:}
\citet{Liu2017-xw} review several of the current methods that exist for visualizing ML models. They identified three main purposes for which visualizations are useful in this context: 1) understanding (why model behave how they do), 2) diagnosis (failures, or unexpected behavior), and 3) refinement (ability to improve performance). We focus on \emph{techniques} that assist in that process.

Two of the main tools in creating visualizations are and reducing the dimensionality, and treating uncertainty in creating the visualizations to assist users in understanding more easily.

\paragraph{Dimensionality Reduction:}
Dimensionality reduction (DR) is one of the key methods used in creating visualizations. \cite{Sacha2017-hf} identify seven different methods by which users interact with DR techniques. They use this to make the human-in-the-loop process model for interactive DR that is shown in Figure~\ref{fig:sacha_fig}.

\begin{figure}[htpb]
    \centering
    \includegraphics[width=0.9\linewidth]{Figures/dimred_framework.png}
    \caption{Human-in-the-loop process model proposed by \cite{Sacha2017-hf}. Included by permission.}
    \label{fig:sacha_fig}
\end{figure}

\citet{Venna2007-yj} discusses DR for ML and reviews many linear and non-linear projection methods. \citet{Vellido2012-nm} also discusses the importance of DR for making ML models interpretable. As one example, \citet{Chipman2005-om} applied this idea by constraining principle component analysis (PCA) in an attempt to make the resulting linear combinations of variables be more interpretable (more homogeneous, or more sparse).

At times a simple visualization is the most efficient way to communicate the results of decision making, planning. For example: \citet{Chadalavada2015-wx} enable a robot to project its path onto the ground so users can see.

\paragraph{Treatment of Uncertainty:}
In the previous section we have already visited the importance of an AIA being able to quantify its uncertainty. Visualization researchers are concerned with how to \emph{convey} that uncertainty to human users (and quantify uncertainty inherent in making visualizations). \cite{Sacha2016-tu} discuss how the propagation of uncertainty through visual analytics systems can affect the trust of human users (see also \cite{Correa2009-hi}).

One excellent example of this is the work by \citet{Wu2012-qi}, who create a tool to visualize the flow of uncertainty in the visualization process. In this way users can understand where uncertainty enters the visualization process.

\begin{figure}[htpb]
    \centering
    \includegraphics[width=0.6\linewidth]{example-image-golden}
    \caption{Example visualization of the flow of uncertainty in the creation of a visualization \cite{Wu2012-qi}. Included by permission.}
    \label{fig:hutchins_fig}
\end{figure}

The relationship between system uncertainty and the effects of uncertainty on the performance of the system can be very complex to understand. \citet{Hutchins2015-if} address this by using expert knowledge, and a `trust annunciator panel' (TAP) that has several `uncertainty level indicators' in order to display how uncertainties in sensors will effect the output quality, and the mission impact; and the same for the planning algorithm (see Figure~\ref{fig:hutchins_fig}).

\begin{figure}[htpb]
    \centering
    \includegraphics[width=0.9\linewidth]{Figures/Hutchins_fig.pdf}
    \caption{Proposed `trust annunciator panel' \cite{Hutchins2015-if}. Included by permission.}
    \label{fig:hutchins_fig}
\end{figure}

\subsubsection{Grounding Example:}
In the case of the `VIP Escort' problem (described in Section~\ref{sec:mot_example}), information visualization might be used as an assurance in the following way:

We make the following assumptions

\begin{itemize}
    \item The UGV has just begun an attempt to escape the road-network
    \item The user has access to an interface like that proposed in \cite{Hutchins2015-if}
\end{itemize}

During the attempt the user is able to see how the sensor uncertainty might possibly effect the outcome of the mission. In this case, the user is assured that the sensors will have little negative impact on the outcome of the mission given the current weather conditions.

\paragraph{\textbf{Discussion of Example:}} Here we see how a visualization is able to assist the user in correlating the effects between sensor uncertainty and mission outcome. This is not a simple relationship for operators (especially untrained) to learn on their own; even if they were able to learn the time required to do so can be very detrimental.

\subsection{User Assessment} \label{sec:user_assessment}
talk about these: \citet{Yu2018-qw, Wickens1999-la, Riley1996-qm, Muir1996-gt, Desai2012-rc}
\brettcomm{this is communicating data (unprocessed), rely on human to do processing (as opposed to vis and DR). human limitations on cognition.}
\subsubsection{Common Approaches:}

\paragraph{stuff:}


\subsubsection{Grounding Example:}
In the case of the `VIP Escort' problem (described in Section~\ref{sec:mot_example}), value alignment might be used as an assurance in the following way:

We make the following assumptions

\begin{itemize}
    \item The UGV has just begun an attempt to escape the road-network
    \item 
    \item 
\end{itemize}

\paragraph{\textbf{Discussion of Example:}} 

