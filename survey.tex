\section{Methodology} \label{sec:methodology}
    In this survey, I attempt to look at research of those who are formally and informally addressing the idea of human-AIA trust. In particular, I focus on a ideas that might be applicable to the trust relationship between a single human user (User), and a single autonomous vehicle. While theoretically a two-way trust model could be considered, I will only be considering a one-way trust relationship, that is that the autonomy has perfect trust towards a user.

    It should be noted that it is almost impossible to perform a comprehensive survey of all assurances due to the broad nature of assurances in general. One could rightly argue that metrics like gain and phase margins are assurances for control engineers, as are training and test accuracy for machine learning practioncioners. However, it is my opinion that the somewhat narrow view of the surveyed literature does not significantly hinder the definition or classification of assurances.

    In order to find applicable research I first looked at papers that formally addressed trust and tried to create models of it; this with the aim of trying to understand how it might be influenced. Secondly, I looked at some historical research regarding trust between humans and some form of non-human entity. This mainly lead to e-commerce literature, automation literature, and human-robot interactions. Third, I investigated work regarding `interpretable', `comprehensible', `transparent', `explainable', \ldots and other types of learning and modeling methods. Finally, I searched for research disciplines that are investigating methods that would be useful as assurances, but of which trust is not the main focus.

    With this information, I try to make an informed definition and classification of assurances based off of empirical information of methods that are currently in use or being investigated. In doing so I was able to identify several areas that are open for further research. 


\section{A Survey of Assurances} \label{sec:survey}
\edit{\textbf{RECOUNT CITED PAPERS IN EACH SECTION, AND UPDATE THE FIGURE}}
Now that AIA, trust, TRBs, and assurances have been defined the survey of assurances is now presented. There are many different ways in which this survey could be organized; we choose to present it based on the different goals of the main groups of researchers who have been working in efforts related to assurances.

Early in reading the related literature it became clear that there were two main groups: 1) those researchers who have formally addressed the topic of trust between humans and AIAs of some form, and 2) a much larger body of those who have informally considered trust in their work (or concepts related to trust). Here we consider formal treatment of trust to include those who acknowledge a human trust model and who gather data from human users in order to measure the effect of assurances on trust. Informal treatment of trust includes those who reference the concept and/or components of trust, but who do not gather user data to verify the effects of proposed assurances. 

Another way that the landscape of researchers might be divided is by the kinds of assurances they investigate. The first group consider what we call `implicit' assurances. Implicit assurances embody any assurances that are not deliberately designed into the AIA to influence trust or TRBs. The second group consider `explicit' assurances, which are assurances that were explicitly created by a designer with the intent of affecting a user's trust. Implicit assurances can be thought of as side-effects of the design process; for example HAL 9000 could have been designed with a circular `red-eye' looking sensor because it was cost-effective, however it is possible that users who interact with HAL might find the `red-eye' sensor to suspicious, and thus lose trust in HAL. Conversely, the same `red-eye' may have been explicitly designed and selected based on several studies that indicated that users find it easier to trust advanced AIAs with `red-eye' sensors instead of similarly shaped green sensors.

Much of the research that formally considers trust has focused on implicit assurances. This is likely due to the focus on investigating what properties of an autonomous systems can affect a user's trust. It is possible to argue that someone who finds that reliability affects a user's trust is investigating an explicit assurance, but for the purposes of this paper we try to stay true to the intent of the researcher when performing the work. More recently, as seen by a large spike in interest in `interpretable', and `explainable' AIAs in government, academic, and public circles, we have seen the emergence of groups who acknowledge that the concept of trust in human-AIA relationships, and who want to design systems accordingly.

In view of these four main groups of researchers, we organize the survey by creating four quadrants shown in Figure~\ref{fig:trust_assurance_intention}. In the remainder of this section we survey each of these quadrants separately in order to gain some understanding of the lessons that each has to offer when we consider the design of assurances.

\begin{figure}[htbp]
    \centering
    \includegraphics[width=0.8\textwidth]{Figures/Trust_vs_Assurance_Intention.pdf}
    \caption{Figure depicting how many of the papers surveyed here consider trust both formally and informally, as well as those who investigate explicit and implicit assurances}
    \label{fig:trust_assurance_intention}
\end{figure}

\begin{itemize}
    \item Quadrant I. (implicit assurances, formal trust treatment) -- Gather user data, consider a trust model, consider assurances that are implicit (i.e. those who care about human-AIA trust, but aren't designing assurance algorithms)
    \item Quadrant II. (explicit assurances, formal trust treatment) -- Gather user data, consider a trust model, consider assurances that are explicit (i.e. those who formally acknowledge human-AIA trust, and design assurances to affect it)
    \item Quadrant III. (explicit assurances, informal trust treatment) -- Do not gather data from users, reference trust (or its components interpretability, etc..), consider assurances that are explicit (i.e. those who know that the concept of `trust' is important, but that only use an informal notion of it when designing assurances)
    \item Quadrant IV. (implicit assurances, informal trust treatment) -- Not interested in affects on user trust, but reference (possibly only allude to) concepts that are related to trust as defined in this paper. Investigate approaches for creating AIAs with improved properties or characteristics. This work is subtly different from that in Quadrant III in the degree/intent to which trust concepts were considered. In Quadrant III trust components were clearly the main focus of the research, whereas in this quadrant the relationship to trust is only visible to someone who knows what they are looking for (i.e. those whose work is relevant for designing assurances, but don't know it)
\end{itemize}

\subsection{Quadrant I. (Implicit Assurances, Formal Trust Treatment)}\label{sec:q1}

Several experiments have formally shown the effect of implicit assurances (AIA features that were not purposefully designed to affect trust) on a user's trust towards an AIA. Generally, implicit assurances can affect any of the three trust dimensions highlighted in Fig. ~\ref{fig:Assurance_classes}. \citet{Muir1996-gt}, for example, showed that trained participants could learn to trust a fully autonomous simulated pasteurizer that operated with a consistently unreliable pump system. The participants could intervene as operators if they felt is was necessary to obtain better performance. The only explicit assurances afforded here to participants were visual performance indicators of the unreliable pump's flow rate and resulting tank liquid level over time, from which the participants implicitly created mental models of the system's reliability. 
Trust was quantified by self-reported questionnaire responses, based on the level of reliance on the automation during the simulation. 
This study attempted to identify the effects of reliability on the user's perception of the AIAs competence and predictability, but the results are difficult to generalize since only six participants were used for this particular system. 

Implicit assurances tied to perceptions of reliability built over contiguous observations of performance have also been experimentally investigated in other contexts. For instance, \citet{Wickens1999-la} investigated how the presentation of semi-reliable information by an automated airplane flight software system affects human pilot performance. Interestingly, while system reliability did have an effect on pilot performance, it did not make a measurable effect on the pilots' self-reported trust levels. In experiments comparing thirty university students and thirty-four professional pilots, \citet{Riley1996-qm} studied how reliability and workload affected the participant's likelihood of trusting in automation aids (with varying reliability levels) in tasks where participants had to classify characters and also perform a distraction task. It was found that the professional pilots (who had extensive experience working with automated systems) had a bias to use more automation, but reacted similarly to students in the face of dynamic reliability changes. Specifically both pilots and students, on average, responded quite quickly when the automation became unreliable, by turning it off, but a larger portion of pilots continued using the automation during the low-reliability portions of the experiment. However, in investigating human operator trust for an autonomous search and rescue task, \citet{Desai2012-rc} found that trust (as measured by the amount of time the human kept the robot engaged in fully manual teleoperation, semi-autonomous, or fully autonomous operating modes) was highly correlated to experimentally controlled robot reliability levels. 

The effects of perceived AIA `humanness' on trust have also been studied. 
Refs. \cite{Lankton2008-ct} and \cite{Tripp2011-rx} compared human trust in other humans against human trust in intelligent interactive technology, which in this case was represented by Microsoft Access, an intelligent recommendation assistant, and Facebook. 
They found that, as the technology becomes more `human-like', self-reported levels of trust in technology become more similar to levels of trust in other humans. 
\citet{Salem2015-md} investigated the effects of autonomous task errors, task types, and `system personality' on cooperation and trust for humans who observed a domestic robot performing house tasks, such that the robot implicitly showed competence by its mannerisms and successes/failures during tasks. When participants were asked to cooperate with the robot on certain other tasks, the faulty operation of the robot was found to affect self-reported trust levels (implicit reliability again), but did not have an effect on whether the human would cooperate on other tasks. %\hlr{WAS THIS BECAUSE HUMANS PERCEIVED THE ROBOT AS KNOWING ITS OWN LIMITATIONS, i.e. PERSONALITY/MANNERISMS?? i.e. more trustworthy because it admits it can't do something? therefore it must implicitly have some sort of `benign awareness' of its surroundings and circumstances, and therefore can be trusted?}  
%This seems to suggest that the effect of institutional trust (i.e. this robot may not be competent, but whoever designed it must have known what they were doing) allowed users to continue to cooperate with a faulty system, even if they have low levels of trust in it. \hlr{WAS THIS ACTUALLY A FACTOR??} 
In a less cooperative/more adversarial context, \citet{Wu2016-ei} investigated how a person's decisions in a coin entrustment game are affected by their belief in whether they are against an AI agent or another human player (which, unbeknownst to participants, was in fact an AI with some programmed human-like idiosyncrasies, e.g. variable wait times between turns). 
Trust in this context was measured directly by the number of coins a participant was willing to lose by putting them at risk to the other player. The experiment found that the participants trusted the AI opponent more than they trusted the `human' opponent; the authors suggest that this may be due to the perception that the AI opponent did not have feelings and operated in a more predictable and consistent `machine-like' way. 
%Given that the `human' was an AI as well, this experiment shows that `machine-like' behavioral consistency can lead to implicit positive effects the trust of the participant in certain contexts. 
%On the surface, their work is meant to be a study of the implicit differences in trustworthiness between humans and robots. 
%However in their experiment the `human' was actually an AI with some programmed idiosyncrasies to lead the human player to believe the AI was a human. 

Finally, several studies have examined how user's observations of AIA performance and the physical presence/embodiment implicitly influences trust. %%Note that `performance' is distinct from `reliability' as discussed above, although both notions are closely related. Performance pertains to implicit trust cues based on the actual outcomes of an AIA's execution of certain tasks and/or achievement of objectives (i.e. features that suggest that an AIA is `getting the job done'); reliability describes implicit trust cues based on the AIA's ability to continue functioning robustly and performing properly (i.e. features that suggest that an AIA will not `fall apart at the seams'). 
\citet{Freedy2007-sg} studied how mixed-initiative human-AIA teams might have their performance measured, and examined the extent to which such teams can only be successful if ``humans know how to appropriately trust and hence appropriately rely on the automation''. 
They explore this idea by using a tactical reconnaissance scenario where human participants supervised an unmanned ground vehicle (UGV)  platoon with three levels of autonomous targeting/firing capability (low, medium, high); these levels were dependent on the experimental conditions. The operator needed to monitor the UGV in case it couldn't perform as desired.
Operators were trained to recognize signs of task failure, and to only intervene if they thought the mission completion time would suffer. %Trust was formally acknowledged in this survey and was quantified by using the Relative Expected Loss (REL) measure, which is the mean number of trials to expected loss of robot control over $n$ experimental trial runs. 
Operators were more likely to trust a `medium' ability UGV if they had first encountered a `high' ability UGV, as opposed to encountering a `low' ability UGV first, which is another manifestation of framing effects like \cite{Riley1996-qm}. %\hlr{sounds like operators were given a choice of which ability level to use???} 
As in \cite{Muir1996-gt}, operators learned to trust a UGV with low competence as long as it behaved consistently. 
Related to physical presence, \citet{Bainbridge2011-pl} studied physical gesturing as a form of cooperative human-robot team communication.  It was found that trust (as measured by willingness of human to cooperate with the robot) was greater in cases where robots are physically present with human users, versus cases where robots were only displayed on screens. 
%%Trust was measured by the willingness of human participants to cooperate with the robot \hlr{what does `cooperate with the robot' mean -- cooperate in what sense?}. Among other interesting results, participants were more likely to cooperate with the robot when it was physically present. \citeauthor{Bainbridge2011-pl} suggest that this is due to increased trust. %, in this case cooperation is a TRB.
\subsubsection{Summary}
Implicit assurances have been formally considered vis-a-vis trust along the following lines:
\textit{perception of an AIA's reliability} (implicit cues on ability to function properly) -- even if an AIA is not reliable, it may behave consistently enough that a human user can still learn to trust it appropriately; 
\textit{perception of an AIA's capabilities} (implicit cues scope of tasks it can carry out and problems it can solve) -- this aspect has  been formally examined for relatively simple systems and tasking contexts, but deserves additional attention to better understand the dangers of potentially underselling/overselling the capabilities of AIAs via implicit associations, e.g. based on names/labels (`autopilots' for self-driving cars), appearance (humanoid robots that have trouble climbing stairs), etc.; 
\textit{perception of AIA's humanness and personality} (cues on ability to demonstrate human-like understanding and communication of tasks, context, environment, and itself) -- this aspect goes beyond the task space in that human qualities that normally affect human-human trust (intentionality, desire, feelings, fear, etc.) can be projected onto AIAs through (mis)interpreted implicit assurances; 
\textit{perception of AIA's performance and physical presence} (implicit cues on actual ability of AIA to execute tasks, communicate effectively with team mates for cooperation, and achieve goals, regardless of how reliably it functions) -- unsurprisingly, both performance and presence of an AIA are significant drivers of trust, but are not the only important drivers. 
% \begin{itemize}
% 	\item perception of an AIA's reliability (i.e. implicit cues on ability to function properly) -- even if an AIA is not reliable, it may behave consistently enough that a human user can still learn to trust it appropriately. 
% 	\item perception of an AIA's capabilities (i.e. implicit cues scope of tasks it can carry out and problems it can solve) -- this aspect has  been formally examined for relatively simple systems and tasking contexts, but deserves additional attention to better understand the dangers of potentially underselling/overselling the capabilities of AIAs via implicit associations, e.g. based on names/labels (`autopilots' for self-driving cars), appearance (humanoid robots that have trouble climbing stairs), etc. 
% 	\item perception of AIA's humanness and personality (i.e. cues on ability to demonstrate human-like understanding and communication of tasks, context, environment, and itself) -- this aspect goes beyond the task space in that human qualities that normally affect human-human trust (intentionality, desire, feelings, fear, etc.) can be projected onto AIAs through (mis)interpreted implicit assurances.
% 	\item perception of AIA's performance and physical presence (i.e. implicit cues on actual ability of AIA to execute tasks, communicate effectively with team mates for cooperation, and achieve goals, regardless of how reliably it functions) -- unsurprisingly, both performance and presence of an AIA are significant drivers of trust, but are not the only important drivers.  
% \end{itemize}

The range of different trust measures used by the works in this quadrant underscores the point that trust-related behaviors (TRBs) ought to be the focus of assurances as opposed to trust itself, since trust is a subjective measure that may (if defined properly) or may not (if not properly defined) predict whether a person uses an AIA appropriately. 
This not only depends heavily on how one `measures trust', but also depends on individual user traits. 
Throughout this paper, we will see evidence for the idea that a user will gather and respond to assurances, whether implicit or explicit, in order to execute TRBs. 
A key takeaway from papers in Quadrant I is that, in the absence of explicitly provided assurances (i.e. if only implicit assurances are available), a user will use other perceived properties, cues and behaviors to gather assurances and inform their TRBs. 
Most of the implicit assurances considered above largely come in the form of visual cues, though other cues like waiting times or spoken language features (e.g. tone, volume, pace, etc.) can also play roles. 
It is thus also important to consider the role of human cognitive tendencies and limitations when assessing what implicit assurances are present in an AIA. %%\hlr{NRA note: recall your definition of implicit assurance: it is an assurance that is NOT something designed into the system on purpose, but it still affects trust outcomes...}. 
This was directly observed in \cite{Freedy2007-sg,Riley1996-qm}, where framing effects biased operator's behaviors towards the AIAs. 
Other cognitive biases such as `recency effects' (being biased based on recent experience), `focusing effects' (being biased based on a single aspect of an event), or `normalcy biases' (refusal to consider situations which have never occurred before) are also important to consider. 
%
%%Finally, humans naturally attempt to construct statistical models (albeit not especially accurate ones) of the world around them in order to predict and operate within it. %%In light of this AIA designers must also consider the limitation (time and otherwise) for human users to build statistical models, e.g. for reliability, when only instance by instance data can be gathered by users.
%%Even if an AIA has the ability to calculate an assurance, it must still have a way by which to express that assurance to a human user. The human user then perceives the assurance, perhaps through interaction with the system, or only through more passive observation of the system. These kinds of perceptions can be based on displayed information, or on how the AIA `behaves' (as in \cite{Salem2015-md}). Once the user perceives some kind of assurances (perhaps not purposefully communicated), those assurances are integrated into the trust of the user towards the AIA. In most cases this group of research focuses on assurances given through visual cues.

%We also see evidence that human cognitive limitations need to be taken into account when designing assurances. This was directly observed in \cite{Freedy2007-sg,Riley1996-qm} where framing effects biased operator's behaviors towards the AIAs. This also suggests that other cognitive biases such as `recency effects' (being biased based on recent experience), and others will also apply as well. A couple of examples include well known cognitive biases such as `focusing effects' (being biased based on a single aspect of an event), or the `normalcy bias' (refusal to consider a situation which has never occurred before). Finally, humans naturally attempt to construct statistical models (albeit not especially accurate ones) of the world around them in order to predict and operate within it. In light of this designers must also consider the limitation (time and otherwise) for humans to build statistical models, such as reliability, when only instance by instance data is available. %These ideas still remain to be verified by further, more focused, experimentation.


%%%OLD
%\citet{Muir1996-gt} performed an experiment where participants were trained to operate a simulated pasteurizer plant. During operation they were able to intervene in the fully-autonomous system if they felt is was necessary to obtain better performance. Trust was quantified by self-reported questionnaire responses, as well as by the level of reliance on the automation during the simulation. She noted that operators could learn to trust an unreliable system if it was consistent. The participants were only able to observe the reliability of the pump (i.e. the performance of the pumps over time, from which the user created a mental model of reliability).

%In experiments involving thirty students and thirty-four professional pilots, \citet{Riley1996-qm} investigated how reliability and workload affected the participant's likelihood of trusting in automation. Two simulated environments were created to this end. First was for participants to use/not use an automated aid (with variable reliability) to classify characters while also performing a distraction task. Interestingly, they found that pilots (those with extensive experience working with automated systems) had a bias to use more automation, but reacted similarly to students in the face of dynamic reliability changes. In this setting, the bias to use more automation would be known as `framing effects' (where a human's trust is biased by the trust they have in previously encountered systems) in cognitive science. Findings also showed that the use of automation is highly based on individual traits.

%Also considering the performance of pilots, \citet{Wickens1999-la} investigated the effect of semi-reliable data while piloting a plane. They also investigated semi-reliable performance of a system that highlighted important data for the pilot to see. The pilots were aware that the measurements/highlighting system might be inaccurate before the experiment. The reliability of the systems did have an effect on the outcome of the experiment, but interestingly did not make a measurable effect on the pilot's self-reported trust. This underscores the point that TRBs ought to be the focus of assurances as opposed to trust itself, since trust is a subjective measure that may or may not actually change a person's TRBs.

%%hlr{Cutting out this reference, since McKnight is not talking about AIAs -- Excel doesn't count!!}
%McKnight and collaborators have spent significant time investigating trust between humans and technology. His initial research was focused on e-commerce settings but later moved to trust between humans and technology. In \cite{Mcknight2011-gv} they gather self-reported trust through a questionnaire. Their experiment was interested in identifying the dimensions of trust effected by learning to use Excel for use in a business class. The results were based solely on the intrinsic properties of Excel and how each individual perceived them.


%\citet{Freedy2007-sg} studied how `mixed-initiative teams' (MITs, their term for human-robot teams) might have their performance measured. The premise of the work is that MITs can only be successful if ``humans know how to appropriately trust and hence appropriately rely on the automation''. They explore this idea by using a tactical game where human participants supervised an unmanned ground vehicle (UGV) as part of a reconnaissance platoon. UGVs had three levels of capability (low, medium, high), and had autonomous targeting and firing capability which the operator needed to monitor in case the UGV could not perform as desired. Operators were trained to recognize signs of failure, and to only intervene if they thought the mission completion time would suffer. Trust was formally acknowledged in this survey and was quantified by using Relative Expected Loss (REL), which is the mean expected loss of robot control over $n$ trial runs. Operators were found to be more likely to use a `medium' ability UGV if they had first encountered a `high' ability UGV, as opposed to encountering a `low' ability UGV first, which is another manifestation of framing effects like \cite{Riley1996-qm}. Similar to \cite{Muir1996-gt} the operators learned to trust a UGV with low competence as long as it behaved consistently.

%In a similar vein \citet{Desai2012-rc} investigated the effects of robot reliability on the trust of human operators. In this case a human participant needed to work with an autonomous robot to search for victims in a building, while avoiding obstacles. The operator had the ability to switch the robot from manual (teleoperated) mode, to semi-autonomous, or autonomous mode depending on how they thought they could trust the system to perform. During this experiment the reliability of the robot was changed in order to observe the effects on the operator's reliance to the robot. Trust was measured by the amount of time the robot spent in different levels of autonomy (i.e. manual vs. autonomous), and it was found that trust changed based on the levels of reliability of the robot.

%\citet{Salem2015-md} investigated the effects of error, task type, and personality on cooperation and trust between a human and robot. In this case the AIA was a domestic robot that performed tasks around the house. A human guest was welcomed to the home and observed the robot operating on different tasks. After this observation (in which the robot implicitly showed competence by its mannerisms and successes/failures) the human participant was asked to cooperate with the robot on certain tasks. Interestingly, it was found that self-reported trust was affected by faulty operation of the robot, but that it didn't seem to have an effect on whether the human would cooperate on other tasks. This seems to suggest that the effect of institutional trust (i.e. this robot may not be competent, but whoever designed it must have known what they were doing) allowed users to continue to cooperate with a faulty system, even if they have low levels of trust in it.

%\citet{Wu2016-ei} use game theory to investigate whether a person's decisions are affected by whether they believe they are playing a game against a human or an AI. This idea was studied in the context of a coin entrustment game, in which trust is measured by the number of coins a participant is willing to lose by putting them at risk of the other player. On the surface, their work is meant to be a study of the implicit differences in trustworthiness between humans and robots; however in their experiment the `human' was actually an AI with some programmed idiosyncrasies to lead the human player to believe the AI was a human. This was done by adding a random wait time, as opposed to an instantaneous move that the AI would make. There were also prompts at the beginning of the `human' version of the experiment that suggested that the participant was waiting for another human player to join the game. The experiment found that humans trust an AI more than they trust a `human'. The authors suggest that this may be due to the perception that an AI does not have feelings and is operating in a more predictable way. Given that the `human' was an algorithm as well, this experiment shows that consistency (i.e. no variable wait times) was a factor that affected the trust of the participant.

%\citet{Bainbridge2011-pl} investigated the difference in trust between a human and a robot, in cases where the robot was physically present and where the robot was only displayed on a screen (i.e. not physically present). In this experiment, the only method of communication from the robot was through physical gestures. Trust was measured by the willingness of the human participants to cooperate with the robot. Among other interesting findings regarding how the participants interacted with the robot, it was found that participants were more likely to cooperate with the robot when it was physically present. \citeauthor{Bainbridge2011-pl} suggest that this is due to increased trust, in this case cooperation is a TRB.

%%\hlr{taking this one out also: not really looking at implicit assurances at all here -- simply a survey that asks whether or not people trust the car -- would be different if users also given pictures or some other description/'feel' for an actual car...}
%With the aim of understanding how individuals currently trust autonomous vehicles, \citet{Munjal_Desai2009-en} performed a survey of around 175 participants. The participants were asked to rate their level of comfort with six different situations. These situations ranged from parking your own car, having an autonomous vehicle with manual override park the car, and having a fully autonomous vehicle that could not be overridden park the car. There were also questions related to user comfort with autonomy in situations where they still retained control, like how comfortable users would be with having autonomous vehicles park near their car. The survey found that the participants were most comfortable with parking their own car, and least comfortable with having a fully autonomous vehicle (with no manual override) park their car. These findings are supposedly related to institutional trust, as those surveyed did not necessarily have any experience with autonomous vehicles.

%\subsubsection{Summary}
%We see that there have been several experiments that have formally shown the effect of implicit assurances (assurances that were not purposefully designed to affect trust) on a user's trust towards an AIA. Generally, implicit assurances can affect any of the three trust dimensions highlighted in Figure~\ref{fig:Assurance_classes}. To use a practical example recall that reliability (or rather the perception of reliability built over contiguous observations of performance) was frequently investigated as an assurance in this section. A reliable AIA can seem more competent, and predictable to a human user. Currently, it isn't very clear how different assurances affect the trust dimensions. \Citet{Muir1996-gt} attempted to identify the effects of reliability on the user's perception of the AIAs competence and predictability, but only six participants were used, so the results are questionable. Quantifying the effects of assurances on different dimensions of a user's trust is still an open research question.

%Throughout this work we see evidence for the idea that a user will gather assurances, whether these are implicit or explicit, in order to execute TRBs. To restate the point, in the absence of explicitly provided assurances (such as investigated by every research paper in this quadrant) a user will still use other perceived properties and behaviors and gather assurances in order to inform their trust-related behaviors.
%
%Even if an AIA has the ability to calculate an assurance, it must still have a way by which to express that assurance to a human user. The human user then perceives the assurance, perhaps through interaction with the system, or only through more passive observation of the system. These kinds of perceptions can be based on displayed information, or on how the AIA `behaves' (as in \cite{Salem2015-md}). Once the user perceives some kind of assurances (perhaps not purposefully communicated), those assurances are integrated into the trust of the user towards the AIA. In most cases this group of research focuses on assurances given through visual cues.
%
%We also see evidence that human cognitive limitations need to be taken into account when designing assurances. This was directly observed in \cite{Freedy2007-sg,Riley1996-qm} where framing effects biased operator's behaviors towards the AIAs. This also suggests that other cognitive biases such as `recency effects' (being biased based on recent experience), and others will also apply as well. A couple of examples include well known cognitive biases such as `focusing effects' (being biased based on a single aspect of an event), or the `normalcy bias' (refusal to consider a situation which has never occurred before). Finally, humans naturally attempt to construct statistical models (albeit not especially accurate ones) of the world around them in order to predict and operate within it. In light of this designers must also consider the limitation (time and otherwise) for humans to build statistical models, such as reliability, when only instance by instance data is available. These ideas still remain to be verified by further, more focused, experimentation.
%
% Finally, we are able to begin to develop ideas regarding what indicates changes in trust, and how those changes should be measured. Specifically: through measuring TRBs, and through self-reported changes in perceived trustworthiness. Implicit assurances are not targeted at specific trust dimensions, this is by definition, because they weren't designed to affect trust at all. It seems, that except in a very controlled environment, that they originate from several different sources of AIA capabilities at once, as they are only based on a user's perception and would be unconsciously combined into an overall `sense' of trustworthiness.


\subsection{Quadrant II. (Explicit Assurances, Formal Trust Treatment)}\label{sec:q2}
One of the first considerations in the study of human-automation trust relationships was the role of human perception. \citet{Sheridan1984-kx} was interested in how human users might understand how users might understand a supervisory control system in a petroleum refinery, and suggested that information be made more `transparent' to the system designers. \citet{Muir1987-mk,Muir1994-ow} considered similar questions and concluded that making systems more `observable' is one way to improve the trust between a human operator and an autonomous control system. The critical question highlighted by this research is: how can a user trust an AIA that they cannot understand in some way? Furthermore, how can a user's trust be affected unless they can perceive the assurances being produced by the AIA?

Also with the aim of improving the user's ability to understand how the AIA functions, \citet{Aitken2016-fb} and \citet{Aitken2016-cv} propose a metric called `self-confidence' that is an assurance for a UGV that is using a POMDP planner (in this case, for the road network application described earlier in Section \ref{sec:mot_example}). This metric is made of a combination of five component assurances: 1) Model Validity, 2) Expected Outcome Assessment, 3) Solver Quality, 4) Interpretation of User Commands, and 5) Past Performance. The components considered are fairly general, and applicable to most planners, but would require new algorithms to be designed for those methods. Model validity attempts to quantify the validity of a model within the current situation. The expected outcome assessment uses the distribution over rewards to indicate how beneficial or detrimental the outcome is likely to be. Solver quality seeks to quantify how well a specific POMDP solver is likely to perform in the given problem setting (i.e. how precise the solution can be given a POMDP description). The interpretation of commands component is meant to quantify how well the objective has been interpreted (i.e. how sure am I that I was commanded to move forward). Finally past performance, is meant to add in empirical experience from past missions, in order to make up for theoretical oversights.

Self-confidence is reported as a single value between $-1$ (complete lack of confidence), and $1$ (complete confidence). A self-confidence value of $0$ reflects total uncertainty. Each of the component assurances would be useful on its own, but the composite `sum' of each factor is meant to distill the information from the five different areas, so that a (possibly novice) user can quickly and easily evaluate the ability of the robot to perform in a given situation. Currently, only one of the five algorithms (Expected Outcome Assessment) has been developed, but there is continuing work on the other metrics. As of the writing of this document, no human experiments have been performed to validate the usefulness of the self-confidence metric. 

Another important consideration is detecting appropriate (and inappropriate) use of the AIA, and then calibrating it. This is something mentioned by \cite{Muir1994-ow,Kaniarasu2013-ho}. Specifically, \citet{Kaniarasu2013-ho} examined whether or not misuse and disuse (over and under trust, in their terminology) can be detected and then calibrated to be within the AIA's competence. They use data from an experiment with a robot that had a confidence indicator displayed through a user interface. A user was asked to provide trust feedback every twenty seconds while operating a robot along a specified path without hitting obstacles and responding to secondary tasks. The user trust was quantified by measuring how frequently they switched between partially autonomous and fully autonomous modes. The experiment found that the indicators of confidence did in fact reduce misuse and disuse. However, we note that at this point in time most robots are not equipped to `know' how reliable they are. This highlights a common shortcoming in the design of current robots: that they cannot quantify their own reliability.

Several researchers have attempted to make interaction between humans and AIAs more `natural'. This is done is several different ways by the literature surveyed here. For example, \citet{Dragan2013-wd} investigated how a robot could move in a `legible' way, or in other words, make movements that in themselves convey the intended destination. This kind of motion is used by humans, and is important for situations in which a robot and person are collaboratively working in close proximity to each other. They investigate this within the context of how quickly a human participant is able to predict the goal of a moving point, before the point actually reaches that goal. They found that legible motion does in fact improve the user's ability to understand and predict where the robot is trying to move. This work focuses mainly on the premise that some deviation from the most cost optimal trajectory makes motion more legible; this idea can be extended in certain situations where humans rely on redundant or non-optimal behavior to predict outcomes (contrast this with \citet{Wu2016-ei} from Section~\ref{sec:q1}).

Another example is the work of \citet{Chadalavada2015-wx} who investigated ways to make interaction between humans and robots more natural (i.e. being more predictable and using common human methods for communicating especially non-verbal communication) in settings where they occupy the same work space. Their approach was to have the robot project its intended movements onto the ground in order to indicate where it would be moving. In their experiments there was a significant improvement in all measures when the robot was projecting its intended movements. This is strong evidence for an assurance aimed at predictability. Similarly, in \citet{Szafir2014-ok,Szafir2015-iy} they investigated using a quad-copter's motion patterns, as well as signaling with light, to help users more easily interpret the intended movements and actions.

It is natural for humans to explain reasoning, and actions, when interacting with other humans. This idea is used by \citet{Wang2016-id} in their experiment in which a human and robot investigate a town. The robot in their experiment has sensors that detect danger, and will recommend that the human wear protective gear if it senses danger. However, the human can choose to not accept the recommendation based on their trust in the robot, and the need to avoid the delay associated with using the protective gear. The robot is able to pose natural-language explanations for its recommendations, as well as report its ability. This was done by creating explanations generated by translating the components of the robot's planning model (the robot uses a partially observable Markov decision process (POMDP) planner) to natural language. Examples of explanations include: ``I think the doctor's office is safe'', or ``My sensors have detected traces of dangerous chemicals'', or ``My image processing will fail to detect armed gunmen 30\% of the time''. During the experiment they used two levels of robot capability: `high', and `low'; and three levels of explanations: `confidence level', `observation', and `none'. They found that when the robot's ability was low, the explanations helped build trust. Generally, confidence level, and observation explanations had a similar effect on trust and both were an improvement over no explanations. Whereas, when the capability was high, the explanations didn't have a significant effect. This suggests that in some cases some kinds of assurances are overridden by other `stronger' ones, specifically the explanations (explicitly designed for affecting trust) were rendered useless by the high reliability (which can be thought of as an implicit assurance in this case) of the AIA.

\citet{Chen2014-dk} lay out a framework for agent transparency based on formal trust models. This work is in the setting of situational awareness (SA, \cite{Endsley1995-ie}) of an autonomous squad member in a military operation. The aim is to make the mission SA and agent more transparent, so that the user will be able to trust the agent and use its assistance. They propose explicit feedback that can support the three levels of an operator's SA. They call their model the SA-based Agent Transparency (SAT) model. The first level -- Basic Information -- includes understanding the purpose, process, and performance of the system. The second level -- Rationale -- includes being able to understand the reasoning process of the system. The third level -- Outcomes -- involves being aware of future projections and potential limitations of the system. They suggest that two main components are display of information, and the display of uncertainty, but note that there are many considerations to take into account when trying to communicate information to human users (for example, numerical probabilities can be confusing and may need to be replaced by confidence limits or ranges). This work indicates the importance of different levels of assurances; in some situations only a certain subset of information will be needed. A key limitation is that, generally, not all of the elements of the SAT framework are available to AIAs at this time, this can be due to design limitations as well as theoretical limitations (i.e. no method might exist to quantify the future performance of a model in an environment in which it has never been deployed).

\subsubsection{Summary}
Explicit assurances have been formally considered vis-a-vis trust in the following ways:

\begin{itemize}
    \item Making internal functions of an AIA observable to a human user (i.e. explicit cues helping users understand different AIA capabilities) -- this includes calculating appropriate values to help a human understand, as well as how to present them to the user. Assurances can only be effective if a user can perceive them.
    \item Quantifying trust (i.e. being able to measure the effects of different assurances) -- in order to ensure that trust is being affected one must be able to measure it. Not only does this aid in verifying whether certain assurances function correctly, but also identifying certain behaviors/characteristics of specific users.
    \item Making interaction with an AIA more natural (i.e. explicit cues that compliment a human's existing capability for interacting in trust relationships) -- in essence, designing AIAs to utilize the existing interaction capabilities of humans, instead of relying on human users to develop new capabilities specialized for interactions with AIAs. This also includes concerns regarding \emph{how} an assurance can be calculated, and not only \emph{what} should be communicated.
    \item Considering different levels of trust -- there are different levels of trust, and by extension assurances. Sometimes more basic information such as understanding the AIAs purpose is adequate for the level of trust necessary. At other times information about the AIAs rationale, or the outcomes is necessary for properly affecting trust.
\end{itemize}

Using explicit assurances is, unknowingly to some, an attempt to directly influence the dimensions of trust from Figure~\ref{fig:Assurance_classes}. Understanding how assurances affect the user's ideas about the `competence', `predictability', and `situational normality' of an AIA in a certain situation is an important consideration, but perhaps only in certain situations. Here researchers used metrics for tracking changes in the TRBs of a human user in lieu of attempting to understand the self-reported level of trust of the user, for example by measuring the frequency of a user switching control from autonomous to semi-autonomous. Further consideration highlights the point that -- unless the goal of the AIA is to affect the user's self-reported trust -- TRBs are really the most appropriate metric that can be used. This is an idea that began in Quadrant I, but that became more clear in this quadrant.

\subsection{Quadrant III. (Explicit Assurances, Informal Trust Treatment)}\label{sec:q3}
    An AIA that has the capability of predicting its own performance on different tasks can use that to communicate with humans about competence, predictability, and the situational normality of a given task. Several authors have worked to improve this ability in visual classification \cite{Zhang2014-he,Gurau2016-hs,Churchill2015-ei,Kaipa2015-hy}. Some examples from that literature include \citet{Zhang2014-he} who approach the problem by, in essence, learning a model of the training error associated with different images and then using this to predict the error on test images. This ensures that the classifier won't fail silently (i.e. without warning). Also, learning from input data makes their method applicable to any classifier. Also, \citet{Kaipa2015-hy} approaches the problem more from the perspective of estimating the likelihood of a sensed part matching the desired one; this within the scenario of a robot picking parts from a bin. To accomplish this they apply the `Iterative Closest Point' (ICP) method, to match a point cloud measurement of the part with a ground-truth 3d model of the part. Each of these approaches enables the AIA to quantify its capability, and do then present appropriate assurances to the human user.

    It is also important to consider the intrinsic limitations of the AIA in predicting performance. To this end \citet{Kuter2015-qh} propose that an AIA be able to calculate the stability of its planner. In essence they ask: how sensitive is the plan to uncertainties? They use `counter planning' and `plan repair' to enable the autonomous system to identify likely contingencies that might interfere with an existing plan and then adapt the plan to account for those contingencies. Similarly, \citet{Hutchins2015-if} investigate the inherent `competency boundaries' of an autonomous systems' components (i.e. sensors, actuators, planners). The competency of each component can change in different situations such as sunny/stormy weather, or physical surroundings (i.e. variation of GPS accuracy). If these competency boundaries could be quantified (currently by an expert using a Likert scale) then performance in different situations could be more accurately understood (further, they consider \emph{how} this information might be communicated to a user).

    A well-known concept in machine learning and pattern recognition is the trade-off between accuracy and interpretability, or that generally improving the accuracy of a certain model reduces the interpretability. \citet{Ruping2006-xj} specifically asks how classification results can be made more interpretable to those who design and use classifiers. To address the accuracy-interpretability trade-off he investigates the use of a simpler global model, and a more complex local model (\citet{Otte2013-oo} and \citet{Ribeiro2016-uc} implement similar ideas as well). Figure \ref{fig:ruping} illustrates this idea on a simple example, the left (global) model can be used as a large scale interpretable model, while the more complex local model (shown on the right) can be used as necessary when more precise understanding is required. While his focus was on classification, the methods could also be useful in regression as well.

    \begin{figure}[htbp]
    \centering
    \includegraphics[width=0.8\textwidth]{Figures/global_local}
    \caption{Example of simple global model on the left, and on the right a more complex local model that can be used for interpretability when more detail is necessary.}
    \label{fig:ruping}
    \end{figure}

    Given a model \citet{Van_Belle2013-ph} suggest that there are three methods that help ascertain the level of interpretability and potential utility of models: 1) Map to domain knowledge, 2) Ensure safe operation across the full operational range of model inputs, and 3) Accurately model non-linear effects (compare to categories proposed by \citet{Lipton2016-ug}). They identify certain strengths and weaknesses of different techniques, and finally conclude that there is no method that is clearly the best in all situations. \citet{Huysmans2011-th} experimented with decision trees, decision tables, propositional if-then rules, and oblique rule sets in order to understand which is the most interpretable method. Through experimentation they found decision trees and tables to be more interpretable, but note that each method has applications in which it can perform better than other methods.
    
    While the work of \citeauthor{Huysmans2011-th} gives some indication about how to choose classifiers for interpretability, \citeauthor{Park2016-ld}, who investigated and proposed methods for making interpretable rules in decision trees, points out that to gain \emph{real} interpretability in complex tasks expert knowledge is still needed to make sense of complicated features. As two examples \citet{Jovanovic2016-gw} use `Tree-Lasso' logistic regression with domain knowledge. Specifically they use medical diagnostic codes to group similar conditions and then use `Tree-Lasso' regression that uses that information to make a more sparse model. \citet{Zycinski2012-jj} also use domain knowledge to structure a data matrix before feature selection and classification. 
    
    \citet{Faghmous2014-og} argues that interpretable models are necessary in data science when studying scientific phenomena such as environmental effects. They propose using `theory guided data science' (TGDS) that provides theoretical, interpretable, structure to data science problems. As one example of TGDS \citet{Morrison2016-fz} address the situation where an imperfect analytical model is available. They use a chemical kinetics application where the theoretical reaction equations are well known, and then add a `stochastic operator' over the top of the known model to account for uncertainties.
    
    There has been numerous other research efforts towards making models more interpretable. Generally the methods presented compete with (or excede) their state-of-the-art, less interpretable, counterparts. \citet{Abdollahi2016-vn} investigate making collaborative filtering models more interpretable by using a conditional restricted Boltzmann machine (RBM). \citet{Ridgeway1998-lv} use `weight of evidence' (WoE) as a boosting method that is more amenable to interpretation, and show that WoE is on par with AdaBoost. \citet{Choi2016-by} construct a recursive attention neural network to remove recurrence on the hidden state vector, and instead add recurrence on the visits of patients to doctors, as well as on different diagnoses during those visits. In this way the model is able to predict possible diagnoses in time, and a visualization can be that that indicates the critical visits and diagnoses that lead to that prediction.  

    One of the simplest approaches to helping users understand AIAs is to display the raw data being used, and rely on the human's processing power to make conclusions. In most real applications, however, there are too many individual variables for a human to attend to. Beyond that the human may have to be highly trained to interpret the results. In such situations, dimensionality reduction (DR) and visualization are tools that can be used to help make a model or data easier to understand. \citet{Venna2007-yj} discusses dimensionality reduction as a tool for data visualization for ML, and reviews many linear and non-linear projection methods. \citet{Vellido2012-nm} also discusses the importance of DR for making ML models interpretable. As one example, \citet{Kadous1999-rx} applied this idea in learning comprehensible descriptions of multivariate time series of Australian sign language by making parameterized event primitives (PEPs), which are commonly occurring patterns in the time series. This resulted in features that were more interpretable to human users.

    In some cases it is desirable to keep a complex, less interpretable, model and then try to explain the results to the user. \citet{Lacave2002-cu} address this from the perspective of explaining Bayesian networks. They are concerned with \emph{how} and \emph{why} a Bayesian network reaches a conclusion. They present three properties of explanation: 1) content (what to explain), 2) communication (how to explain), and 3) adaptation (how to adapt based on who the user is). It is not possible to cover all of the ideas that they present in their paper, but they are key to the idea of designing assurances. Some key points are that they highlight the differences between explaining evidence, the model, or the reasoning. These are three key considerations in making assurances. They also discuss whether an explanation is meant to be a description or for comprehension, as well as whether explanations need to be on a macro or micro scale (as mentioned by \citeauthor{Ruping2006-xj}). They also consider whether explanations should occur by two-way interaction between system and user, by natural language interaction, or by probabilities. Finally, when considering adaptation, they hit on another key point of assurances, which is that in general application not all users will require (or desire) the same kinds of assurances. This paper points out many challenges and considerations in designing assurances, and illustrates that, as with the `No free lunch' theorem, there is no single `best' assurance that will address every possible situation. Other discussion regarding how explanations can be presented is found in \cite{Rouse1986-dz,Wallace2001-fm,Kuhn1997-qc,Lomas2012-ie,Swartout1983-ko}.

    Models and logic are not trustworthy by themselves; they may be flawed to begin with, or when certain assumptions or specifications are violated. Thus, there is also great interest in providing assurances that the models and assumptions underlying different AIA processes are, in fact, sound. \citet{Laskey1991-mf} -- with the intention of helping users of `probability based decision aids' by communicating the validity of the model -- notes that it is infeasible to perform a decision theoretic calculation to decide if revision of the model is necessary. She then presents a class of theoretically justified model revision indicators which are based on the idea of constructing a computationally simple alternate model and then to initiate model revision if the likelihood ratio of alternate model becomes too large (see \citet{Zagorecki2015-qy,Habbema1976-xd} as well).

    \citet{Ghosh2016-dl} present a method, in the framework of a practical self-driving car problem, called Trusted Machine Learning (TML). The main approach of TML is to make ML models fit constraints (be trustable). To do this they utilize tools from formal methods to provide theoretical proof of the functionality of the system. They present `model repair' and `data repair' that they can utilize when the current model doesn't match the data, at which point the model and data can be repaired and control can be replanned in order to conform with the formal method specifications. One challenge that presents itself is how to identify the `trustable' constraints, this returns a lot of responsibility to the designer to foresee all possible failures, which is a strong assumption.

    Another possible way to assure a human user is to use the human in the learning process. \citet{Freitas2006-qo} addressed this point with regards to discovering `interesting' knowledge from data, by comparing two main approaches. Given large datasets (as are typical in many of today's problems), human users require assistance from complex systems in order to find patterns and other `interesting' insights. He mentions `user-driven' methods that involve a user pointing out `interesting' templates, or in another method general impressions in the form of IF-THEN rules. He compares these methods to other `data-driven' methods that have been used, and cites other research that suggests that data-driven approaches are not very effective in practice. This is a cautionary tale that many times engineering methods to assist humans are not as effective as we would like to believe. Although, the `user-driven' approach may not fair any better when compared over many users, as each user will likely have different preferences. \citet{Chang2017-kl} also consider a similar, scaled up, `user-driven' approach called `Revolt' that crowd-sources the labeling of images. It is able to attain high accuracy labeling, while also exploring new or ambiguous classes that can be ignored with traditional approaches.

    Validation and Verification (V\&V) typically refers to using formal methods to guarantee the performance of a system within in some set of specifications. Not all practitioners are conscientious that V\&V provides ways to assure users. A prime example is given by \citet{Raman2013-mz}, who developed a way by which a user can provide natural language specifications to a robot and a `correct-by-construction' controller will be built if the specification is valid. Otherwise, the robot will provide an explanation about which specification(s) cause failure. They study this with the goal of implementing the system on a robot assistant in a hospital. Their method involves parsing natural language input (such as: ``Go to the kitchen''), and converting that to linear temporal logic (LTL) that represents a task specification. This is then used to construct a controller if possible, otherwise the `unrealizable' specifications need to be communicated back to the user. This approach is promising in that it presents a way to communicate that a specification cannot be met, although it does not formally account for effects on user trust or TRBs in formulating explanations. The expression of assurances is also asymmetrically limited to cases where the robot cannot meet the specifications. 

\subsubsection{Summary}
    There are a few main approaches that researchers have used to create explicit assurances:

    \begin{itemize}
        \item Making AIA capabilities more interpretable -- using techniques to simplify models, visualize data, explain reasoning and decision making, or to involve humans directly in the learning process is an attempt to create methods that are well-suited to convey assurances to human users. The output of these methods is the assurance, their interpretable form is well-suited to expressing that assurance.
        \item Predicting AIA performance -- it is critical to predict performance in order for humans to understand how AIAs will behave in possible scenarios. By using methods that can predict performance, designers can give AIAs the capability to explicitly communicate regarding competence, predictability, and situational normality to human users.
        \item Ensuring capability of AIAs -- approaches such as model checking, or formal V\&V allow the AIA to assess whether it is still functioning according to its design; this information is useful in making explicit assurances for human users.
    \end{itemize}

    The underlying hypothesis in this body of research is that the proposed methods \emph{should} affect the user's perception of the `competency', and `predictability' of the AIA, or the `situational normality' of the task being performed. However, none of this has been tested by experimentation that gathers self-reported changes in trust. Similarly, there has been no formalization of how the effects that the proposed assurances have on TRBs might be quantified in different applications. In essence this work is only addressing a small subset of important considerations for designing assurances, or the subset that includes the methods by which to calculate the assurance. The experimental testing of the effects of the proposed assurances on both self-reported trust, and TRBs remains open. Indeed, the research in this quadrant is mainly focused on what AIA engineers and designers \emph{think} needs to be explained, or what assurance they \emph{think} users should have. These ideas definitely have merit, but need to be tested to identify if they are effective, to what extent they are effective, and whether there is a more effective, or more efficient way to achieve similar results.

\subsection{Quadrant IV. (Implicit Assurances, Informal Trust Treatment)}\label{sec:q4}
    The work in this section is easily distinguished from that in Quadrant I because it does not discuss trust in any way. However, it is only subtly different from that in Quadrant III. The research in Quadrant III is explicitly focused on things like interpretability and explanation, according to the authors of those works. Conversely, the research found in this quadrant is only related to trust and assurances by those who are familiar with the underlying concepts in this paper. This research is created with the intent of making the AIAs: inherently more safe; aware of reasoning processes; and possess better task, world, and/or self representations in some way, without intending to communicate these properties to users. 
%This research group thus features design ideas and concepts that can be exploited deliberately to create trustworthy AIAs. 
Here are found the researchers who created AIAs with properties, like reliability, that can then be investigated by those who formally acknowledge human-AIA trust in Quadrants I and II. In other words, these are the methods can be turned into explicit assurances by designers who intentionally do so. 

One very promising area is research regarding safety and learning under nonstationarity. While a fairly high-level treatment, \citet{Amodei2016-xi} are concerned with `AI safety', which is in essence: how to make sure that there is no ``unintended and harmful behavior that [emerges] from machine learning systems''. Given this definition, much of the paper discusses concepts that are critical to AIA assurances. Among the more directly applicable topics in the scope of this paper are: safe exploration (how to learn in a safe manner), and robustness to distributional shift (a difference between training data and test data). They also discuss designing objective functions that are more appropriate. %To restate more concretely, there is a need to design objective functions that more accurately reflect the true objective function. 
A popular example (roughly summarized here) from \citet{Bostrom2014-fz} is a robot that has an objective of making paper clips, it then decides to take over the world in order to maximize its resources and ability to make more paper clips. This highlights the point that sometimes over-simplistic objective functions can result in unintended and unsafe behaviors. There are several different researchers who have investigated these ideas (for example see \cite{Sugiyama2013-ci,Quinonero-Candela2009-fj,Hadfield-Menell2016-ws,Da_Veiga2012-gh,Garcia2015-rs}.

Another promising direction is safe reinforcement learning (safe RL), which considers reinforcement learning is environments where failure is extremely costly, such as when using an expensive aerospace vehicle. 
Safe RL is a particularly important area that requires assurances, as the systems are designed specifically to evolve without supervision. \citet{Garcia2015-rs} perform a survey about safe reinforcement learning (RL). They state that there are two main methods: 1) modifying the optimality criterion with a safety factor, and 2) modification of the exploration process through the incorporation of external knowledge. They present a useful hierarchy of approaches and implementations that currently exist. 
As one example, \citet{Lipton2016-dq} design a reinforcement learner that uses a deep Q-network (DQN) and a `supervised danger model'. They call their method `intrinsic fear'. Basically, the danger model stores the likelihood of entering a catastrophe state within a `short number of steps'. This model can be learned by detecting catastrophes through experience and can be improved over time. \citet{Curran2016-ij}, in a more specific application, asks how a robot can learn when a task is too risky, and then avoid those situations, or ask for help. To do this, they use a hierarchical POMDP planner that explicitly represents failures as additional state-action pairs. In this way, the resulting policy can be averse to risk.

    Even though active learning does not intrinsically consider safety, the underlying approaches can be useful because active learners need to be able to search environment in order to reduce uncertainty. This means that they have an internal representation of uncertainty. \citet{Paul2011-vr} introduced `perplexity' as it applied to image classification. In this setting, perplexity is a metric that represents uncertainty in predicting a single class. This measure is used to select the most perplexing images for further learning. 
Recently, there have been several papers that attempt to use Gaussian processes (GPs) as a method to actively learn and assign probabilistic classifications (see \citet{MacKay1992-sp,Triebel2016-kj,Triebel2013-ow,Triebel2013-ku,Grimmett2013-gj,Grimmett2016-yc,Berczi2015-rd,Dequaire2016-kh}). The applications surveyed here are all mainly related to image classification and robotics. As with perplexity-based classifiers, the key insight is that if a classifier possesses a measure of uncertainty, then that uncertainty can be used for efficient instance searching, comparison, and learning, as well as reporting a measure of confidence to users. The key property of GPs that makes them attractive for this purpose is their ability to produce confidence/uncertainty estimates that grow more uncertain away from the training data. That is, GPs have the inherent ability to `know what they don't know', and this information can be readily assessed and conveyed to users, even in high-dimensional reasoning problems. This property of GPs has also found great use in other active learning applications for AIAs, such as Bayesian optimization (see \citet{Williams1998-kr}, \citet{Snoek2012-tt}, \citet{Brochu2010-tj}, and \citet{Israelsen2017-zb}).

Another promising field of research is related to learning representations of data and selecting data features. These two topics are surveyed by \citet{Bengio2013-uv} and \citet{Guyon2003-fj} respectively. From some of the discussion of interpretable models in section \ref{sec:q3}, we find that representation is important for developing interpretable models. 
Having appropriate representations (i.e. like the ones humans use and understand) is a large step forward in developing assurances for users. As another example, \citet{Mikolov2013-lt} studied how to represent words and phrases in a vector format for natural language text learning. Using this representation, they are able to perform simple vector operations to understand similar words, and the relative relationships learned. For example the operation $airlines+German$ yields similar entries that include $Lufthansa$. This type of representation encodes information that can be easily checked and understood by humans, and thus implicitly facilitates interaction and calibration of trust (see \cite{Haury2011-zi} for another example). 
The problem of discovering human understandable features and representations in more general settings still remains an open question. Currently, the main question in the representation learning field is how to find the `best representations' for a particular application, not necessarily the representations and features that are `most humanly understandable'. This is not surprising, since human representations and features are not necessarily optimal, and AIAs are being designed to be optimal using other objective functions (which are arguably more appropriate, if humans don't need to understand what is going on). 

\subsubsection{Summary}
    Here we have presented several promising fields of research that focus on implicit assurances and informal treatments of trust, which we believe will contribute to the development of explicit assurances from AIAs to humans.
    \begin{itemize}
        \item Safety, learning under non-stationarity, and risk averse learning -- the methods applied to address these challenging problems can be used for creating assurances.
        \item Active learning -- AIAs that are able to actively learn are able to assess holes in their capabilities. The ability to assess these limitations can be used to create better assurances.
        \item Representation learning and feature selection -- learning representations and selecting features that are inherently meaningful to humans will help improve the ability for AIAs to convey assurances to human users.
    \end{itemize}

The literature surveyed in this section is \emph{not} exhaustive, nor could it reasonably ever claim to be. One key takeaway point is that, in every application where designers want to ensure reliable and correct application of an AIA (an informal acknowledgment of the importance of user trust), implicit assurances at a minimum are available which can be intentionally converted into explicit assurances by the designer. 
The disciplines selected in this paper are a subset that are aligned with the author's interests in unmanned vehicles. There are doubtless many other applicable areas of research, and to this end we hope to have whetted readers' minds in order to promote the design of more advanced and capable explicit assurances.

