\emph{Tutoring vs Telling:}
Most assurances investigated to date are `telling', in that they do not consider the experience or other traits of different users. The ability to adapt to different users, and to tutor them to appropriate trust will become more critical as time passes due to the diversity of users bases for advanced AIAs and time that users will interact with them. A tutoring assurance would be a planned, dynamic, sequence of assurances that would change in time to adapt to the user's needs. This might include modification of assurances to help a user avoid boredom, or to use the system differently in varying circumstances. It isn't surprising that, to our knowledge, no research has been done with respect to tutoring a user in a trust relationship. This is a complex problem to address that would involve understanding how different users learn, and what an appropriate strategy would be to teach them to have appropriate TRBs. However, a rich resource (not investigated in this paper) would be the work on tutoring systems \citet{Wenger2014-ld} and algorithmic teaching \citet{Balbach2009-jw}.
