\subsubsection{Tutoring vs Telling -- \textbf{could be called dynamic, vs. static?}} \label{sec:teach_tell}
    Assurances can be classified by whether they consider tutoring (or leading) the user to help them understand, or whether they just produce a static and unchanging value regardless of the user, their experience, or their expertise.

    A tutoring assurances would be a planned, dynamic, set of assurances that would change in time to adapt to the user's needs. This might include modification of assurances to help a user avoid boredom, or to use the system differently in varying circumstances. For example the first time an autonomous vehicle encounters snow with a certain user, it might take time to give special assurances. Or a user that is so used to an AIA that its TRBs begin to drift to disuse, and the AIA gives a special assurance to correct that.

    It isn't surprising that, to my knowledge, no research has been done with respect to tutoring a user in a trust relationship. This is a complex problem to address that would involve understanding how different users learn, and what an appropriate strategy would be to teach them to have appropriate TRBs.
