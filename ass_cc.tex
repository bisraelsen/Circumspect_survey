\emph{Component and Composite:}
A component assurance is an assurance that originates from a single AIA capability source, and targets a single trust dimension target. Component assurances are perhaps the most well researched in the existing literature. A component assurance might include displaying the confidence of a classification prediction, or visualizing a model. A composite assurance is the combination of more than one component assurance into a single assurance. Composite assurances are assurances that are built of several component ones. A notable example is the work by \citet{Aitken2016-cv} who propose a measurement called `self-confidence'. They propose that this metric combine five component assurances into a single composite assurance that is meant to distill the information into a value that a novice operator could understand easily. Figure \ref{fig:assurance_mapping} illustrates the concepts of component and composite assurances. 

How can component assurances be combined to create a composite one? Also, To what extent can component and composite assurances be used in concert to provide assurances for users of differing expertise? Are the effects of a composite assurance equal to the sum of its components?

\begin{figure}[!htbp]
    \centering
    \includegraphics[width=0.7\textwidth]{Figures/Assurance_component_composite.pdf}
    \caption{Figure illustrating the difference between component and composite assurances. The existence of multiple assurances does not imply a composite assurance, rather the combination of multiple component assurances into a single assurance constitutes a composite assurance. On the left there are two component assurances $a1$ and $a2$, whereas on the right only $a3$ affects the trust dimension.}
    \label{fig:assurance_mapping}
\end{figure}
