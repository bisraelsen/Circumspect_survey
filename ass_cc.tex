\subsubsection{Component and Composite Assurances}
Assurances can be either component or composite. This was seen a little through the survey. The definitions are as follows:

\begin{description}
    \item [Component Assurance:] An assurance that originates from a single AIA capability source, and targets a single trust dimension target.
    \item [Composite Assurance:] The combination of more than one component assurance into a single assurance. 
\end{description}

\begin{figure}[!htbp]
    \centering
    \includegraphics[width=0.9\textwidth]{Figures/Assurance_component_composite.pdf}
    \caption{Figure illustrating the difference between component and composite assurances. The existence of multiple assurances does not imply a composite assurances, rather the combination of multiple component assurances into a single assurance constitutes a composite assurance.}
    \label{fig:assurance_mapping}
\end{figure}

Figure \ref{fig:assurance_mapping} illustrates the concepts of component and composite assurances.

\paragraph{Component Assurances:} Component assurances are perhaps the most well researched in the existing literature. This is likely because several verified component assurances are the predecessors to composite ones. A component assurance might include displaying the confidence of a classification prediction, or visualizing a model as discussed in section \ref{sec:q2}.

\paragraph{Composite Assurances:} Composite assurances are assurances that are built of several components. A notable example is the work by \citet{Aitken2016-cv} who propose a measurement called `self-confidence', applicable to Partially Observable Markov Decision Processes (POMDPs). This metric combines five component assurances into a single composite assurance that is meant to distill the information into a value that a novice operator could understand easily. This paper was discussed in more detail in \ref{sec:q2}. 
