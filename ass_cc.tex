\subsubsection{Component and Composite Assurances}
Assurances can be either component or composite. The definitions are found below: \textbf{definitions aren't quite good enough yet}

\begin{description}
    \item [Component:] An assurance that originates from a single algorithm.
    \item [Composite:] The combination of more than one component assurance into a single assurance. 
\end{description}

\begin{figure}[!htbp]
    \centering
    \includegraphics[width=0.9\textwidth]{Figures/Assurance_component_composite.pdf}
    \caption{Figure illustrating the difference between component and composite assurances. The existence of multiple assurances does not imply a composite assurances, rather the combination of multiple component assurances into a single assurance constitutes a composite assurance.}
    \label{fig:assurance_mapping}
\end{figure}

\paragraph{Component Assurances:} Component assurances are perhaps the most well researched in the existing literature. This is perhaps because component assurances are the best type of assurances for controlled experiments. This includes ideas like displaying the confidence of a classification prediction.

\paragraph{Composite Assurances:} Composite assurances are assurances that are built of several components. A notable example is the work by \citet{Aitken2016-cv} who propose a measurement called `self-confidence', applicable to Partially Observable Markov Decision Processes (POMDPs). This metric combines five component assurances into a single composite assurance that is meant to distill the information into a value that a novice operator could understand easily. This paper will be discussed further in \ref{sec:survey}. 
