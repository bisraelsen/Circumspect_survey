\subsubsection{Explicit and Implicit Assurances}
Herein we have only considered \emph{designed algorithmic assurances}, however human's will always form some kind of trust relationship to an AIA, regardless of the presence of designed assurances. In the absence of designed assurances how can a user inform their trust? They do so by using \emph{implicit}, undesigned, assurances. These can be thought of as assurances that are side-effects of other design decisions not meant to affect a user's trust.

Why is it important to consider implicit assurances? There is always a danger that users can attend to the `wrong' assurances, i.e. AIA features that are not meant to be interpreted as assurances but are nevertheless easily perceived (possibly moreso than intended explicit assurances). For example a designer may create a planning-predictability assurance for an AIA, however it could be rendered ineffective by an implicit assurance given by the appearance, or smell, or color of the AIA (research in marketing seems immediately relevant here).

Generally, designers overlook implicit assurances (i.e. do not consider them explicitly in design) because they \emph{assume} that they will have no effect (i.e. why does it matter if there are rust-holes if the UGV works?). This can stem from ignorance of human-AIA trust dynamics, or failure to identify which assurances are most important to users. While it might be desirable, it is generally unreasonable and practically inefficient to attempt a study of \emph{every possible} assurance from an AIA to a user and then select the most important. Perhaps one way a designer might try to identify which assurances are important is to perform user studies, to obtain feedback about which characteristics of the AIA most affected user trust. An approach like this would help determine if explicit assurances are being picked up, and if there are implicit assurances that are overly influential or that overwhelm explicitly designed assurances. With such feedback, designers would have a realistic idea about whether their explicitly designed assurances are having the desired effect on user TRBs. We use the UGV road-network problem to illustrate: after designing an explicit assurance, the supervisor-UGV team could work together in a training mission. Afterwards, the supervisor could rank the different behaviors/properties of the AIA affected their trust in it. In this way, the critical implicit and explicit assurances will be identified. If the explicit assurance is near the top of the list of influencing assurances, then it is working; if not a re-design may need to occur.

What are the most efficient ways in which influential implicit assurances can be identified and dealt with? To what extent are implicit assurances more or less effective than explicit ones? How can designers know when implicit assurances are overriding explicit ones? These questions are open for further investigation.
