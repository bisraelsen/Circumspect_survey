\subsubsection{Explicit and Implicit Assurances}
\citet{Sheridan1984-kx} briefly alluded to the existence of explicit and implicit assurances when they discussed the nature of how humans behave when working with automated systems. They suggested that the operator's perception of the automated system can be effected by `performance' and its `reports on its own performance'. The terms are more formally defined as,

\begin{description}
    \item [Explicit:] Assurances that are purposefully given to affect the trust of a user.
    \begin{itemize}
        \item Legible motion \cite{Dragan2013-wd}, which is motion calculated with the intent of being more understandable by a human
        \item $R^2$ value, gives some indication of how well the regression accounts for the variance of the data
    \end{itemize}
    \item [Implicit:] All other assurances that aren't explicit.
    \begin{itemize}
        \item Reliability in completing a task. Generally, the object of success is not to affect the user's trust (although this is a nice side-effect).
        \item The way an autonomous vehicle appears. For example something that looks neat will have a different effect on trust, than an AIA with wires dragging on the ground. 
    \end{itemize}
\end{description}

It is important for designers of AIAs to be aware of both kinds of assurances, but anything that is consciously designed with the goal of affecting trust is automatically an explicit assurance, from the perspective of the AIA. Another way of stating the ideas is that trust relationships between humans and AIAs will form, but all assurances will be implicit if designers do not consciously consider the trust relationship. 

\nisarcomm{this subsection should be revised: this was clearly meant to go in the beginning, but now feels out of place...it feels like you should instead recap the different important features of explicit and implicit assurances here, based on the work you've surveyed...that is: what was learned about explicit assurances? what was learned about implicit assurances? what are the major themes/ideas that come up as far as realizing each kinds of assurance? where are the gaps/opportunities for performing research for both? basically, this should be a "grand summary and analysis of summaries" across all the different quadrants. What about revisiting the trust ideas? Which areas/aspects of trust and/or TRBs seem to be most frequently addressed in the literature, or which do researchers/designers seem to be most preoccupied with? Which ones seem to have been ignored, or which ones will be important to look at in the future? What big open questions emerge from each quadrant? What big ideas live in each quadrant? Ultimately, you must revisit your thesis and your argument about what all these different methods you looked at have to say about the state of the art for assurances and trust/TRB interactions in AIAs...discuss the justifications and implications for the argument you put forth in the introduction...}
