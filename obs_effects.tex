\subsection{Observing Effects of Assurances} \label{sec:measuring_effects}
    Since assurances are meant to influence TRBs, it is important to quantify these effects so that:  1) the AIA system designer can understand how effective the assurances actually are; and 2) the AIA can observe and respond to/adjust the efficacy of its assurances. 
    To our knowledge, there has not been any work that enables an AIA to observe user responses to assurances and then adapt behaviors appropriately (at least not in the trust cycle setting). 

    There are two known approaches to measuring the effects of assurances: gathering self-reported changes~\cite{Mcknight2011-gv,Muir1996-gt,Wickens1999-la,Salem2015-md,Kaniarasu2013-ho}, and measuring changes in TRBs~\cite{Freedy2007-sg,Desai2012-rc,Salem2015-md,Wu2016-ei,Bainbridge2011-pl}. Measuring changes in TRBs is the more objective approach generally speaking, but the choice between one method and the other depends on the application. Still, more investigation is needed to identify the \emph{principles} behind measuring the effects of assurances. Some interesting, yet unanswered, questions include: are there some TRB measurement strategies that fare better than others for particular kinds of applications or assurances? In what ways, if any, do these methods need to be adapted to suit different kinds of users? Is it possible to show that there are in fact causal relations from specific assurances to specific TRBs? 