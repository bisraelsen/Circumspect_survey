\subsection{Observing Effects of Assurances} \label{sec:measuring_effects}
    Since assurances are meant to influence TRBs, it is important to quantify those effects so that:  1) the AIA system designer can understand how effective the assurances are (see previous section); and 2) the AIA can observe and respond to/adjust the efficacy of its assurances. To our knowledge, there has not been any work that enables an AIA to observe user responses to assurances and then adapt behaviors appropriately (at least not in the trust cycle setting). 

    There are two main approaches to measuring the effects of assurances. First, gathering self-reported changes~\cite{Mcknight2011-gv,Muir1996-gt,Wickens1999-la,Salem2015-md,Kaniarasu2013-ho}, and second measuring changes in TRBs~\cite{Freedy2007-sg,Desai2012-rc,Salem2015-md,Wu2016-ei,Bainbridge2011-pl}. Generally measuring changes in TRBs is the more objective approach, but the choice between one method and the other depends on the application. Still, more investigation needs to occur in order to identify the \emph{principals} behind measuring the effects of assurances. Some interesting, yet unanswered, questions include: Are there some approaches that fare better than others? In what ways, if any, do these methods need to be adapted to suit different kinds of users? To what extent is it possible to ascertain the causal link between an assurance and a TRB?
