\subsection{Expression and Perception of Assurances} \label{sec:express_assurances}
Expression and perception of assurances have been combined in this section because they share several critical aspects. The key points to be considered here in design of assurances are:
(1) Mediums; (2) Methods (; and (3) Efficacy.     
    
For explicit assurance design, the medium and method of expression must account for the AIA's limitations. 
Here medium denotes the means by which an assurances is expressed; this could be through any of the senses by which humans perceive, such as sight, sound, touch, smell, and taste. The method of assurance is the way by which the assurance is expressed. For example: a time series plot may be conveyed visually in the typical way, or via a spoken or textual description; in this case the plot or text description is the method, and sight or sound are the different mediums through which it can be communicated. An AIA might be limited in methods of expression, e.g. because it does not have a display or a speaker.

A designer must also consider whether a human can perceive the assurances being given. If so, to what extent is the information from the assurance transfered, or how efficacious is the assurance? A few examples include: an AIA giving an auditory assurance in a noisy room and the user not hearing it (such as an alert bell in a factory where the workers use ear-plugs), or an AIA attempting to display an assurance to a user that has obstructed vision.
An AIA may also have the ability to store data about its performance, and compute a statistic regarding its reliability -- but if it cannot successfully express (or communicate) that information in some way, the information is useless.

Users will generally not know which assurances are implicit and which are implicit.

\subsubsection{Mediums}
In general, assurances are most often expressed visually. For example an AIA might give visual performance feedback to display different performance characteristics \cite{Chadalavada2015-wx,Muir1996-gt}. Written or spoken natural language can also be used \cite{Wang2016-id} -- given the impressive strides made by NLP researchers and practitioners lately, it is nowadays a simple matter to convert between written natural language and spoken natural language. 
These can be used to augment or replace more conventional audio-visual assurance indicators traditionally used and studied for human-machine/human-automation interaction, e.g. blinking lights, colored boxes in graphical displays, ringing bells/buzzers, recorded voice alerts, etc.
    
Other senses (touch, smell, and taste) are not well explored in literature related to human-AIA trust. Generally, any human sense could be used as a medium. Besides sight and sound, tactile feedback has been used extensively in robotics for `haptic feedback' (where the user receives mechanical feedback through the robot controls). This medium is used to create a more immersive user interface in robotics, to help users feel more connected to the robot (especially important for telerobotics applications). 
While one can imagine smell and taste having obvious applications in designing assurances for a cooking robot, other applications very likely exist and are open to further research.

\subsubsection{Methods}
Assurances associated with displaying AIA performance variables sound banal (e.g. flow rate for an automated pump \cite{Muir1996-gt}), but actually involves a nuanced point: the displayed performance value actually serves to inform the user's own mental model of the trustworthiness of an AIA capability. That is, the user's trust in the AIA's capability does not change only in response to the instantaneous `goodness/badness' of the AIA's performance, but accounts for the past history of the AIA's performance as well as any observed discrepancies between the AIA's expected behavior and its actual behavior.  
The user's trust dimensions (`competence', 'predictability', etc) are then affected by their perception of trustworthiness according to the combined model and data delivered by the display. 
This approach (also noted and discussed by \cite{Wickens1999-la,Sheridan1984-kx,Hutchins2015-if}) is effective, but relies heavily on the implicit assumption that the user will create a `good enough statistical model' of the AIA's behavior from data presented by the AIA. With this in mind, one might train a user to recognize signs of failure/success in different interactions with an AIA as assurances \cite{Freedy2007-sg,Desai2012-rc,Salem2015-md}. 
The main drawback of this idea is that it still relies on users' ability to construct `good enough' mental models of AIA behavior and characteristics from noisy observations to avoid misinterpreting AIA behaviors. 
However, this training can require intensive and costly special effort for non-expert or non-specialist users. 
A more ideal approach in such cases would be to design explicit assurances that help users construct correct/consistent mental process models of AIA behavior and thus reduce the risk of misinterpretation.

One final point is that several potential sources for explicit assurances lack well-established human-understandable expressions, and thus are not yet widely utilized as effective assurances. For example, it is unclear how an AIA can best express that it has been formally validated and verified for similar operational settings. 
Similarly, it is unclear how information related to random variables can be best communicated to users besides showing them histogram or probability distribution plots (only useful for 1 or 2 dimensional distributions), or displaying statistics such as means and variances. 
Investigating and understanding how such useful, but otherwise difficult to understand, types of information can be expressed as explicit assurances will be critical to enhancing human-AIA trust relationships. 

\subsubsection{Efficacy}
Some kinds of expression are very `one-dimensional' in that they only rely on one medium or method. This, again, has been seen in practice by the use of plotting a certain AIA performance variable value over time. Because of this, much of the research to date involves assurances that are not robust to loss in transfer, i.e. the approaches rely heavily on a specific medium and method to work, otherwise the whole assurance is rendered useless. Hence, the problem of robustly communicating assurances remains an open research question. An analogy can be made here to a person speaking to another person with their voice, while also making facial expressions and gestures with their hands, thus simultaneously utilizing several mediums/methods helps to ensure the effectiveness of an assurance. If the person instead tries to simply repeat the same message over and over many times to the other person in a monotone voice without changing facial expressions or making any other gestures, then this would be considered inefficient, especially if the message is not received or considered to be effective after the first attempt. %This raises the idea of efficiency in expression. 
    
Ideally, a highly effective assurance communicate in a precise and direct way that is easy for the user to recognize and understand, with little loss in content or ability to act correctly on the assurance in a given situation. 
The keys to efficacy lie in the selection of appropriate methods and mediums for assurance expression, and in the design of appropriate levels of redundancy to ensure that assurances are correctly received and interpreted in a timely manner.
