\subsection{Expression and Perception of Assurances} \label{sec:express_assurances}
A specific algorithm may be responsible for selecting the content of an assurance, but it is also critical to consider the \emph{form} of the assurance. The expression (and subsequent perception) of an assurance involves considering 1) mediums, 2) methods, and 3) efficacy. 

The medium of an assurance includes the form in which it expressed, such as visually (the most popular), audibly, or otherwise. The method of expression includes for example using a plot, or a natural language phrase (and based on the medium, for example, the natural language could be written or spoken). Finally, the efficacy of the assurance must also be considered; as an illustration of when an assurance might become less `efficient' consider using an audible assurance in a noisy environment. Humans generally address this by utilizing different methods/mediums in order to compensate; in fact at times multiple assurances may be used simultaneously to, in essence, make the assurance more robust to `loss in transfer'.

The greatest challenge in using different mediums and methods is not in their implementation, the challenge is designing the ability to recognize and decide when they should be applied. Some interesting questions are: In what circumstances are different methods most useful? And the same for mediums? How can different methods/mediums be selected in order to optimize the efficacy of an assurance while also taking into account that using all possible combinations will \emph{not} help the user? To what extent can the efficacy of an assurance be assessed during operation, and how would this be done?
