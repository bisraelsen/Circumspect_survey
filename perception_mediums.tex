\subsection{Expression and Perception of Assurances} \label{sec:express_assurances}
Although specific algorithms can be used to build the contents of assurances, it is also critical to consider the actual communication of assurances. The expression (and subsequent perception) of an assurance involves considering mediums, methods, and efficacy. The medium of an assurance includes the form in which it expressed, e.g. visually, audibly, or otherwise. 
The method of expression includes for example using a plot, or a natural language phrase (which could be text-based or speech-based, depending on the medium).  Finally, the factors influencing the efficacy of the assurance must also be considered (e.g. consider using an audible assurance in a noisy environment). Humans generally utilize different methods/mediums when communicating assurances to each other to maintain efficacy when potential `losses in transfer' might occur. 
However, arguably the greatest challenge in using different mediums and methods is not in their implementation, but in designing the ability to recognize and decide when they should be applied. Some interesting questions are: In what circumstances are different methods most useful? And the same for mediums? How can different methods/mediums be selected in order to maximize assurance efficacy while also taking into account that using all possible combinations will \emph{not} help the user? How and to what extent can AIAs assess the efficacy of an assurance before, during, or after operation?