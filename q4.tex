\subsection{Quadrant IV. (Implicit Assurances, Informal Trust Treatment)}\label{sec:q4}
\subsubsection{Safety and Learning under nonstationarity and risk aversion} \label{sec:safety}
    While a fairly high-level treatment, \citet{Amodei2016-xi} are concerned with `AI safety', which is in essence how to make sure that there is no ``unintended and harmful behavior that [emerges] from machine learning systems''. Given this definition, much of the paper discusses concepts that are critical to AIA assurances. Among the more directly applicable topics in the scope of this paper are: safe exploration (how to learn in a safe manner), and robustness to distributional shift (a difference between training data and test data). They also discuss designing objective functions that are more appropriate. To restate more concretely, there is a need to design objective functions that more accurately reflect the true objective function. A popular example (roughly summarized here) from \citet{Bostrom2014-fz} is a robot that has an objective of making paper clips, it then decides to take over the world in order to maximize its resources and ability to make more paper clips. This highlights the point that sometimes over-simplistic objective functions can result in unintended and unsafe behaviors.

    Generally, stationary data (data whose distribution does not change after training) is assumed in supervised ML. \citet{Sugiyama2013-ci} considers what to do when there is `covariate shift' (or when both training and test data change distribution, but their relation does not change between training and test phases), and `class-balance change' (where the class-prior probabilities are different in training and test phases, but the input distribution of each class does not change). They design and present tools that help diagnose and treat those conditions (this is follow on work for some of what is presented in \citet{Quinonero-Candela2009-fj} where they consider dataset shift). A key approach is to use importance sampling, which involves weighting the training loss by the ratio between the probability of the test data and that of the training data.

    \citet{Hadfield-Menell2016-ws}, in considering an AI safety problem, address the problem of verifying that a robot can be turned off, even when it might have an objective function that indicates that disabling the `off-switch' will reduce the chance of failure. This kind of scenario, or something similar, can easily occur with a sufficiently sophisticated and capable AIA, and a complex enough set of objectives that might result in unintended consequences. They propose that one objective would be for the robot to maximize value for the human, and to not assume that it knows how to perfectly measure that value. 

    \citet{Da_Veiga2012-gh} discuss a form of safety where they are concerned with nonparametric classification and regression with constraints. More specifically, they are concerned about learning Gaussian process (GP) models with inequality constraints, and present a method to do this by using conditional expectations of the truncated multivariate normal distribution (or Tallis formulas). This is not the only work that references learning with constrained GPs. It is also not the only work that considers constrained modeling, but it would take too long to review all of those papers. The main claim here is that constrained models are a way to guarantee the properties of a model within some specifications.

    \citet{Garcia2015-rs} perform a survey about safe reinforcement learning (RL). Safety in RL can be critical based on the application, such as an aerospace vehicle that can cost several thousands of dollars. They state that there are two main methods: 1) modifying the optimality criterion with a safety factor, and 2) modification of the exploration process through the incorporation of external knowledge. They also present a hierarchy of approaches and implementations. Some approaches used when modifying the optimality criterion are worst-case, risk-sensitive, and constrained criterion. Whereas, modifying the exploration process is done through using external knowledge, and as well as using risk-directed exploration. Safe RL is a particularly important area that requires assurances, as the systems are designed specifically to evolve without supervision.

    As one example, \citet{Lipton2016-dq} design a reinforcement learner that uses a deep Q-network (DQN) and a `supervised danger model'. Basically, the danger model stores the likelihood of entering a catastrophe state within a `short number of steps', this model can be learned by detecting catastrophes through experience and can be improved over time.  In this way they show that their method, they call `intrinsic fear', is able to overcome the sensitive nature of DQNs. There is a clear limitation of the danger model, in that it does not contain useful information until a catastrophe is detected. 

    \citet{Curran2016-ij}, in a more specific application, asks how a robot can learn when a task is too risky, and then avoid those situations, or ask for help. To do this, they use a hierarchical POMDP planner that explicitly represents failures as additional state-action pairs. In this way the resulting policy can be averse to risk. They say that this method can be especially useful when optimal actions are not straight-forward, and they state that it can use any reward function. It seems that this method might suffer from some of the typical problems of POMDPs, which are computational complexity in high-dimensional state spaces.

\subsubsection{Active Learning} \label{sec:active_learning}
    \citet{Paul2011-vr} is concerned with whether a robot can improve its own performance over time, with the goal of `life-long learning'. They use `perplexity' which is a method first introduced in language models and adapted to work with images. Perplexity, in their application, is a measure that indicates the uncertainty in predicting a single class. Over time, the most perplexing images are stored and used in expanding the sample set. This work is interesting for application in assurances because the ability to quantify something that is perplexing is a predecessor to being able to communicate that to a human user.

    Recently, there have been several papers that attempt to use Gaussian processes (GPs) as a method to actively learn and assign probabilistic classifications (see \citet{MacKay1992-sp,Triebel2016-kj,Triebel2013-ow,Triebel2013-ku,Grimmett2013-gj,Grimmett2016-yc,Berczi2015-rd,Dequaire2016-kh}). The applications surveyed here are all mainly related to image classification and robotics. As with perplexity-based classifiers, the key insight is that if a classifier possesses a measure of uncertainty, then that uncertainty can be used for efficient instance searching, comparison, and learning, as well as reporting a measure of confidence to users. The key property of GPs that makes them an attractive for this purpose is their ability to produce confidence/uncertainty estimates that grow more uncertain away from the training data. That is, GPs have the inherent ability to `know what they don't know', and this information can be readily assessed and conveyed to users, even in high-dimensional reasoning problems. This property of GPs has also found great use in other active learning applications, such as  Bayesian optimization (see \citet{Williams1998-kr}, \citet{Snoek2012-tt}, \citet{Brochu2010-tj}, and \citet{Israelsen2017-zb}).

\subsubsection{Representation learning and Feature Selection} \label{sec:rep_learning}
    Another promising field of research is related to learning representations of data and selecting data features. These two topics are surveyed by \citet{Bengio2013-uv} and \citet{Guyon2003-fj} respectively. From some of the discussion of interpretable models in section \ref{sec:model_interp} we find that representation is important for making interpretable models. Having appropriate representations (i.e. like the ones humans use and understand) is a large step forward in making assurances for humans.

    For instance, in their work related to interpreting molecular signatures \citet{Haury2011-zi} investigate the influence of different feature selection methods on the accuracy, stability and interpretability of molecular signatures. They compared different feature selection methods such as: filter methods, wrapper methods, and ensemble feature selection. They found that the effects of feature learning greatly influenced the results. 

    As another example, \citet{Mikolov2013-lt} studied how to represent words and phrases in a vector format. Using this representation, they are able to perform simple vector operations to understand similar words, and the relative relationships learned. For example the operation $airlines+German$ yields similar entries that include $Lufthansa$. This type of representation encodes information that can be checked and understood by humans. 
    
    How can human understandable features and representations be discovered? This is still an open question. The main question in the representation learning world is how to find the best representations, not necessarily the representation and features that are most human. This is not surprising because human representations and features are not necessarily optimal, and AIAs are being designed to be optimal using other objective functions (arguably more appropriate functions, if humans don't need to understand what is going on).

\subsubsection{Summary}
    The literature surveyed in this section is \emph{not} exhaustive, nor could it reasonably ever claim to be. One thing that should be clear is that in every field where designers want to ensure reliable and correct application of an AIA, there will be assurances that are created. The disciplines selected in this paper are a subset that are aligned with the author's interests in unmanned vehicles.

    The work in this section is easily distinguished from that in Quadrant I because it does not discuss trust in any way. However, it is only subtly different from that in Quadrant III. The research in Quadrant III is explicitly focused on things like interpretability and explanation by direct statement of the authors. Conversely, the research found in this quadrant is only related to trust by those who are familiar with the underlying concepts in this paper. This research is created with the intent of making the AIAs intrinsically more safe, aware of reasoning processes, having better representations in some way, and others. This group unintentionally, creates foundations for trustworthy AIAs. Here are found the researchers who created AIAs with properties, like reliability, that can then be investigated by those who formally acknowledge human-AIA trust in Quadrant I. These are the methods can be turned into explicit assurances by designers who intentionally do so.
