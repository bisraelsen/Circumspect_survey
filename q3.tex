\subsection{Quadrant III. (Explicit Assurances, Informal Trust Treatment)}\label{sec:q3}
\subsubsection{Performance Prediction}
    \citet{Zhang2014-he} are concerned with the performance of vision systems; particularly with the common occurrence of vision systems to failure `silently', that is with no warning. They suggest that the two possible solutions are to: 1) analyze the output of the vision system to assess its confidence, and 2) analyze the input itself in order to assess the classifier's confidence. They pursue the second, with the goal of minimizing failures as well as trying to verify the safety of a classifier. They argue that the second method is more desirable because it can be applied to any classifier. In essence they learn a model of the training error associated with different images and then use this to predict the error on test images. They demonstrate that their methods on several different classification tasks and show that it predicts failure ``surprisingly reliably''. \nisarcomm{Comment on any limiting assumptions or possible hurdles to overcome for generalizing to other AIAs??}

    % \citet{Churchill2015-ei} use a Gaussian Process (GP) to model uncertainty of a localization map during a training phase. Uncertainties may come from the lighting conditions that poorly affect the video used for localization. Later during testing, the uncertainty information can be used when the robot is planning and moving within the map. They mention that this technique could apply to other types of mapping that use topological localizers, and laser-based systems.

    \citet{Gurau2016-hs} (following from ideas in \citet{Churchill2015-ei}) consider the task of predicting the performance of a robot's navigation when using cameras for localization. Their aim was to have the robot operate autonomously only when it was confident that it could do so, otherwise it would request human assistance. To achieve this they make use of GPS localized observations and `Experience Based Navigation' (EBN) to identify which past experience matches the current state of the robot. Using this approach they are able to reduce the number of mistakes that the robot makes when navigating in a real scenario. \nisarcomm{ditto prev comment}

    \citet{Kaipa2015-hy} also consider the confidence of a visual classifier, but with the application being a robot that is picking an object out of a bin. To do this they utilize the `Iterative Closest Point' (ICP) method to match which points in a point cloud are likely associated with the object of interest in the bin. With that information the robot can then asses how confidently it can perform the task and if its confidence is below some threshold it will request assistance from a human. \nisarcomm{worth noting that this approach is actually quite similar to `chi-square innovation hypothesis testing' used for adaptive Kalman filtering and used for tuning/debugging Kalman filters [can cite the Bar Shalom book]...also related somewhat to `surprise index' stuff discussed later...main drawback of this approach is that precise model knowledge is required to pull this kind of hypothesis test off, i.e. to accurately predict what the residuals are. Also, these kinds of tests don't necessarily tell you *why* the residuals are off -- only that something is wrong (not what specifically is wrong)...However, they are quite attractive for model-based `sanity checking' for decision-making and perception under uncertainty...}

    \citet{Kuter2015-qh} recognize that plans can be fragile, and suggest that the planner calculate its stability. That is to say, how sensitive is the plan to uncertainties. The suggest `counter planning' and `plan repair' so that the autonomous system can identify likely contingencies \edit{that might interfere with an existing plan} and then adapt the plan to account for those contingencies. If the system \emph{is} able to adapt to contingencies then it will have a higher `self-confidence'. This allows humans to be able to understand the system better, and trust it more appropriately.

    \citet{Hutchins2015-if} consider the fact that autonomous systems have components, and that those components have `competency boundaries'. That is to say that different sensors, planners, and actuators have different capabilities in different conditions. For example a GPS has high competency in an open-air situation, but very low competency inside of a tunnel or building. They suggest that these competency boundaries be quantified, and then displayed to a user, so the user better understands the system's capabilities and can trust it more. \nisarcomm{key limitation: they only propose an expert-based Likert scale for quantifying competency boundaries, i.e. they have no formal backend to assess competency...} \nisarcomm{also worth mentioning that this idea and idea from Kuter/Miller are closely linked to self-confidence idea discussed earlier -- key difference: the Kuter/Miller and Hutchins works do not explicitly relate their approaches to trust components/TRBs; however, Hutchins' work does address the problem of how s/c should be communicated to users, which neither Kuter/Miller nor Aitken really address...}

\subsubsection{Interpretable Models}
    A well-known concept is that there is a trade-off \edit{in machine learning and pattern recognition} is between accuracy and interpretability. \citet{Van_Belle2012-dt} examine this issue in the context of making clinical decision support systems for use in the medical field. To this end they propose an `interval coded scoring' system that imposes a constraint such that each variable has a constant effect on the \edit{estimated decision-making risk (e.g. the Bayes' risk for classification)}.  They demonstrate the method on two datasets, and show that their method can be visualized using scoring tables and color bars. They point out that their method needs to be extended in order to detect interaction effects automatically, and to handle multi-class decision support systems. \nisarcomm{what motivates their approach? this seems like a good spot to point out that humans are notoriously bad at interpreting risk/probabilities on certain scales -- hence the need for something like this, i.e. why not just show them raw numbers or report log-odds and probabilities directly?...}

  Along these lines,  \citet{Ruping2006-xj} specifically asks how classification results can be made more interpretable \nisarcomm{to whom?}. He reviews several possible methods, and states that ratings by experts is perhaps the most accurate way, although it is at the same time subjective. To address the accuracy-interpretability trade-off he investigates the use of a simpler global model, and a more complex local model (\citet{Otte2013-oo} and \citet{Ribeiro2016-uc} implement these ideas as well) \nisarcomm{it's not clear what you mean by "he uses a simpler global model and a more complex local model" -- to do what, exactly?? How does this make classification more interpretable?? Is there a simple figure you can slide in here to illustrate what you mean by this idea?}. This allows simple interpretation of global properties, but more complex local models to maintain accuracy. While his focus was on classification, the methods could also be useful in regression as well. \nisarcomm{Ribiero paper also adds in another sparse LASSO-type regression step for determining which features in the local decision model are most important for `explaining' a particular classification instance, i.e. which features contribute the most to the local decision boundary...are there limitations to this kind of approach, e.g. for one thing, the expert/designer of the system is imposing a canned explanation mechanism onto the user, i.e. it is not responding or adapting to what the user may/may not actually care to know for a specific decision instance...}

    Later, \citet{Van_Belle2013-ph} address the challenge that some of the highest performing ML models are too complicated to be interpreted. They investigate different `interpretable' and visualization methods in order to understand what opportunities they find. They suggest that there are three methods that help ascertain the level of interpretability and potential utility of models: 1) Map to domain knowledge, 2) Ensure safe operation across the full operational range of model inputs, and 3) Accurately model non-linear effects (compare to categories proposed by \citet{Lipton2016-ug}). They analyze some of the existing methods (nomograms, rule induction, graphical models, data visualization) and point out their weaknesses. They finish by pointing to more recent research in sparse and interpretable models, and suggest that it is a promising line of research. In summary, they discuss that each `interpretable' method has its benefits and weaknesses and there is no methods that clearly out-performs any other.

  \nisarcomm{Transition needed here...}  \citet{Caruana2015-za} are interested in predicting the probability that a patient will be re-admitted to the hospital within thirty days. They also mention the trade-off between accuracy and interpretability, and propose a generalized additive model with pairwise interactions ($GA^2M$) model. A generalized additive model (GAM) is generally thought of as interpretable as the effects of individual variables can easily be quantified; however GAMs suffer from low accuracy. In a comparison with other methods, they show that adding pairwise interactions allows the $GA^2M$ model to be as accurate as the current methods, while still maintaining reasonable interpretability. \nisarcomm{unclear what is meant by `the current methods'??} 
\edit{In a similar vein,} \citet{Choi2016-by} modifies a recursive attention neural network to remove recurrence on the hidden state vector, and instead add recurrence on the doctor visits and diagnoses. In this way the model is able to predict possible diagnoses in time, and a visualization can be that that indicates the critical visits and diagnoses that lead to that prediction. These methods are promising because it restructures an advanced learning model in a way that useful information can be extracted. \nisarcomm{However, one might also argue that these methods also require domain expert knowledge to know which additional connections and `out of band information' to incorporate, i.e. how would AIA know what to modify and what else it should link in to a simple model to remain interpretable? would this necessarily require expert intervention from beginning? (this might be a stretch, but basically I am trying to insert the `devil's advocate' view here, like with the other methods above...)}

    \citet{Abdollahi2016-vn} construct a conditional restricted Boltzmann machine (RBM) in order to create a collaborative filtering algorithm, that can suggest `explainable' items, while maintaining accuracy. \nisarcomm{So what? Is there anything special about RBMs that merits bringing this up? What's the strenghts/weaknesses of approach in relation to extending to other applications or adopting ideas for AIAs?}

    \citet{Ridgeway1998-lv} recognize that boosting methods, or classifiers with voting methods, are very accurate but not interpretable. They propose boosting by `weight of evidence' (WoE), where WoE refers to having a weight that indicates whether the observation is positively or negatively correlated to the class. Each weight can then be used to gain some understanding about how the observation affects the classification. They demonstrate its utility using multiple experiments that it has performance on par with AdaBoost, and with a Naive Bayes classifier.  \nisarcomm{again, so what??}

    \citet{Huysmans2011-th} investigate decision trees, decision tables, propositional if-then rules, and oblique rule sets in order to understand which is the most interpretable and perform an experiment to identify which method works most effectively. The experiment involved interpreting a model and answering questions about what the correct classification would be. They made several interesting observations. First, they confirmed that a model with larger representation leads to lower answer accuracy responses. They found that overall decision trees and decision tables were the most interpretable, but that different tasks made the tree or table more desirable (the layout of the data in a table can be superior for certain `look-up' tasks). 
One drawback with decision trees is that the rules can get very complicated in a large tree. \citet{Park2016-ld} investigate how to make rules in decision trees more `intuitive'. They propose a method that learns chains of decisions that together increase the ratio of positive class. They present a method that is called $\alpha$-Carving decision Chain (ACDC), and say that it is a greedy alternative to a general framework known as RuleLists (\citet{Wang2015-ww}). 
Similarly, \citet{Jovanovic2016-gw} use `Tree-Lasso' logistic regression with domain knowledge. Specifically they use medical diagnostic codes to group similar conditions and then use `Tree-Lasso' regression that uses that information to make a more sparse model. \citet{Zycinski2012-jj} also use domain knowledge to structure the data matrix before feature selection and classification. \nisarcomm{...takeaways/so whats, a la previous paragraphs?}

    \citet{Faghmous2014-og} argue that`theory guided data science' is necessary in science applications; for instance, when studying environmental effects, black-box models are of little use. Instead, in order to gain insight they need a theoretical framework, to help highlight causation. Similarly, \citet{Morrison2016-fz} address the situation where an analytical model is available but imperfect. They use a chemical kinetics application where the theoretical reaction equations are well known, and then add a `stochastic operator' over the top of the known model to account for uncertainties. One drawback for these methods is that they require a lot of domain knowledge. However, in the areas of science and engineering where physical models are well understood this type of approach could be useful to indicate the performance of the system with regards to the theoretical model. \nisarcomm{excellent -- this last sentence is the kind of insight/analysis you should tag onto each set of papers you mention (i.e. what I'm cueing/prodding you to do up above)}

\subsubsection{Visualization and Dimensionality Reduction}
    \edit{Intuitively, it might make sense to simply show users the inputs, raw data, and/or intermediate processing steps that led to a particular AIA behavior.} In many real applications, however, there are too many individual variables for a human to attend to. In such situations, dimensionality reduction (DR) and visualization are tools that can be used to help make a model or data easier to understand. \citet{Venna2007-yj} discusses dimensionality reduction as a tool for data visualization \edit{for ML???}, and reviews many linear and non-linear \edit{projection} methods. \citet{Vellido2012-nm} also discusses the importance of DR for making ML models interpretable.

    In an interesting application to time series data, \citet{Kadous1999-rx} asks how comprehensible descriptions of multivariate time series can be learned, with the end goal of interpreting Australian sign language. He focuses on reducing the feature space fed into a learning algorithm (i.e. The data was gathered using an instrumented glove. Naive Bayes), And does so using `parameterized event primitives' (PEPs), which are commonly occurring events or patterns that are expected to occur. \nisarcomm{last 2 sentences are weird -- looks like some sort of cut/paste formatting error???} He shows that his method reduced test error while also having more comprehensible features. It seems likely that parameterized primitives might be automatically learned. \nisarcomm{this observation feels vague/disconnected...}

% \paragraph{designed behavior}
    % Another approach to being able to trust a system that has been investigated, is that of purposfully designing behavior in such a way as to comply with certain requirements.
%
    % \citet{Dwork2012-fq} is one of the few who have directly looked at this question with an explicit focus on making the results of classifiers `fair'. Their work is driven by the interface between ML and the public. They propose making a metric that quantifies the similarity between individuals, and then designing \ldots you know, I don't really like this method. It seems pretty specifically to me.

\subsubsection{Explanation}
    In the context of POMDP planning \citet{Lomas2012-ie} recognize that in order for humans to appropriately trust robots they must be able to predict their behavior, and the robot must be able to communicate in order for that to happen. To this end they present the Explaining Robot Actions (ERA) system that includes a model that represents semantic and physically based information along with other factors to respond to questions in a human-understandable way.

    \citet{Swartout1983-ko} was interested in making explanations for expert systems. He noted that ``trust in a system is developed not only by the quality of its results, but also by clear description of how they were derived''. Often the data (or knowledge) used to train an expert system is not kept in the system for later use. He proposed a method called `XPLAIN' that not only describes \emph{what} it did, but \emph{why} it did it. It does this by using description facts of the domain and prescriptive principles simultaneously; in essence it learns how to make decisions and how to explain them at the same time.

    \citet{Rouse1986-dz} asked how computer-based decision support systems should be designed in order to help humans cope with their complexity. He suggests that methods need to be designed so as to provide different kinds of information which include: patterns vs. elements, and current vs. projected. This work is important in pointing out that assurances also depend on what the role of the human is and what information they need. This is a manifestation of the teaching and telling assurances discussed in section \ref{sec:teach_tell}.

    This was also investigated to some extent by \citet{Wallace2001-fm} in their work concerning explaining outcomes of constraint solvers. They discuss how to distinguish between `good' and `bad' explanations. In other words, those explanations that facilitate or detract from the user's ability to understand how the explanation actually applies to the solution being explained. They critically ask the question given an explanation how should it be presented to the user (something that \citet{Kuhn1997-qc} explores more formally).

    \citet{Lacave2002-cu} revisits some of these ideas from the perspective of explaining Bayesian networks. They are concerned with \emph{how} and \emph{why} a system reaches a conclusion. They present three properties of explanation: 1) content (what to explain), 2) communication (how to explain), and 3) adaptation (how to adapt based on who the user is). Each of the properties has several components that are generally applicable to the design of assurances in other venues, and there are myriad papers that are references with detail on technical application.


\subsubsection{Model Checking}
    \citet{Laskey1991-mf}, with the intention of helping users of `probability based decision aids', notes that it is infeasible to perform a decision theoretic calculation to decide if revision is necessary. She then presents a class of theoretically justified model revision indicators which are based on the idea of constructing a computationally simple alternate model and then to initiate model revision if the likelihood ratio of alternate model becomes too large.

    \citet{Zagorecki2015-qy}, discusses the `surprise index' introduced by \citet{Habbema1976-xd}. The major flaw of the surprise index is that it needs to calculate the likelihood of an observation in the context of a model, which is computationally infeasible. \citeauthor{Zagorecki2015-qy} suggests an approximation by using the log-normal distribution to approximate the distribution of values in the joint probability distribution.

    \citet{Ghosh2016-dl} present a method, in the framework of a practical self-driving car problem, called Trusted Machine Learning (TML). The main approach of TML is to make ML models fit constraints (be trustable). To do this they utilize tools from formal methods to provide theoretical proof of the functionality of the system. They present `model repair' and `data repair' that they can utilize when the current model doesn't match the data, at which point the model and data can be repaired and control can be replanned in order to conform with the formal method specifications. 

\subsubsection{Human-Involved Learning}
    Another possible way to make system assure a human user is to use the human in the learning process. \citet{Freitas2006-qo} addressed this point with regards to discovering `interesting' knowledge from data. Given such large datasets, human users require assistance from complex systems in order to find patters and other `interesting' insights. He mentions `user-driven' methods that involve a user pointing out `interesting' templates, or in another method general impressions in the form of IF-THEN rules. He compares these methods to other `data-driven' methods that have been used, and cites other research that suggests that data-driven approaches are not very effective in practice. This is a cautionary tale that many times engineering methods to assist humans is not as effective as we would like to believe.

\subsubsection{Summary}
Currently, informally designed assurances fall into several different categories listed above. Many methods focus on informing the user how the model/logic works, there is also some concern about predicting future performance. However, this is a survey of many different disciplines, analogs for these methods have not been universally adopted, or even developed in some disciplines. Technical challenges exist that make designing assurances very difficult in some cases. Generally, having humans more involved (as in visualization, or human-involved learning) helps connect the representations/functionality more human-like and thus more trustworthy.
