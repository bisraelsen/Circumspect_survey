\subsubsection{Source-Target Classification}
    It is convenient to refer to assurances by way of their source and target. Intuitively, there may be a set of different algorithms that are useful for making assurances that convey information about planning to the competence dimension of the user's trust. It is easier to refer to these assurances in terms of their source and target. So, for this example that class of algorithms would be the `planning-competence' class.
    
    Not only is this useful shorthand for communicating about the purpose of the algorithms, but it is useful in classifying the range of assurance algorithms that exist. There may also be a class of algorithms that span multiple source-target capabilities. For example there may be a kind of algorithm that can give a `learning-competence' assurance, as well as a `planning-competence' assurance.

    This is especially true since many of the AIA capabilities can overlap. Also, the effects of assurances cannot be guaranteed to affect only one trust dimension.

    \textbf{ugggg, this gets a little complicated, but it's not supposed to be}.
