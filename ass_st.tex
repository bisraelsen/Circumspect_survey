\subsubsection{Source-Target Classification}
It is convenient to refer to assurances by way of their source and target. More specifically their source AIA capability (see Figure \ref{fig:AIcapabilities}) and their target user trust dimension (see figure \ref{fig:Assurance_classes}). For example human-like gestures could be considered a `motion-predictability' assurance. Even though, in practice, it is not easy to clearly separate AIA capabilities or trust dimensions due to the inherent cross-over in many cases. In retrospect, the reader may be able to identify work from \cite{Dragan2013-wd} as a `motion-predictability' assurance. \citet{Wang2016-id} considered `perception-competence', and `planning-predictability' assurances (among others). Meanwhile, \citet{Aitken2016-fb} considered a large set of assurances that span several source capabilities, and target trust dimensions. 

This classification is useful because different classes of algorithms will likely present themselves as useful for assurances whose target is `predictability' for example, than those whose target is `situational normality'. It is not immediately clear what the landscape of assurances from that perspective.

Future research might begin by looking for holes in a certain source-target pair. For example, are there satisfactory assurances for the learning-situational normality source-target pair? Or can assurance $x$ for perception-competence also be applied to learning-competence? Finally, are there certain classes of algorithms that are suited for communicating to the `predictability' dimension of trust, and can they be adapted from one AIA capability to another?

% Intuitively, there may be a set of different algorithms that are useful for making assurances that convey information about planning to the competence dimension of the user's trust. It is easier to refer to these assurances in terms of their source and target. So, for this example that class of algorithms would be the `planning-competence' class.
%
% Not only is the source-target notation useful shorthand for communicating about the purpose of the assurance, but it is useful in classifying the range of assurance algorithms that exist. There may also be a class of algorithms that span multiple source-target capabilities. For example there may be a kind of algorithm that can give a `learning-competence' assurance, as well as a `planning-competence' assurance.

