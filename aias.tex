\subsection{Artificially Intelligent Agents} \label{sec:aias}
    %As noted by \citet{Tripp2011-rx}, 
    Intelligent technology spans a wide spectrum of capabilities. With regards to autonomous systems, these might include anything from a thermostat, to the fabled HAL 9000. While the main interest of the authors is geared towards human trust in `advanced' technology, this survey we will take a more holistic view and use the term \textit{Artificially Intelligent Agent (AIA)} to encompass a broad range of technologies that can be considered `autonomous'. This is done in order to provide generally applicable definitions and insights. 

    To this end, an artificially intelligent system needs to possess at least some of the capabilities shown in Figure~\ref{fig:AIcapabilities}~\cite{Russell2010-wv,Nilsson2009-rp,Luger2008-vf}. Some might argue that it is also necessary to add other categories like creativity and social intelligence~\cite{Tao2005-kh}. 
    \brettcomm{SEEMS TO DETRACT---}Some of these categories are also not clearly separable; for instance, where does the capability to `plan' end, and `reasoning' begin? Nevertheless, these capabilities are conceptually useful in defining an AIA:     
    \begin{description}
        \item[Artificially Intelligent Agent (AIA):] an agent that acts on an internally/externally generated goal, and possesses, to some extent, at least one of the capabilities shown in Fig.~\ref{fig:AIcapabilities}.
    \end{description}

	\begin{figure}[htbp]
    	\centering
     	\includegraphics[width=0.55\textwidth]{Figures/AI_capabilities}
    	\caption{List of possible AIA capabilities.}
        \label{fig:AIcapabilities}
    \end{figure}

    The broad range of AIAs implied by this definition is most usefully viewed in terms of scope and adaptability. Scope refers to the range of possible applications for an AIA: does it have a small number of specialized application, or can it be used in many different applications? Adaptability refers to the ability of the AIA to become better at executing its goal over time. Low adaptability has often been associated with `weak AI' whereas high adaptability is often associated with `strong AI'.  Figure~\ref{fig:StrongWeak} depicts these axes for some (real and fictitious) AIAs.

	\begin{figure}[htbp]
    	\centering
     	\includegraphics[width=0.7\textwidth]{Figures/strong_weak_narrow_broad.pdf}
    	\caption{Illustration of the range of systems encompassed by the AIA definition. Horizontal axis reflects the scope of the AIA, the vertical axis reflects the adaptability of the AIA.}
        \label{fig:StrongWeak}
    \end{figure}

Arguably, we might just as well use the term `artificial intelligence' (AI) instead of AIA. However, `AI' carries too much ambiguity (in its fullest meaning, it would possess all capabilities from Figure~\ref{fig:AIcapabilities}, and more). AIA allows the broad inclusion of \emph{any} system in the adaptability/scope plane. The research discipline of machine learning (ML) is a subset of the AI research landscape. Individual ML algorithms might be thought of as being a narrowly scoped AI that is contained within only one of the AIA capabilities. 

One might also question the need to define AIAs in the first place. This is to aid in the search for and understanding of assurances. As will be shown later, different methods of assurance can be found over the entire range of AIAs, so that an automation system such as a factory robot might be able to use similar assurances -- or more generally, similar principles of assurance -- as might a self-driving car, and vice-versa. The capabilities of AIAs (Fig.~\ref{fig:AIcapabilities}) are the sources of assurances; in other words, assurances cannot exist without some grounding set of AIA capabilities. 

This definition, while broad, is still useful because it encompasses many of the systems that are typically described as `artificially intelligent'. More importantly, many of the assurances that exist for the simplest AIAs (e.g. chi-square consistency tests for a Kalman filter state estimator) can be adapted/generalized for use in more advanced AIAs. In other words, the proposed definition of AIAs sets an \emph{appropriate scope} for the bodies of research that are likely to have investigated assurances and assurance principles that can be generalized/extended to any intelligent computing system. The definition of AIAs and their range of capabilities also helps to understand and establish what kinds of assurances might be needed in future systems. For example, assurances from an AIA that can only carry out planning tasks will probably differ in design and/or application from assurances from an AIA that can only carry out perception tasks. 
