\subsection{Assurances} \label{sec:assurances}
    The term assurances was introduced in the previous section as the name by which feedback will be known in a human-AIA trust relationship. As assurances are the main topic of this paper, and have received little attention in trust literature, a more detailed definition and discussion is merited. The term `assurances' is perhaps earliest used in the context of human-AIA relationships by \citet{Sheridan1984-kx}. \citet{McKnight2001-fa} allude to this kind of feedback in an e-commerce relationship as `Web Vendor Interventions' and mention some possible actions that might be used in that specific application. They go as far as making a diagram that indicates that these interventions could affect the `Trusting Beliefs', `Trusting Intentions', and `Trust-Related Behaviors' (see Figure~\ref{fig:UserTrust}) of an online human user. \citet{Corritore2003-gx} refer to assurances as `trust cues' that can influence how online users trust e-commerce vendors. \citet{Lee2004-pv} discuss `display characteristics', which are methods by which an autonomous systems can communicate information to an operator. More recently, and formally, \citet{Lillard2016-yg} defined the term `assurance'; we have adapted their definition to be more general:    
    \begin{description}
        \item [Assurance:] A property or behavior of an AIA that affects a user's trust. This effect can lead to increased \emph{or} decreased trust. 
    \end{description}

    Most researchers familiar with the fields of AI, ML, data science, and robotics will recognize terms like \emph{interpretable}, \emph{comprehensible}, \emph{transparent}, \emph{verified and validated} (V\&V), \emph{certified}, and \emph{explainable AI}, with respect to the models or performance of a designed system. A key claim of this paper is that, \textbf{from a high level, all of these approaches have the same underlying aim: for a user to be able to trust an AIA to operate in a certain way, and based on that trust behave appropriately towards the AIA}. These methods are therefore means of assuring or calculating assurances to be given to a human user. 
Refined concepts about assurances are presented in Section~\ref{sec:synthesis} which elaborates on Figure~\ref{fig:SimpleTrust_one_way}. There, important concepts and considerations for designing assurances are reviewed. 
