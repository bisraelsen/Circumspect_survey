\subsection{Assurances}
    The term assurances was introduced in the previous section as the name by which feedback will be known in a human-AIA trust relationship. As assurances are the main topic of this paper, and are have received very little attention in trust literature, a more detailed definition and discussion is merited.

    \citet{McKnight2001-fa} allude to this kind of feedback in an e-commerce relationship as `Web Vendor Interventions' and mention some possible actions that might be used in that specific application. They go as far as making a diagram that indicates that these interventions could affect the `Trusting Beliefs', `Trusting Intentions', and `Trust-Related Behaviors' (see Figure \ref{fig:UserTrust}).

    \citet{Corritore2003-gx} refers to assurances as `trust cues' that can influence how online users trust vendors in an e-commerce setting. \citet{Lee2004-pv} discuss `display characteristics', which are methods by which an autonomous can communicate information to an operator.
    
    The term assurances is perhaps earliest used in the context of human-automation relationships by \citet{Sheridan1984-kx}. More recently, and formally, \citet{Lillard2016-yg} defined the term `assurances' as (modified to fit the AIA terminology):
    
    \begin{description}
        \item [Assurances:] The AIA's ability to affect the user's trust. As used here, the term is not intended to have a positive or negative connotation -- assurances can decrease trust.
    \end{description}

    \begin{itemize}
        \item source and target
        \item component/composite
        \item tutoring and telling (\textbf{static and dynamic??})
        \item explicit/implicit
    \end{itemize}

    Discuss \citet{Lacave2002-cu}, all of their categories are applicable to designing assurances.

    Most familiar with the fields of AI, ML, data science, and robotics will recognize terms like \emph{interpretable}, \emph{comprehensible}, \emph{transparent}, \emph{verification and validation}, \emph{certified}, \emph{explainable AI}, with respect to the models or performance of a designed system. A key claim of this paper is that from a high level all of these terms have the same aim: for a user to be able to trust the designed system to operate in a certain way, and then act appropriately based on that trust. Those actions might include re-design, as well as adjusting TRBs.

    Assurances and `interpretability' are delicately linked. In fact, interpretability would be classified as one embodiment of an assurance. This relationship is highlighted by \citet{Vellido2012-nm} where he illustrates interpreting an AIA as part of the knowledge discovery process.

    Figure \ref{fig:Assurance_classes} shows the hierarchy of proposed assurance classes. The categories mirror those of the trust model proposed by \citet{McKnight2001-fa}, but with the emphasis on what an AIA has the ability to most readily influence (and consequently where most research is found). The boxes with the beveled corner identify and define the different classes of assurances. All classes are included here for completeness and generality. Although, while it is hypothetically possible for an AIA to influence a persons general `Trusting stance' given enough time\footnote{One might imagine an AIA that specifically speaks to the human about the benefits or drawbacks about trusting even though there might not be evidence to do so, similar to the role a counselor might play}, the gray boxes are not considered further in this survey, as practically no direct research exists in the realm of human-AIA relationships.

    \subsubsection{Source-Target Classification}
    It is convenient to refer to assurances by way of their source and target. More specifically their source AIA behavior (see Figure \ref{fig:AIcapabilities}) and their user trust target component (see figure \ref{fig:Asurane_classes}. Intuitively, there may be a set of different algorithms that are useful for making assurances that convey information about planning to the competence dimension of the user's trust. It is easier to refer to these assurances in terms of their source and target. So, for this example that class of algorithms would be the `planning-competence' class.

    The source AIA capability for an assurance might be most easily thought of by the algorithm it has to operate on. If the assurance operates on, or comes from, a planning algorithm then it would be considered a `planning' source. If the algorithm operated on data for learning patterns, the it would be considered a `learning' source.

    The trust dimensions shown in Figure \ref{fig:Assurance_classes} are the possible targets for assurances. The categories mirror those of the trust model proposed by \citet{McKnight2001-fa}, but with the emphasis on what an AIA has the ability to most readily influence (and consequently where most research is found). The boxes with the beveled corner identify and define the different classes of assurances. All classes are included here for completeness and generality. Although, while it is hypothetically possible for an AIA to influence a persons general `Trusting stance' given enough time\footnote{One might imagine an AIA that specifically speaks to the human about the benefits or drawbacks about trusting even though there might not be evidence to do so, similar to the role a counselor might play}, the gray boxes are not considered further in this survey, as practically no direct research exists in the realm of human-AIA relationships.
    
    Not only is the source-target notation useful shorthand for communicating about the purpose of the assurance, but it is useful in classifying the range of assurance algorithms that exist. There may also be a class of algorithms that span multiple source-target capabilities. For example there may be a kind of algorithm that can give a `learning-competence' assurance, as well as a `planning-competence' assurance. This is especially true since many of the AIA capabilities can overlap. Also, the effects of assurances cannot be guaranteed to affect only one trust dimension (see section \ref{sec:imprecise}).

    \subsubsection{Component and Composite Assurances}
Assurances can be either component or composite. This was seen a little through the survey. The definitions are as follows:

\begin{description}
    \item [Component Assurance:] An assurance that originates from a single AIA capability source, and targets a single trust dimension target.
    \item [Composite Assurance:] The combination of more than one component assurance into a single assurance. 
\end{description}

\begin{figure}[!htbp]
    \centering
    \includegraphics[width=0.9\textwidth]{Figures/Assurance_component_composite.pdf}
    \caption{Figure illustrating the difference between component and composite assurances. The existence of multiple assurances does not imply a composite assurances, rather the combination of multiple component assurances into a single assurance constitutes a composite assurance.}
    \label{fig:assurance_mapping}
\end{figure}

Figure \ref{fig:assurance_mapping} illustrates the concepts of component and composite assurances.

\paragraph{Component Assurances:} Component assurances are perhaps the most well researched in the existing literature. This is likely because several verified component assurances are the predecessors to composite ones. A component assurance might include displaying the confidence of a classification prediction, or visualizing a model as discussed in section \ref{sec:q2}.

\paragraph{Composite Assurances:} Composite assurances are assurances that are built of several components. A notable example is the work by \citet{Aitken2016-cv} who propose a measurement called `self-confidence', applicable to Partially Observable Markov Decision Processes (POMDPs). This metric combines five component assurances into a single composite assurance that is meant to distill the information into a value that a novice operator could understand easily. This paper was discussed in more detail in \ref{sec:q2}. 

    \subsubsection{Tutoring vs Telling} \label{sec:teach_tell}
    This is also a point rarely seen in the literature surveyed, but critical in the context of different users and long-term human-AIA interaction. We suggest that assurances can also be classified by whether they consider tutoring (or leading) the user to help them understand, or whether they just produce a static and unchanging value regardless of the user, their experience, or their expertise. This is a point mentioned (in terms of explainability) by \citet{Lacave2002-cu}, and \citet{Lacher2014-yc}.

    A tutoring assurance would be a planned, dynamic, sequence of assurances that would change in time to adapt to the user's needs (as discussed in section \ref{sec:consider_human}. This might include modification of assurances to help a user avoid boredom, or to use the system differently in varying circumstances. For example the first time an autonomous vehicle encounters snow with a certain user, it might take time to give special assurances. Or a user that is so used to an AIA that its TRBs begin to drift to disuse, and the AIA gives a special assurance to correct that.

    It isn't surprising that, to our knowledge, no research has been done with respect to tutoring a user in a trust relationship. This is a complex problem to address that would involve understanding how different users learn, and what an appropriate strategy would be to teach them to have appropriate TRBs. However, a rich resource (not investigated in this paper) would be the work on tutoring systems. There is definitely an open area for research that investigates the advantages of tutoring assurances versus those of telling assurances, and how easy they are to implement in contrast to the added time and effort needed to design tutoring assurances.

    \subsubsection{Explicit and Implicit Assurances}
\citet{Sheridan1984-kx} briefly alluded to the existence of explicit and implicit assurances when they discussed the nature of how humans behave when working with automated systems. They suggested that the operator's perception of the automated system can be effected by `performance' and its `reports on its own performance'. The terms are more formally defined as,

\begin{description}
    \item [Explicit:] Assurances that are purposefully given to affect the trust of a user.
    \begin{itemize}
        \item Legible motion \cite{Dragan2013-wd}, which is motion calculated with the intent of being more understandable by a human
        \item $R^2$ value, gives some indication of how well the regression accounts for the variance of the data
    \end{itemize}
    \item [Implicit:] All other assurances that aren't explicit.
    \begin{itemize}
        \item Reliability in completing a task. Generally, the object of success is not to affect the user's trust (although this is a nice side-effect).
        \item The way an autonomous vehicle appears. For example something that looks neat will have a different effect on trust, than an AIA with wires dragging on the ground. 
    \end{itemize}
\end{description}

It is important for designers of AIAs to be aware of both, but anything that is consciously designed with the goal of affecting trust is automatically an explicit assurance, from the perspective of the AIA. Another way of stating the ideas is that trust relationships between humans and AIAs will form, but all assurances will be implicit if designers do not consciously consider the trust relationship. 


    However, for the case of calibrating TRBs it is proposed that the classes should be constrained as shown in Figure \ref{fig:Assurance_classes}. An AI that is trying to calibrate the TRBs of a user (as opposed to just influencing them to its own benefit), should not actively try to convince the user that it should have faith in autonomy, or that it is benevolent, or that the institutional system should be trusted.

    \subsubsection{The Uncertain Nature of Assurances}
        Due to the nature of trust (and humans in general), a single assurance might be targeted at influencing the competence dimension of trust, but it may also have affects on other dimensions. As an example an assurance aimed at influencing Predictability may also have an affect on the Probability of Depending.

        Besides being difficult to separate, each user is different. Thus no assurance will work identically when given to two separate users. This makes it difficult to have precise effects on user trust behaviors.

        \textbf{I am not sure how I want this argument to go, I want to highlight that it is theoretically possible to have some affect on each of these attributes, but that some are more practical. Two main things need to be considred 1) time-scale (how long will it take to make a noticeable change), and 2) what SHOULD be influenced in order to appropriately calibrate TRBs (it probably isn't acceptable to lie in order to manipulate a user's trust)?}
