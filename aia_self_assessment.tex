%%(i) assurance argument: what specifically is the assurance signal? 
%%(ii) what is mechanism for generating appropriate TRBs? 
%%(iii) how can designers build/exploit for AIA assurances, i.e. what techniques available for ...?: 
%%(a) ...  
%%(b) ...
%%(c) ...

\subsection{AIA Self-Assessment} \label{sec:aia_self_assessment}

\subsubsection{Common Approaches:}
The literature in this category can be split into two high-level categories. The first concern is for an AIA to be able to account for its uncertanties (i.e. uncertainty in its models of its world, sensors, and capabilities). The second is to attempt to reduce its complex, `uninterpretable', capabilities into more interpretable ones.

Here the AIA makes an active attempt at processing data and making information available to the user. This is done in a post-hoc manner, or in a way such that the quantification of uncertainty is more supplemental, rather than integral, to the main functions of the AIA.

\paragraph{Quantify Uncertainty} \label{sec:QU}

Although active learning does not explicitly consider safety, the underlying approaches can be useful because active learners need to be able to search the problem space to reduce uncertainty; this requires an internal representation of uncertainty. The applications surveyed here are all mainly related to image classification and robotics. In the context of image classification, \citet{Paul2011-vr} introduced `perplexity' as a metric that represents uncertainty in predicting a single class and is used to select the `most perplexing' images for further learning. There have also been several attempts to use Gaussian processes (GPs) to actively learn and assign probabilistic classifications \cite{MacKay1992-sp,Triebel2016-kj,Triebel2013-ow,Triebel2013-ku,Grimmett2013-gj,Grimmett2016-yc,Berczi2015-rd,Dequaire2016-kh}. As with perplexity-based classifiers, the key insight is that if a classifier possesses a measure of uncertainty, then that uncertainty can be used for efficient instance searching, comparison, and learning, as well as reporting a measure of confidence to users. The key property of GPs to this end is their ability to produce output confidence/uncertainty estimates that grow more uncertain away from the training data. This information can be readily assessed and conveyed to users, even in high-dimensional problems. This property has also found much use in other AIA active learning problems, e.g. Bayesian optimization \cite{Snoek2012-tt, Brochu2010-tj,Israelsen2017-zb}. 

\citet{Choi2017-th} investigates how mixture density networks (MDNs)---neural networks that learn parameters of a Gaussian mixture distributions---can be used to help a controller switch modes based on the MDN's prediction of 

Bayesian neural networks (BNNs) are a method by which we can have insight into the uncertainty of a neural network model. Using BNNs \citet{Kendall2017-ry}, in the context of computer vision, also use deep BNNs to help visualize epistemic (input) and aleatoric (model) uncertainty for each pixel of an image. 

Similarly \citet{Kahn2017-vy} use deep BNNs to learn about the probability (with uncertainty) of an autonomous vehicle colliding in an environment given its current state, observations, and sequence of controls. Using this model they formulate a `velocity-dependent collision cost' that is used for model-based reinforcement learning. With this approach the vehicle naturally proceeds slowly when there is an elevated risk of collision. \brettcom{not sure if this goes here, or in the `value alignment' section\ldots it goes here if i downplay the built-in nature of the behavior, and instead focus on the ability to quantify uncertainty}

An AIA that can predict its performance on different tasks can provide assurances about competence, predictability, and the situational normality of a given task. Several authors have worked to improve this ability in visual classification \cite{Zhang2014-he,Gurau2016-hs,Churchill2015-ei,Kaipa2015-hy}. 
For example, to ensure that visual classifiers don't fail silently in novel scenarios, 
\citet{Zhang2014-he} learned models of errors on training images to predict errors on test images. 
\citet{Kaipa2015-hy} consider 3D visual classification of assembly line parts for robotic pick and place tasks, and develop statistical goodness-of-fit tests to estimate the likelihood that robots can use their sensors to find parts matching desired ones. %To accomplish this they apply the `Iterative Closest Point' (ICP) method, to match a point cloud measurement of the part with a ground-truth 3D model of the part. 
These approaches allow the AIA to assess capability and present appropriate assurances to users, though without any formal notions of trust. 

Models and logic are not trustworthy by themselves; they may be flawed to begin with, or become invalid when certain assumptions or specifications are violated. Thus, there is great interest in providing assurances that the models and assumptions underlying different AIA processes are in fact sound. \citet{Laskey1991-mf} -- with the intention of communicating model validity to users of `probability-based decision aids' -- notes that it is infeasible to perform a decision-theoretic calculation to determine if model revision is necessary. 
She presents a class of theoretically justified model revision indicators which are based on the idea of constructing a computationally simple alternate model and then initiating model revision if the likelihood ratio of alternate model becomes too large (see also \citet{Zagorecki2015-qy,Habbema1976-xd} --these ideas also provide a potential basis for the `model validity' machine self-confidence factors from Quadrant II).
\citet{Ghosh2016-dl}  present `model repair' and `data repair' strategies that can be used when the current model doesn't match the observed data, at which point the model and data can be repaired, and control actions can be replanned in order to conform with the formal method specifications. One challenge is how the `trustable' constraints should be identified, as this places a strong burden on the certifying authorities and system designer to foresee all possible failures.


\paragraph{Reduce Complexity} \label{sec:reduce_complexity}
Representations within an AIA are often complex. Sometimes using inherently less complex, `interpretable models' (as discussed in \ref{sec:interp_models}), is the most straight forward way to address this challenge. However, in some cases it is desirable to maintain complex, less interpretable representations (e.g. for performance reasons) and then reduce the inherent complexities (possibly post-hoc) to aid human users.

One typical approach is to generate explanations, but how should explanations be provided? There are also considerations regarding whether explanations should occur by two-way interaction between system and user, by natural language interaction, or by probabilities. Some of the answers to these questions lie more in the realm of cognitive science. Still, natural language and other communication modalities could be used~\cite{Hayes2017-nt}. Specifically, \citet{Olah2018-rp} investigate how predictions of NNs can be explained through visualizing how different parts of the network respond to certain images. They propose combining several different approaches to get a holistic view of the NN behavior. Specifically, they use feature visualization (what a neuron is looking for), and attribution (how it affects the output).

There are several classes of explanations. \citet{Abdollahi2018-uw} propose three in the context of collaborative filtering: `neighbor style' (explanation based on examples from similar situations), `influence style' (present the most influential items that led to a certain model output), and `keyword style' (identify common features between user keywords, and content). \citet{Otte2013-oo} and \citet{Ribeiro2016-uc} implement analagous ideas in the realm of safe ML and interpreting classifiers respectively. 
\citet{Huang2017-lk} use `algorithmic teaching' (see~\cite{Balbach2009-jw}) as inspiration for helping human users learn a robot's true objectives. Algorithmic teaching involves having a model of a students learning algorithm, and then presenting training examples to allow the student to learn a target model. In this case the `student' is the human user, and the `teacher' is the robot that is trying to teach the human its own objective function by presenting a set of (optimal) training examples. Here we would consider the robot's training examples as assurances.

Another consideration is whether an explanation is meant to be descriptive or aimed at ensuring comprehension, as well as whether explanations need to be on a macro or micro scale relative for parts of the Bayesian network (similar to globally/locally interpretable learned models \cite{Ruping2006-xj}). 
\citet{Lacave2002-cu} address the AIA reduction of complexity from the perspective of explaining probabilistic inference in Bayesian networks---specifically, \emph{how} and \emph{why} a Bayesian network reaches a conclusion given some imputed evidence. 
They present three properties of explanation: 1) content (what to explain), 2) communication (how to explain), and 3) adaptation (how to adapt based on who the user is). Several key points for designing assurances arise from considering the differences between explaining evidence (i.e. data), the model (i.e. the Bayesian network itself), or the reasoning (i.e. the inference process).

\citet{Aitken2016-cv} propose a metric called `machine self-confidence' for providing users with better insight into autonomous decision making under uncertainty. This insight comes from breaking down the complex influences of uncertainty on the decision making process and presenting them to the user in a simple way. Self-confidence is defined as the machine's own perception of its ability to carry out tasks in light of uncertainties in its knowledge of the world, its own self/states, and its reasoning process and execution abilities. In this sense, self-confidence is an AIA's metacognitive assessment of its own behavior and `competency boundaries'. A computational measure for POMDP-based autonomous planning is defined from five component assurances (which are fairly general and applicable to most other kinds of planners): 1) Model Validity, 2) Expected Outcome Assessment, 3) Solver Quality, 4) Interpretation of User Commands, and 5) Past Performance. 
The key idea behind this set of measures is to assess where and when approximations required for planning under uncertainty are expected to break down. Model validity attempts to quantify the validity of a model within the current situation. The expected outcome assessment uses the distribution over rewards to indicate how beneficial or detrimental the outcome is likely to be. Solver quality quantifies how a specific POMDP solver is likely to perform in the given problem setting (i.e. how close to optimal the solution policy and approximate solution policy can get). The interpretation of commands component is meant to quantify how well the objective has been interpreted (i.e. how sure is the AIA that it correctly interpreted mission specifications into relevant tasks and suitable goals). Finally, past performance is meant to add in empirical experience from past missions, in order to make up for theoretical oversights and account for learning-based processes.

\citet{Aitken2016-cv} proposes that self-confidence values could, for instance, be reported as a single value between $-1$ (complete lack of confidence in achieving mission objectives) and $1$ (complete confidence in achieving mission objectives); a self-confidence value of $0$ reflects total uncertainty. Each of the component assurances could be useful on its own, but the composite `sum' of the factors is meant to distill the information from the five different areas, so that a (possibly novice) user can quickly and easily evaluate the ability of the AIA to perform in a given situation. Currently, only two of the five metrics (Expected Outcome Assessment, and Solver Quality) have been developed quantitatively, but there is continuing work on the other metrics and they plan to perform human experiments to evaluate the usefulness of the self-confidence metrics for AIAs. Other approaches for computing and communicating AIA self-confidence have also been proposed for more specific applications \cite{Hutchins2015-if, Kaipa2015-hy, Zagorecki2015-qy, Kuter2015-qh}. 


\subsubsection{Grounding Example:}
In the case of the `VIP Escort' problem (described in Section~\ref{sec:mot_example}), self-assessment might be used as an assurance in the following way:

We make the following assumptions

\begin{itemize}
    \item The UGV is about to being an attempt to escape the road-network
    \item The UGV is using the `solver quality' metric mentioned by \citet{Aitken2016-fb}
    \item The operator has access to an interface where they can view the self-confidence metric calculated by the UGV
\end{itemize}

Before the UGV begins its attempt it is able to assess its `solver quality' given the specific, unseen road-network, based on similarities between the current network and ones that it has encountered before. The UGV reports that it has high confidence in its solver quality, and the operator is assured that they can trust the solver in this situation.

\paragraph{\textbf{Discussion of Example:}} In this case the UGV is able to assure the operator of the quality of the solver in the specific road-network. Generally, the UGV reduced what could be a very complex analysis into a simple format for the operator to interpret. This is in contrast to the operator viewing policies, models, algorithms, and complex probability distributions.
